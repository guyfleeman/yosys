\documentclass[oneside,a4paper]{book}

\usepackage[T1]{fontenc}   % required for luximono!
\usepackage{lmodern}
\usepackage[scaled=0.8]{luximono}  % typewriter font with bold face

% To install the luximono font files:
% getnonfreefonts-sys --all        or
% getnonfreefonts-sys luximono
%
% when there are trouble you might need to:
% - Create /etc/texmf/updmap.d/99local-luximono.cfg
%   containing the single line: Map ul9.map
% - Run update-updmap followed by mktexlsr and updmap-sys
%
% This commands must be executed as root with a root environment
% (i.e. run "sudo su" and then execute the commands in the root
% shell, don't just prefix the commands with "sudo").

% formats the text according the set language
\usepackage[english]{babel}
\usepackage[table,usenames]{xcolor}
% generates indices with the "\index" command
\usepackage{makeidx}
% enables import of graphics. We use pdflatex here so do the pdf optimisation.
%\usepackage[dvips]{graphicx}
\usepackage[pdftex]{graphicx}
\usepackage{pdfpages}
% includes floating objects like tables and figures.
\usepackage{float}
% for generating subfigures with ohne indented captions
\usepackage[hang]{subfigure}
% redefines and smartens captions of figures and tables (indentation, smaller and boldface)
\usepackage[hang,small,bf,center]{caption}
% enables tabstops and the numeration of lines
\usepackage{moreverb}
% enables user defined header and footer lines (former "fancyheadings")
\usepackage{fancyhdr}
% Some smart mathematical stuff
\usepackage{amsmath}
% Package for rotating several objects
\usepackage{rotating}
\usepackage{natbib}
\usepackage{epsf}
\usepackage{dsfont}
\usepackage[algochapter, boxruled, vlined]{algorithm2e}
%Activating and setting of character protruding - if you like
%\usepackage[activate,DVIoutput]{pdfcprot}
% If you really need special chars...
\usepackage[latin1]{inputenc}
% Hyperlinks
\usepackage[colorlinks,hyperindex,plainpages=false,%
pdftitle={Yosys Manual},%
pdfauthor={Clifford Wolf},%
%pdfkeywords={keyword},%
pdfpagelabels,%
pagebackref,%
bookmarksopen=false%
]{hyperref}
% For the two different reference lists ...
\usepackage{multibib}
\usepackage{multirow}
\usepackage{booktabs}
\usepackage{pdfpages}

\usepackage{listings}
\usepackage{pifont}
\usepackage{skull}
% \usepackage{draftwatermark}

\usepackage{tikz}
\usetikzlibrary{calc}
\usetikzlibrary{arrows}
\usetikzlibrary{scopes}
\usetikzlibrary{through}
\usetikzlibrary{shapes.geometric}

\lstset{basicstyle=\ttfamily}

\def\B#1{{\tt\textbackslash{}#1}}
\def\C#1{\lstinline[language=C++]{#1}}
\def\V#1{\lstinline[language=Verilog]{#1}}

\newsavebox{\fixmebox}
\newenvironment{fixme}%
{\newcommand\colboxcolor{FFBBBB}%
\begin{lrbox}{\fixmebox}%
\begin{minipage}{\dimexpr\columnwidth-2\fboxsep\relax}}
{\end{minipage}\end{lrbox}\textbf{FIXME: }\\%
\colorbox[HTML]{\colboxcolor}{\usebox{\fixmebox}}}

\newcites{weblink}{Internet References}

\setcounter{secnumdepth}{3}
\makeindex

\setlength{\oddsidemargin}{4mm}
\setlength{\evensidemargin}{-6mm}
\setlength{\textwidth}{162mm}
\setlength{\textheight}{230mm}
\setlength{\topmargin}{-5mm}

\setlength{\parskip}{1.5ex plus 1ex minus 0.5ex}
\setlength{\parindent}{0pt}

\lstdefinelanguage{liberty}{
	morecomment=[s]{/*}{*/},
	morekeywords={library,cell,area,pin,direction,function,clocked_on,next_state,clock,ff},
	morestring=[b]",
}

\lstdefinelanguage{rtlil}{
	morecomment=[l]{\#},
	morekeywords={module,attribute,parameter,wire,memory,auto,width,offset,size,input,output,inout,cell,connect,switch,case,assign,sync,low,high,posedge,negedge,edge,always,update,process,end},
	morestring=[b]",
}



\begin{document}

\fancypagestyle{mypagestyle}{%
\fancyhf{}%
\fancyhead[C]{\leftmark}%
\fancyfoot[C]{\thepage}%
\renewcommand{\headrulewidth}{0pt}%
\renewcommand{\footrulewidth}{0pt}}
\pagestyle{mypagestyle}

\thispagestyle{empty}
\null\vfil

\begin{center}
\bf\Huge Yosys Manual

\bigskip
\large Clifford Wolf
\end{center}

\vfil\null
\eject

\chapter*{Abstract}
Most of today's digital design is done in HDL code (mostly Verilog or VHDL) and
with the help of HDL synthesis tools.

In special cases such as synthesis for coarse-grain cell libraries or when
testing new synthesis algorithms it might be necessary to write a custom HDL
synthesis tool or add new features to an existing one. In these cases the
availability of a Free and Open Source (FOSS) synthesis tool that can be used
as basis for custom tools would be helpful.

In the absence of such a tool, the Yosys Open SYnthesis Suite (Yosys) was
developed. This document covers the design and implementation of this tool.
At the moment the main focus of Yosys lies on the high-level aspects of
digital synthesis. The pre-existing FOSS logic-synthesis tool ABC is used
by Yosys to perform advanced gate-level optimizations.

An evaluation of Yosys based on real-world designs is included. It is shown
that Yosys can be used as-is to synthesize such designs. The results produced
by Yosys in this tests where successfully verified using formal verification
and are comparable in quality to the results produced by a commercial
synthesis tool.

\bigskip

This document was originally published as bachelor thesis at the Vienna
University of Technology \cite{BACC}.

\chapter*{Abbreviations}
\begin{tabular}{ll}
AIG         & And-Inverter-Graph \\
ASIC        & Application-Specific Integrated Circuit \\
AST         & Abstract Syntax Tree \\
BDD         & Binary Decision Diagram \\
BLIF        & Berkeley Logic Interchange Format \\
EDA         & Electronic Design Automation \\
EDIF        & Electronic Design Interchange Format \\
ER Diagram  & Entity-Relationship Diagram \\
FOSS        & Free and Open-Source Software \\
FPGA        & Field-Programmable Gate Array \\
FSM         & Finite-state machine \\
HDL         & Hardware Description Language \\
LPM         & Library of Parameterized Modules \\
RTLIL       & RTL Intermediate Language \\
RTL         & Register Transfer Level \\
SAT         & Satisfiability Problem \\
% SSA       & Static Single Assignment Form \\
VHDL        & VHSIC Hardware Description Language \\
VHSIC       & Very-High-Speed Integrated Circuit \\
YOSYS       & Yosys Open SYnthesis Suite \\
\end{tabular}

\tableofcontents

\include{CHAPTER_Intro}
\include{CHAPTER_Basics}
\include{CHAPTER_Approach}

\chapter{Implementation Overview}
\label{chapter:overview}

Yosys is an extensible open source hardware synthesis tool. It is aimed at
designers who are looking for an easily accessible, universal, and
vendor-independent synthesis tool, as well as scientists who do research in
electronic design automation (EDA) and are looking for an open synthesis
framework that can be used to test algorithms on complex real-world designs.

Yosys can synthesize a large subset of Verilog 2005 and has been tested with a
wide range of real-world designs, including the OpenRISC 1200 CPU
\citeweblink{OR1200}, the openMSP430 CPU \citeweblink{openMSP430}, the
OpenCores I$^2$C master \citeweblink{i2cmaster} and the k68 CPU \citeweblink{k68}.

As of this writing a Yosys VHDL frontend is in development.

Yosys is written in C++ (using some features from the new C++11 standard). This
chapter describes some of the fundamental Yosys data structures. For the sake
of simplicity the C++ type names used in the Yosys implementation are used in
this chapter, even though the chapter only explains the conceptual idea behind
it and can be used as reference to implement a similar system in any language.

\section{Simplified Data Flow}

Figure~\ref{fig:Overview_flow} shows the simplified data flow within Yosys.
Rectangles in the figure represent program modules and ellipses internal
data structures that are used to exchange design data between the program
modules.

Design data is read in using one of the frontend modules. The high-level HDL
frontends for Verilog and VHDL code generate an abstract syntax tree (AST) that
is then passed to the AST frontend. Note that both HDL frontends use the same
AST representation that is powerful enough to cover the Verilog HDL and VHDL
language.

The AST Frontend then compiles the AST to Yosys's main internal data format,
the RTL Intermediate Language (RTLIL). A more detailed description of this format
is given in the next section.

There is also a text representation of the RTLIL data structure that can be
parsed using the ILANG Frontend.

The design data may then be transformed using a series of passes that all
operate on the RTLIL representation of the design.

Finally the design in RTLIL representation is converted back to text by one
of the backends, namely the Verilog Backend for generating Verilog netlists
and the ILANG Backend for writing the RTLIL data in the same format that is
understood by the ILANG Frontend.

With the exception of the AST Frontend, which is called by the high-level HDL
frontends and can't be called directly by the user, all program modules are
called by the user (usually using a synthesis script that contains text
commands for Yosys).

By combining passes in different ways and/or adding additional passes to Yosys
it is possible to adapt Yosys to a wide range of applications. For this to be
possible it is key that (1) all passes operate on the same data structure
(RTLIL) and (2) that this data structure is powerful enough to represent the design
in different stages of the synthesis.

\begin{figure}[t]
	\hfil
	\begin{tikzpicture}
		\tikzstyle{process} = [draw, fill=green!10, rectangle, minimum height=3em, minimum width=10em, node distance=15em]
		\tikzstyle{data} = [draw, fill=blue!10, ellipse, minimum height=3em, minimum width=7em, node distance=15em]
		\node[process] (vlog) {Verilog Frontend};
		\node[process, dashed, fill=green!5] (vhdl) [right of=vlog] {VHDL Frontend};
		\node[process] (ilang) [right of=vhdl] {ILANG Frontend};
		\node[data] (ast) [below of=vlog, node distance=5em, xshift=7.5em] {AST};
		\node[process] (astfe) [below of=ast, node distance=5em] {AST Frontend};
		\node[data] (rtlil) [below of=astfe, node distance=5em, xshift=7.5em] {RTLIL};
		\node[process] (pass) [right of=rtlil, node distance=5em, xshift=7.5em] {Passes};
		\node[process] (vlbe) [below of=rtlil, node distance=7em, xshift=-13em] {Verilog Backend};
		\node[process] (ilangbe) [below of=rtlil, node distance=7em, xshift=0em] {ILANG Backend};
		\node[process, dashed, fill=green!5] (otherbe) [below of=rtlil, node distance=7em, xshift=+13em] {Other Backends};

		\draw[-latex] (vlog) -- (ast);
		\draw[-latex] (vhdl) -- (ast);
		\draw[-latex] (ast) -- (astfe);
		\draw[-latex] (astfe) -- (rtlil);
		\draw[-latex] (ilang) -- (rtlil);
		\draw[latex-latex] (rtlil) -- (pass);
		\draw[-latex] (rtlil) -- (vlbe);
		\draw[-latex] (rtlil) -- (ilangbe);
		\draw[-latex] (rtlil) -- (otherbe);
	\end{tikzpicture}
	\caption{Yosys simplified data flow (ellipses: data structures, rectangles: program modules)}
	\label{fig:Overview_flow}
\end{figure}

\section{The RTL Intermediate Language}

All frontends, passes and backends in Yosys operate on a design in RTLIL\footnote{The {\it Language} in {\it RTL Intermediate Language}
refers to the fact, that RTLIL also has a text representation, usually referred to as {\it Intermediate Language} (ILANG).} representation.
The only exception are the high-level frontends that use the AST representation as an intermediate step before generating RTLIL
data.

In order to avoid reinventing names for the RTLIL classes, they are simply referred to by their full C++ name, i.e.~including
the {\tt RTLIL::} namespace prefix, in this document.

Figure~\ref{fig:Overview_RTLIL} shows a simplified Entity-Relationship Diagram (ER Diagram) of RTLIL. In $1:N$ relationships the arrow
points from the $N$ side to the $1$. For example one RTLIL::Design contains $N$ (zero to many) instances of RTLIL::Module.
A two-pointed arrow indicates a $1:1$ relationship.

The RTLIL::Design is the root object of the RTLIL data structure. There is always one ``current design'' in memory
which passes operate on, frontends add data to and backends convert to exportable formats. But in some cases passes
internally generate additional RTLIL::Design objects. For example when a pass is reading an auxiliary Verilog file such
as a cell library, it might create an additional RTLIL::Design object and call the Verilog frontend with this
other object to parse the cell library.

\begin{figure}[t]
	\hfil
	\begin{tikzpicture}
		\tikzstyle{entity} = [draw, fill=gray!10, rectangle, minimum height=3em, minimum width=7em, node distance=5em, font={\ttfamily}]
		\node[entity] (design) {RTLIL::Design};
		\node[entity] (module) [right of=design, node distance=11em] {RTLIL::Module} edge [-latex] node[above] {\tiny 1 \hskip3em N} (design);

		\node[entity] (process) [fill=green!10, right of=module, node distance=10em] {RTLIL::Process} (process.west) edge [-latex] (module);
		\node[entity] (memory) [fill=red!10, below of=process] {RTLIL::Memory} edge [-latex] (module);
		\node[entity] (wire) [fill=blue!10, above of=process] {RTLIL::Wire} (wire.west) edge [-latex] (module);
		\node[entity] (cell) [fill=blue!10, above of=wire] {RTLIL::Cell} (cell.west) edge [-latex] (module);

		\node[entity] (case) [fill=green!10, right of=process, node distance=10em] {RTLIL::CaseRule} edge [latex-latex] (process);
		\node[entity] (sync) [fill=green!10, above of=case] {RTLIL::SyncRule} edge [-latex] (process);
		\node[entity] (switch) [fill=green!10, below of=case] {RTLIL::SwitchRule} edge [-latex] (case);
		\draw[latex-] (switch.east) -- ++(1em,0) |- (case.east);
	\end{tikzpicture}
	\caption{Simplified RTLIL Entity-Relationship Diagram}
	\label{fig:Overview_RTLIL}
\end{figure}

There is only one active RTLIL::Design object that is used by all frontends,
passes and backends called by the user, e.g.~using a synthesis script. The RTLIL::Design then contains
zero to many RTLIL::Module objects. This corresponds to modules in Verilog or entities in VHDL. Each
module in turn contains objects from three different categories:

\begin{itemize}
\item RTLIL::Cell and RTLIL::Wire objects represent classical netlist data.
\item RTLIL::Process objects represent the decision trees (if-then-else statements, etc.) and synchronization
declarations (clock signals and sensitivity) from Verilog {\tt always} and VHDL {\tt process} blocks.
\item RTLIL::Memory objects represent addressable memories (arrays).
\end{itemize}

\begin{sloppypar}
Usually the output of the synthesis procedure is a netlist, i.e. all
RTLIL::Process and RTLIL::Memory objects must be replaced by RTLIL::Cell and
RTLIL::Wire objects by synthesis passes.
\end{sloppypar}

All features of the HDL that cannot be mapped directly to these RTLIL classes must be
transformed to an RTLIL-compatible representation by the HDL frontend. This includes
Verilog-features such as generate-blocks, loops and parameters.

The following sections contain a more detailed description of the different
parts of RTLIL and rationale behind some of the design decisions.

\subsection{RTLIL Identifiers}

All identifiers in RTLIL (such as module names, port names, signal names, cell
types, etc.) follow the following naming convention: they must either start with
a backslash (\textbackslash) or a dollar sign (\$).

Identifiers starting with a backslash are public visible identifiers. Usually
they originate from one of the HDL input files. For example the signal name ``{\tt \textbackslash sig42}''
is most likely a signal that was declared using the name ``{\tt sig42}'' in an HDL input file.
On the other hand the signal name ``{\tt \$sig42}'' is an auto-generated signal name. The backends
convert all identifiers that start with a dollar sign to identifiers that do not collide with
identifiers that start with a backslash.

This has three advantages:

\begin{itemize}
\item First, it is impossible that an auto-generated identifier collides with
an identifier that was provided by the user.
\item Second, the information about which identifiers were originally
provided by the user is always available which can help guide some optimizations. For example the ``opt\_rmunused''
tries to preserve signals with a user-provided name but doesn't hesitate to delete signals that have
auto-generated names when they just duplicate other signals.
\item Third, the delicate job of finding suitable auto-generated public visible
names is deferred to one central location. Internally auto-generated names that
may hold important information for Yosys developers can be used without
disturbing external tools. For example the Verilog backend assigns names in the form {\tt \_{\it integer}\_}.
\end{itemize}

In order to avoid programming errors, the RTLIL data structures check if all
identifiers start with either a backslash or a dollar sign and generate a
runtime error if this rule is violated.

All RTLIL identifiers are case sensitive.

\subsection{RTLIL::Design and RTLIL::Module}

The RTLIL::Design object is basically just a container for RTLIL::Module objects. In addition to
a list of RTLIL::Module objects the RTLIL::Design also keeps a list of {\it selected objects}, i.e.
the objects that passes should operate on. In most cases the whole design is selected and therefore
passes operate on the whole design. But this mechanism can be useful for more complex synthesis jobs
in which only parts of the design should be affected by certain passes.

Besides the objects shown in the ER diagram in Fig.~\ref{fig:Overview_RTLIL} an RTLIL::Module object
contains the following additional properties:

\begin{itemize}
\item The module name
\item A list of attributes
\item A list of connections between wires
\item An optional frontend callback used to derive parametrized variations of the module
\end{itemize}

The attributes can be Verilog attributes imported by the Verilog frontend or attributes assigned
by passes. They can be used to store additional metadata about modules or just mark them to be
used by certain part of the synthesis script but not by others.

Verilog and VHDL both support parametric modules (known as ``generic entities'' in VHDL). The RTLIL
format does not support parametric modules itself. Instead each module contains a callback function
into the AST frontend to generate a parametrized variation of the RTLIL::Module as needed. This
callback then returns the auto-generated name of the parametrized variation of the module. (A hash
over the parameters and the module name is used to prohibit the same parametrized variation from being
generated twice. For modules with only a few parameters, a name directly containing all parameters
is generated instead of a hash string.)

\subsection{RTLIL::Cell and RTLIL::Wire}

A module contains zero to many RTLIL::Cell and RTLIL::Wire objects. Objects of
these types are used to model netlists. Usually the goal of all synthesis efforts is to convert
all modules to a state where the functionality of the module is implemented only by cells
from a given cell library and wires to connect these cells with each other. Note that module
ports are just wires with a special property.

An RTLIL::Wire object has the following properties:

\begin{itemize}
\item The wire name
\item A list of attributes
\item A width (buses are just wires with a width > 1)
\item Bus direction (MSB to LSB or vice versa)
\item Lowest valid bit index (LSB or MSB depending on bus direction)
\item If the wire is a port: port number and direction (input/output/inout)
\end{itemize}

As with modules, the attributes can be Verilog attributes imported by the
Verilog frontend or attributes assigned by passes.

In Yosys, busses (signal vectors) are represented using a single wire object
with a width > 1. So Yosys does not convert signal vectors to individual signals.
This makes some aspects of RTLIL more complex but enables Yosys to be used for
coarse grain synthesis where the cells of the target architecture operate on
entire signal vectors instead of single bit wires.

In Verilog and VHDL, busses may have arbitrary bounds, and LSB can have either
the lowest or the highest bit index. In RTLIL, bit 0 always corresponds to LSB;
however, information from the HDL frontend is preserved so that the bus will be
correctly indexed in error messages, backend output, constraint files, etc.

An RTLIL::Cell object has the following properties:

\begin{itemize}
\item The cell name and type
\item A list of attributes
\item A list of parameters (for parametric cells)
\item Cell ports and the connections of ports to wires and constants
\end{itemize}

The connections of ports to wires are coded by assigning an RTLIL::SigSpec
to each cell port. The RTLIL::SigSpec data type is described in the next section.

\subsection{RTLIL::SigSpec}

A ``signal'' is everything that can be applied to a cell port. I.e.

\begin{itemize}
\item Any constant value of arbitrary bit-width \\
\null\hskip1em For example: \lstinline[language=Verilog]{1337, 16'b0000010100111001, 1'b1, 1'bx}
\item All bits of a wire or a selection of bits from a wire \\
\null\hskip1em For example: \lstinline[language=Verilog]{mywire, mywire[24], mywire[15:8]}
\item Concatenations of the above \\
\null\hskip1em For example: \lstinline[language=Verilog]|{16'd1337, mywire[15:8]}|
\end{itemize}

The RTLIL::SigSpec data type is used to represent signals. The RTLIL::Cell
object contains one RTLIL::SigSpec for each cell port.

In addition, connections between wires are represented using a pair of
RTLIL::SigSpec objects. Such pairs are needed in different locations. Therefore
the type name RTLIL::SigSig was defined for such a pair.

\subsection{RTLIL::Process}

When a high-level HDL frontend processes behavioural code it splits it up into
data path logic (e.g.~the expression {\tt a + b} is replaced by the output of an
adder that takes {\tt a} and {\tt b} as inputs) and an RTLIL::Process that models
the control logic of the behavioural code. Let's consider a simple example:

\begin{lstlisting}[numbers=left,frame=single,language=Verilog]
module ff_with_en_and_async_reset(clock, reset, enable, d, q);
input clock, reset, enable, d;
output reg q;
always @(posedge clock, posedge reset)
	if (reset)
		q <= 0;
	else if (enable)
		q <= d;
endmodule
\end{lstlisting}

In this example there is no data path and therefore the RTLIL::Module generated by
the frontend only contains a few RTLIL::Wire objects and an RTLIL::Process.
The RTLIL::Process in ILANG syntax:

\begin{lstlisting}[numbers=left,frame=single,language=rtlil]
process $proc$ff_with_en_and_async_reset.v:4$1
	assign $0\q[0:0] \q
	switch \reset
		case 1'1
			assign $0\q[0:0] 1'0
		case
			switch \enable
				case 1'1
					assign $0\q[0:0] \d
				case
			end
	end
	sync posedge \clock
		update \q $0\q[0:0]
	sync posedge \reset
		update \q $0\q[0:0]
end
\end{lstlisting}

This RTLIL::Process contains two RTLIL::SyncRule objects, two RTLIL::SwitchRule
objects and five RTLIL::CaseRule objects. The wire {\tt \$0\textbackslash{}q[0:0]}
is an automatically created wire that holds the next value of {\tt \textbackslash{}q}. The lines
$2 \dots 12$ describe how {\tt \$0\textbackslash{}q[0:0]} should be calculated. The
lines $13 \dots 16$ describe how the value of {\tt \$0\textbackslash{}q[0:0]} is used
to update {\tt \textbackslash{}q}.

An RTLIL::Process is a container for zero or more RTLIL::SyncRule objects and
exactly one RTLIL::CaseRule object, which is called the {\it root case}.

An RTLIL::SyncRule object contains an (optional) synchronization condition (signal and edge-type) and zero or
more assignments (RTLIL::SigSig). The {\tt always} synchronization condition is used to break combinatorial
loops when a latch should be inferred instead.

An RTLIL::CaseRule is a container for zero or more assignments (RTLIL::SigSig)
and zero or more RTLIL::SwitchRule objects. An RTLIL::SwitchRule objects is a
container for zero or more RTLIL::CaseRule objects.

In the above example the lines $2 \dots 12$ are the root case. Here {\tt \$0\textbackslash{}q[0:0]} is first
assigned the old value {\tt \textbackslash{}q} as default value (line 2). The root case
also contains an RTLIL::SwitchRule object (lines $3 \dots 12$). Such an object is very similar to the C {\tt switch}
statement as it uses a control signal ({\tt \textbackslash{}reset} in this case) to determine
which of its cases should be active. The RTLIL::SwitchRule object then contains one RTLIL::CaseRule
object per case. In this example there is a case\footnote{The
syntax {\tt 1'1} in the ILANG code specifies a constant with a length of one bit (the first ``1''),
and this bit is a one (the second ``1'').} for {\tt \textbackslash{}reset == 1} that causes
{\tt \$0\textbackslash{}q[0:0]} to be set (lines 4 and 5) and a default case that in turn contains a switch that
sets {\tt \$0\textbackslash{}q[0:0]} to the value of {\tt \textbackslash{}d} if {\tt
\textbackslash{}enable} is active (lines $6 \dots 11$).

A case can specify zero or more compare values that will determine whether it matches. Each of the compare values
must be the exact same width as the control signal. When more than one compare value is specified, the case matches
if any of them matches the control signal; when zero compare values are specified, the case always matches (i.e.
it is the default case).

A switch prioritizes cases from first to last: multiple cases can match, but only the first matched case becomes
active. This normally synthesizes to a priority encoder. The {\tt parallel\_case} attribute allows passes to assume
that no more than one case will match, and {\tt full\_case} attribute allows passes to assume that exactly one
case will match; if these invariants are ever dynamically violated, the behavior is undefined. These attributes
are useful when an invariant invisible to the synthesizer causes the control signal to never take certain
bit patterns.

The lines $13 \dots 16$  then cause {\tt \textbackslash{}q} to be updated whenever there is
a positive clock edge on {\tt \textbackslash{}clock} or {\tt \textbackslash{}reset}.

In order to generate such a representation, the language frontend must be able to handle blocking
and nonblocking assignments correctly. However, the language frontend does not need to identify
the correct type of storage element for the output signal or generate multiplexers for the
decision tree. This is done by passes that work on the RTLIL representation. Therefore it is
relatively easy to substitute these steps with other algorithms that target different target
architectures or perform optimizations or other transformations on the decision trees before
further processing them.

One of the first actions performed on a design in RTLIL representation in most
synthesis scripts is identifying asynchronous resets. This is usually done using the {\tt proc\_arst}
pass. This pass transforms the above example to the following RTLIL::Process:

\begin{lstlisting}[numbers=left,frame=single,language=rtlil]
process $proc$ff_with_en_and_async_reset.v:4$1
	assign $0\q[0:0] \q
	switch \enable
		case 1'1
			assign $0\q[0:0] \d
		case
	end
	sync posedge \clock
		update \q $0\q[0:0]
	sync high \reset
		update \q 1'0
end
\end{lstlisting}

This pass has transformed the outer RTLIL::SwitchRule into a modified RTLIL::SyncRule object
for the {\tt \textbackslash{}reset} signal. Further processing converts the RTLIL::Process
into e.g.~a d-type flip-flop with asynchronous reset and a multiplexer for the enable signal:

\begin{lstlisting}[numbers=left,frame=single,language=rtlil]
cell $adff $procdff$6
	parameter \ARST_POLARITY 1'1
	parameter \ARST_VALUE 1'0
	parameter \CLK_POLARITY 1'1
	parameter \WIDTH 1
	connect \ARST \reset
	connect \CLK \clock
	connect \D $0\q[0:0]
	connect \Q \q
end
cell $mux $procmux$3
	parameter \WIDTH 1
	connect \A \q
	connect \B \d
	connect \S \enable
	connect \Y $0\q[0:0]
end
\end{lstlisting}

Different combinations of passes may yield different results. Note that {\tt \$adff} and {\tt
\$mux} are internal cell types that still need to be mapped to cell types from the
target cell library.

Some passes refuse to operate on modules that still contain RTLIL::Process objects as the
presence of these objects in a module increases the complexity. Therefore the passes to translate
processes to a netlist of cells are usually called early in a synthesis script. The {\tt proc}
pass calls a series of other passes that together perform this conversion in a way that is suitable
for most synthesis tasks.

\subsection{RTLIL::Memory}

For every array (memory) in the HDL code an RTLIL::Memory object is created. A
memory object has the following properties:

\begin{itemize}
\item The memory name
\item A list of attributes
\item The width of an addressable word
\item The size of the memory in number of words
\end{itemize}

All read accesses to the memory are transformed to {\tt \$memrd} cells and all write accesses to
{\tt \$memwr} cells by the language frontend. These cells consist of independent read- and write-ports
to the memory. Memory initialization is transformed to {\tt \$meminit} cells by the language frontend.
The \B{MEMID} parameter on these cells is used to link them together and to the RTLIL::Memory object they belong to.

The rationale behind using separate cells for the individual ports versus
creating a large multiport memory cell right in the language frontend is that
the separate {\tt \$memrd} and {\tt \$memwr} cells can be consolidated using resource sharing.
As resource sharing is a non-trivial optimization problem where different synthesis tasks
can have different requirements it lends itself to do the optimisation in separate passes and merge
the RTLIL::Memory objects and {\tt \$memrd} and {\tt \$memwr} cells to multiport memory blocks after resource sharing is completed.

The {\tt memory} pass performs this conversion and can (depending on the options passed
to it) transform the memories directly to d-type flip-flops and address logic or yield
multiport memory blocks (represented using {\tt \$mem} cells).

See Sec.~\ref{sec:memcells} for details about the memory cell types.

\section{Command Interface and Synthesis Scripts}

Yosys reads and processes commands from synthesis scripts, command line arguments and
an interactive command prompt. Yosys commands consist of a command name and an optional
whitespace separated list of arguments. Commands are terminated using the newline character
or a semicolon ({\tt ;}). Empty lines and lines starting with the hash sign ({\tt \#}) are ignored.
See Sec.~\ref{sec:typusecase} for an example synthesis script.

The command {\tt help} can be used to access the command reference manual.

Most commands can operate not only on the entire design but also specifically on {\it selected}
parts of the design. For example the command {\tt dump} will print all selected objects
in the current design while {\tt dump foobar} will only print the module {\tt foobar}
and {\tt dump *} will print the entire design regardless of the current selection.

The selection mechanism is very powerful. For example the command {\tt dump */t:\$add
\%x:+[A] */w:* \%i} will print all wires that are connected to the \B{A} port of
a {\tt \$add} cell.  Detailed documentation of the select framework can be
found in the command reference for the {\tt select} command.

\section{Source Tree and Build System}

The Yosys source tree is organized into the following top-level directories:

\begin{itemize}

\item {\tt backends/} \\
This directory contains a subdirectory for each of the backend modules.

\item {\tt frontends/} \\
This directory contains a subdirectory for each of the frontend modules.

\item {\tt kernel/} \\
This directory contains all the core functionality of Yosys. This includes the
functions and definitions for working with the RTLIL data structures ({\tt
rtlil.h} and {\tt rtlil.cc}), the main() function ({\tt driver.cc}), the
internal framework for generating log messages ({\tt log.h} and {\tt log.cc}),
the internal framework for registering and calling passes ({\tt register.h} and
{\tt register.cc}), some core commands that are not really passes ({\tt
select.cc}, {\tt show.cc}, \dots) and a couple of other small utility libraries.

\item {\tt passes/} \\
This directory contains a subdirectory for each pass or group of passes. For example as
of this writing the directory {\tt passes/opt/} contains the code for seven
passes: {\tt opt}, {\tt opt\_expr}, {\tt opt\_muxtree}, {\tt opt\_reduce},
{\tt opt\_rmdff}, {\tt opt\_rmunused} and {\tt opt\_merge}.

\item {\tt techlibs/} \\
This directory contains simulation models and standard implementations for the
cells from the internal cell library.

\item {\tt tests/} \\
This directory contains a couple of test cases. Most of the smaller tests are executed
automatically when {\tt make test} is called. The larger tests must be executed
manually. Most of the larger tests require downloading external HDL source code
and/or external tools. The tests range from comparing simulation results of the synthesized
design to the original sources to logic equivalence checking of entire CPU cores.

\end{itemize}

\begin{sloppypar}
The top-level Makefile includes {\tt frontends/*/Makefile.inc}, {\tt passes/*/Makefile.inc}
and {\tt backends/*/Makefile.inc}. So when extending Yosys it is enough to create
a new directory in {\tt frontends/}, {\tt passes/} or {\tt backends/} with your sources
and a {\tt Makefile.inc}. The Yosys kernel automatically detects all commands linked with
Yosys. So it is not needed to add additional commands to a central list of commands.
\end{sloppypar}

Good starting points for reading example source code to learn how to write passes
are {\tt passes/opt/opt\_rmdff.cc} and {\tt passes/opt/opt\_merge.cc}.

See the top-level README file for a quick {\it Getting Started} guide and build
instructions. The Yosys build is based solely on Makefiles.

Users of the Qt Creator IDE can generate a QT Creator project file using {\tt
make qtcreator}. Users of the Eclipse IDE can use the ``Makefile Project with
Existing Code'' project type in the Eclipse ``New Project'' dialog (only
available after the CDT plugin has been installed) to create an Eclipse project
in order to programming extensions to Yosys or just browse the Yosys code base.



\chapter{Internal Cell Library}
\label{chapter:celllib}

Most of the passes in Yosys operate on netlists, i.e.~they only care about the RTLIL::Wire and RTLIL::Cell
objects in an RTLIL::Module. This chapter discusses the cell types used by Yosys to represent a behavioural
design internally.

This chapter is split in two parts. In the first part the internal RTL cells are covered. These cells
are used to represent the design on a coarse grain level. Like in the original HDL code on this level the
cells operate on vectors of signals and complex cells like adders exist. In the second part the internal
gate cells are covered. These cells are used to represent the design on a fine-grain gate-level. All cells
from this category operate on single bit signals.

\section{RTL Cells}

Most of the RTL cells closely resemble the operators available in HDLs such as
Verilog or VHDL. Therefore Verilog operators are used in the following sections
to define the behaviour of the RTL cells.

Note that all RTL cells have parameters indicating the size of inputs and outputs. When
passes modify RTL cells they must always keep the values of these parameters in sync with
the size of the signals connected to the inputs and outputs.

Simulation models for the RTL cells can be found in the file {\tt techlibs/common/simlib.v} in the Yosys
source tree.

\subsection{Unary Operators}

All unary RTL cells have one input port \B{A} and one output port \B{Y}. They also
have the following parameters:

\begin{itemize}
\item \B{A\_SIGNED} \\
Set to a non-zero value if the input \B{A} is signed and therefore should be sign-extended
when needed.

\item \B{A\_WIDTH} \\
The width of the input port \B{A}.

\item \B{Y\_WIDTH} \\
The width of the output port \B{Y}.
\end{itemize}

Table~\ref{tab:CellLib_unary} lists all cells for unary RTL operators.

\begin{table}[t!]
\hfil
\begin{tabular}{ll}
Verilog & Cell Type \\
\hline
\lstinline[language=Verilog]; Y =  ~A ; & {\tt \$not} \\
\lstinline[language=Verilog]; Y =  +A ; & {\tt \$pos} \\
\lstinline[language=Verilog]; Y =  -A ; & {\tt \$neg} \\
\hline
\lstinline[language=Verilog]; Y =  &A ; & {\tt \$reduce\_and} \\
\lstinline[language=Verilog]; Y =  |A ; & {\tt \$reduce\_or} \\
\lstinline[language=Verilog]; Y =  ^A ; & {\tt \$reduce\_xor} \\
\lstinline[language=Verilog]; Y = ~^A ; & {\tt \$reduce\_xnor} \\
\hline
\lstinline[language=Verilog]; Y =  |A ; & {\tt \$reduce\_bool} \\
\lstinline[language=Verilog]; Y =  !A ; & {\tt \$logic\_not}
\end{tabular}
\caption{Cell types for unary operators with their corresponding Verilog expressions.}
\label{tab:CellLib_unary}
\end{table}

For the unary cells that output a logical value ({\tt \$reduce\_and}, {\tt \$reduce\_or},
{\tt \$reduce\_xor}, {\tt \$reduce\_xnor}, {\tt \$reduce\_bool}, {\tt \$logic\_not}),
when the \B{Y\_WIDTH} parameter is greater than 1, the output is zero-extended,
and only the least significant bit varies.

Note that {\tt \$reduce\_or} and {\tt \$reduce\_bool} actually represent the same
logic function. But the HDL frontends generate them in different situations. A
{\tt \$reduce\_or} cell is generated when the prefix {\tt |} operator is being used. A
{\tt \$reduce\_bool} cell is generated when a bit vector is used as a condition in
an {\tt if}-statement or {\tt ?:}-expression.

\subsection{Binary Operators}

All binary RTL cells have two input ports \B{A} and \B{B} and one output port \B{Y}. They
also have the following parameters:

\begin{itemize}
\item \B{A\_SIGNED} \\
Set to a non-zero value if the input \B{A} is signed and therefore should be sign-extended
when needed.

\item \B{A\_WIDTH} \\
The width of the input port \B{A}.

\item \B{B\_SIGNED} \\
Set to a non-zero value if the input \B{B} is signed and therefore should be sign-extended
when needed.

\item \B{B\_WIDTH} \\
The width of the input port \B{B}.

\item \B{Y\_WIDTH} \\
The width of the output port \B{Y}.
\end{itemize}

Table~\ref{tab:CellLib_binary} lists all cells for binary RTL operators.

\begin{table}[t!]
\hfil
\begin{tabular}[t]{ll}
Verilog & Cell Type \\
\hline
\lstinline[language=Verilog]; Y = A  & B; & {\tt \$and} \\
\lstinline[language=Verilog]; Y = A  | B; & {\tt \$or} \\
\lstinline[language=Verilog]; Y = A  ^ B; & {\tt \$xor} \\
\lstinline[language=Verilog]; Y = A ~^ B; & {\tt \$xnor} \\
\hline
\lstinline[language=Verilog]; Y = A << B; & {\tt \$shl} \\
\lstinline[language=Verilog]; Y = A >> B; & {\tt \$shr} \\
\lstinline[language=Verilog]; Y = A <<< B; & {\tt \$sshl} \\
\lstinline[language=Verilog]; Y = A >>> B; & {\tt \$sshr} \\
\hline
\lstinline[language=Verilog]; Y = A && B; & {\tt \$logic\_and} \\
\lstinline[language=Verilog]; Y = A || B; & {\tt \$logic\_or} \\
\hline
\lstinline[language=Verilog]; Y = A === B; & {\tt \$eqx} \\
\lstinline[language=Verilog]; Y = A !== B; & {\tt \$nex} \\
\end{tabular}
\hfil
\begin{tabular}[t]{ll}
Verilog & Cell Type \\
\hline
\lstinline[language=Verilog]; Y = A <  B; & {\tt \$lt} \\
\lstinline[language=Verilog]; Y = A <= B; & {\tt \$le} \\
\lstinline[language=Verilog]; Y = A == B; & {\tt \$eq} \\
\lstinline[language=Verilog]; Y = A != B; & {\tt \$ne} \\
\lstinline[language=Verilog]; Y = A >= B; & {\tt \$ge} \\
\lstinline[language=Verilog]; Y = A >  B; & {\tt \$gt} \\
\hline
\lstinline[language=Verilog]; Y = A  + B; & {\tt \$add} \\
\lstinline[language=Verilog]; Y = A  - B; & {\tt \$sub} \\
\lstinline[language=Verilog]; Y = A  * B; & {\tt \$mul} \\
\lstinline[language=Verilog]; Y = A  / B; & {\tt \$div} \\
\lstinline[language=Verilog]; Y = A  % B; & {\tt \$mod} \\
\lstinline[language=Verilog]; Y = A ** B; & {\tt \$pow} \\
\end{tabular}
\caption{Cell types for binary operators with their corresponding Verilog expressions.}
\label{tab:CellLib_binary}
\end{table}

The {\tt \$shl} and {\tt \$shr} cells implement logical shifts, whereas the {\tt \$sshl} and
{\tt \$sshr} cells implement arithmetic shifts. The {\tt \$shl} and {\tt \$sshl} cells implement
the same operation. All four of these cells interpret the second operand as unsigned, and require
\B{B\_SIGNED} to be zero.

Two additional shift operator cells are available that do not directly correspond to any operator
in Verilog, {\tt \$shift} and {\tt \$shiftx}. The {\tt \$shift} cell performs a right logical shift
if the second operand is positive (or unsigned), and a left logical shift if it is negative.
The {\tt \$shiftx} cell performs the same operation as the {\tt \$shift} cell, but the vacated bit
positions are filled with undef (x) bits, and corresponds to the Verilog indexed part-select expression.

For the binary cells that output a logical value ({\tt \$logic\_and}, {\tt \$logic\_or},
{\tt \$eqx}, {\tt \$nex}, {\tt \$lt}, {\tt \$le}, {\tt \$eq}, {\tt \$ne}, {\tt \$ge},
{\tt \$gt}), when the \B{Y\_WIDTH} parameter is greater than 1, the output is zero-extended,
and only the least significant bit varies.

\subsection{Multiplexers}

Multiplexers are generated by the Verilog HDL frontend for {\tt
?:}-expressions. Multiplexers are also generated by the {\tt proc} pass to map the decision trees
from RTLIL::Process objects to logic.

The simplest multiplexer cell type is {\tt \$mux}. Cells of this type have a \B{WIDTH} parameter
and data inputs \B{A} and \B{B} and a data output \B{Y}, all of the specified width. This cell also
has a single bit control input \B{S}. If \B{S} is 0 the value from the \B{A} input is sent to
the output, if it is 1 the value from the \B{B} input is sent to the output. So the {\tt \$mux}
cell implements the function \lstinline[language=Verilog]; Y = S ? B : A;.

The {\tt \$pmux} cell is used to multiplex between many inputs using a one-hot select signal. Cells
of this type have a \B{WIDTH} and a \B{S\_WIDTH} parameter and inputs \B{A}, \B{B}, and \B{S} and
an output \B{Y}. The \B{S} input is \B{S\_WIDTH} bits wide. The \B{A} input and the output are both
\B{WIDTH} bits wide and the \B{B} input is \B{WIDTH}*\B{S\_WIDTH} bits wide. When all bits of
\B{S} are zero, the value from \B{A} input is sent to the output. If the $n$'th bit from \B{S} is
set, the value $n$'th \B{WIDTH} bits wide slice of the \B{B} input is sent to the output. When more
than one bit from \B{S} is set the output is undefined. Cells of this type are used to model
``parallel cases'' (defined by using the {\tt parallel\_case} attribute or detected by
an optimization).

The {\tt \$tribuf} cell is used to implement tristate logic. Cells of this type have a \B{WIDTH}
parameter and inputs \B{A} and \B{EN} and an output \B{Y}. The \B{A} input and \B{Y} output are
\B{WIDTH} bits wide, and the \B{EN} input is one bit wide. When \B{EN} is 0, the output \B{Y}
is not driven. When \B{EN} is 1, the value from \B{A} input is sent to the \B{Y} output. Therefore,
the {\tt \$tribuf} cell implements the function \lstinline[language=Verilog]; Y = EN ? A : 'bz;.

Behavioural code with cascaded {\tt if-then-else}- and {\tt case}-statements
usually results in trees of multiplexer cells. Many passes (from various
optimizations to FSM extraction) heavily depend on these multiplexer trees to
understand dependencies between signals. Therefore optimizations should not
break these multiplexer trees (e.g.~by replacing a multiplexer between a
calculated signal and a constant zero with an {\tt \$and} gate).

\subsection{Registers}

D-type flip-flops are represented by {\tt \$dff} cells. These cells have a clock port \B{CLK},
an input port \B{D} and an output port \B{Q}. The following parameters are available for {\tt \$dff}
cells:

\begin{itemize}
\item \B{WIDTH} \\
The width of input \B{D} and output \B{Q}.

\item \B{CLK\_POLARITY} \\
Clock is active on the positive edge if this parameter has the value {\tt 1'b1} and on the negative
edge if this parameter is {\tt 1'b0}.
\end{itemize}

D-type flip-flops with enable are represented by {\tt \$dffe} cells. As the {\tt \$dff}
cells they have \B{CLK}, \B{D} and \B{Q} ports. In addition they also have a single-bit \B{EN}
input port for the enable pin and the following parameter:

\begin{itemize}
\item \B{EN\_POLARITY} \\
The enable input is active-high if this parameter has the value {\tt 1'b1} and active-low
if this parameter is {\tt 1'b0}.
\end{itemize}

D-type flip-flops with asynchronous reset are represented by {\tt \$adff} cells. As the {\tt \$dff}
cells they have \B{CLK}, \B{D} and \B{Q} ports. In addition they also have a single-bit \B{ARST}
input port for the reset pin and the following additional two parameters:

\begin{itemize}
\item \B{ARST\_POLARITY} \\
The asynchronous reset is active-high if this parameter has the value {\tt 1'b1} and active-low
if this parameter is {\tt 1'b0}.

\item \B{ARST\_VALUE} \\
The state of \B{Q} will be set to this value when the reset is active.
\end{itemize}

Note that the {\tt \$adff} cell can only be used when the reset value is constant.

\begin{sloppypar}
Usually these cells are generated by the {\tt proc} pass using the information
in the designs RTLIL::Process objects.
\end{sloppypar}

D-type flip-flops with asynchronous set and reset are represented by {\tt \$dffsr} cells.
As the {\tt \$dff} cells they have \B{CLK}, \B{D} and \B{Q} ports. In addition they also have
a single-bit \B{SET} input port for the set pin, a single-bit \B{CLR} input port for the reset pin,
and the following two parameters:

\begin{itemize}
\item \B{SET\_POLARITY} \\
The set input is active-high if this parameter has the value {\tt 1'b1} and active-low
if this parameter is {\tt 1'b0}.

\item \B{CLR\_POLARITY} \\
The reset input is active-high if this parameter has the value {\tt 1'b1} and active-low
if this parameter is {\tt 1'b0}.
\end{itemize}

When both the set and reset inputs of a {\tt \$dffsr} cell are active, the reset input takes
precedence.

\begin{fixme}
Add information about {\tt \$sr} cells (set-reset flip-flops), {\tt \$dlatch} cells (d-type latches),
and {\tt \$dlatchsr} cells (d-type latches with set/reset).
\end{fixme}

\subsection{Memories}
\label{sec:memcells}

Memories are either represented using RTLIL::Memory objects, {\tt \$memrd}, {\tt \$memwr}, and {\tt \$meminit}
cells, or by {\tt \$mem} cells alone.

In the first alternative the RTLIL::Memory objects hold the general metadata for the memory (bit width,
size in number of words, etc.) and for each port a {\tt \$memrd} (read port) or {\tt \$memwr} (write port)
cell is created. Having individual cells for read and write ports has the advantage that they can be
consolidated using resource sharing passes. In some cases this drastically reduces the number of required
ports on the memory cell. In this alternative, memory initialization data is represented by {\tt \$meminit} cells,
which allow delaying constant folding for initialization addresses and data until after the frontend finishes.

The {\tt \$memrd} cells have a clock input \B{CLK}, an enable input \B{EN}, an
address input \B{ADDR}, and a data output \B{DATA}. They also have the
following parameters:

\begin{itemize}
\item \B{MEMID} \\
The name of the RTLIL::Memory object that is associated with this read port.

\item \B{ABITS} \\
The number of address bits (width of the \B{ADDR} input port).

\item \B{WIDTH} \\
The number of data bits (width of the \B{DATA} output port).

\item \B{CLK\_ENABLE} \\
When this parameter is non-zero, the clock is used. Otherwise this read port is asynchronous and
the \B{CLK} input is not used.

\item \B{CLK\_POLARITY} \\
Clock is active on the positive edge if this parameter has the value {\tt 1'b1} and on the negative
edge if this parameter is {\tt 1'b0}.

\item \B{TRANSPARENT} \\
If this parameter is set to {\tt 1'b1}, a read and write to the same address in the same cycle will
return the new value. Otherwise the old value is returned.
\end{itemize}

The {\tt \$memwr} cells have a clock input \B{CLK}, an enable input \B{EN} (one
enable bit for each data bit), an address input \B{ADDR} and a data input
\B{DATA}. They also have the following parameters:

\begin{itemize}
\item \B{MEMID} \\
The name of the RTLIL::Memory object that is associated with this write port.

\item \B{ABITS} \\
The number of address bits (width of the \B{ADDR} input port).

\item \B{WIDTH} \\
The number of data bits (width of the \B{DATA} output port).

\item \B{CLK\_ENABLE} \\
When this parameter is non-zero, the clock is used. Otherwise this write port is asynchronous and
the \B{CLK} input is not used.

\item \B{CLK\_POLARITY} \\
Clock is active on positive edge if this parameter has the value {\tt 1'b1} and on the negative
edge if this parameter is {\tt 1'b0}.

\item \B{PRIORITY} \\
The cell with the higher integer value in this parameter wins a write conflict.
\end{itemize}

The {\tt \$meminit} cells have an address input \B{ADDR} and a data input \B{DATA}, with the width
of the \B{DATA} port equal to \B{WIDTH} parameter times \B{WORDS} parameter. Both of the inputs
must resolve to a constant for synthesis to succeed.

\begin{itemize}
\item \B{MEMID} \\
The name of the RTLIL::Memory object that is associated with this initialization cell.

\item \B{ABITS} \\
The number of address bits (width of the \B{ADDR} input port).

\item \B{WIDTH} \\
The number of data bits per memory location.

\item \B{WORDS} \\
The number of consecutive memory locations initialized by this cell.

\item \B{PRIORITY} \\
The cell with the higher integer value in this parameter wins an initialization conflict.
\end{itemize}

The HDL frontend models a memory using RTLIL::Memory objects and asynchronous
{\tt \$memrd} and {\tt \$memwr} cells. The {\tt memory} pass (i.e.~its various sub-passes) migrates
{\tt \$dff} cells into the {\tt \$memrd} and {\tt \$memwr} cells making them synchronous, then
converts them to a single {\tt \$mem} cell and (optionally) maps this cell type
to {\tt \$dff} cells for the individual words and multiplexer-based address decoders for the read and
write interfaces. When the last step is disabled or not possible, a {\tt \$mem} cell is left in the design.

The {\tt \$mem} cell provides the following parameters:

\begin{itemize}
\item \B{MEMID} \\
The name of the original RTLIL::Memory object that became this {\tt \$mem} cell.

\item \B{SIZE} \\
The number of words in the memory.

\item \B{ABITS} \\
The number of address bits.

\item \B{WIDTH} \\
The number of data bits per word.

\item \B{INIT} \\
The initial memory contents.

\item \B{RD\_PORTS} \\
The number of read ports on this memory cell.

\item \B{RD\_CLK\_ENABLE} \\
This parameter is \B{RD\_PORTS} bits wide, containing a clock enable bit for each read port.

\item \B{RD\_CLK\_POLARITY} \\
This parameter is \B{RD\_PORTS} bits wide, containing a clock polarity bit for each read port.

\item \B{RD\_TRANSPARENT} \\
This parameter is \B{RD\_PORTS} bits wide, containing a transparent bit for each read port.

\item \B{WR\_PORTS} \\
The number of write ports on this memory cell.

\item \B{WR\_CLK\_ENABLE} \\
This parameter is \B{WR\_PORTS} bits wide, containing a clock enable bit for each write port.

\item \B{WR\_CLK\_POLARITY} \\
This parameter is \B{WR\_PORTS} bits wide, containing a clock polarity bit for each write port.
\end{itemize}

The {\tt \$mem} cell has the following ports:

\begin{itemize}
\item \B{RD\_CLK} \\
This input is \B{RD\_PORTS} bits wide, containing all clock signals for the read ports.

\item \B{RD\_EN} \\
This input is \B{RD\_PORTS} bits wide, containing all enable signals for the read ports.

\item \B{RD\_ADDR} \\
This input is \B{RD\_PORTS}*\B{ABITS} bits wide, containing all address signals for the read ports.

\item \B{RD\_DATA} \\
This input is \B{RD\_PORTS}*\B{WIDTH} bits wide, containing all data signals for the read ports.

\item \B{WR\_CLK} \\
This input is \B{WR\_PORTS} bits wide, containing all clock signals for the write ports.

\item \B{WR\_EN} \\
This input is \B{WR\_PORTS}*\B{WIDTH} bits wide, containing all enable signals for the write ports.

\item \B{WR\_ADDR} \\
This input is \B{WR\_PORTS}*\B{ABITS} bits wide, containing all address signals for the write ports.

\item \B{WR\_DATA} \\
This input is \B{WR\_PORTS}*\B{WIDTH} bits wide, containing all data signals for the write ports.
\end{itemize}

The {\tt memory\_collect} pass can be used to convert discrete {\tt \$memrd}, {\tt \$memwr}, and {\tt \$meminit} cells
belonging to the same memory to a single {\tt \$mem} cell, whereas the {\tt memory\_unpack} pass performs the inverse operation.
The {\tt memory\_dff} pass can combine asynchronous memory ports that are fed by or feeding registers into synchronous memory ports.
The {\tt memory\_bram} pass can be used to recognize {\tt \$mem} cells that can be implemented with a block RAM resource on an FPGA.
The {\tt memory\_map} pass can be used to implement {\tt \$mem} cells as basic logic: word-wide DFFs and address decoders.

\subsection{Finite State Machines}

\begin{fixme}
Add a brief description of the {\tt \$fsm} cell type.
\end{fixme}

\section{Gates}
\label{sec:celllib_gates}

For gate level logic networks, fixed function single bit cells are used that do
not provide any parameters.

Simulation models for these cells can be found in the file {\tt techlibs/common/simcells.v} in the Yosys
source tree.

\begin{table}[t]
\hfil
\begin{tabular}[t]{ll}
Verilog & Cell Type \\
\hline
\lstinline[language=Verilog]; Y = ~A;    & {\tt \$\_NOT\_} \\
\lstinline[language=Verilog]; Y = A & B; & {\tt \$\_AND\_} \\
\lstinline[language=Verilog]; Y = ~(A & B); & {\tt \$\_NAND\_} \\
\lstinline[language=Verilog]; Y = A & ~B; & {\tt \$\_ANDNOT\_} \\
\lstinline[language=Verilog]; Y = A | B; & {\tt \$\_OR\_} \\
\lstinline[language=Verilog]; Y = ~(A | B); & {\tt \$\_NOR\_} \\
\lstinline[language=Verilog]; Y = A | ~B; & {\tt \$\_ORNOT\_} \\
\lstinline[language=Verilog]; Y = A ^ B; & {\tt \$\_XOR\_} \\
\lstinline[language=Verilog]; Y = ~(A ^ B); & {\tt \$\_XNOR\_} \\
\lstinline[language=Verilog]; Y = S ? B : A; & {\tt \$\_MUX\_} \\
\lstinline[language=Verilog]; Y = EN ? A : 'bz; & {\tt \$\_TBUF\_} \\
\hline
\lstinline[language=Verilog]; always @(negedge C) Q <= D; & {\tt \$\_DFF\_N\_} \\
\lstinline[language=Verilog]; always @(posedge C) Q <= D; & {\tt \$\_DFF\_P\_} \\
\end{tabular}
\hfil
\begin{tabular}[t]{llll}
$ClkEdge$ & $RstLvl$ & $RstVal$ & Cell Type \\
\hline
\lstinline[language=Verilog];negedge; & \lstinline[language=Verilog];0; & \lstinline[language=Verilog];0; & {\tt \$\_DFF\_NN0\_} \\
\lstinline[language=Verilog];negedge; & \lstinline[language=Verilog];0; & \lstinline[language=Verilog];1; & {\tt \$\_DFF\_NN1\_} \\
\lstinline[language=Verilog];negedge; & \lstinline[language=Verilog];1; & \lstinline[language=Verilog];0; & {\tt \$\_DFF\_NP0\_} \\
\lstinline[language=Verilog];negedge; & \lstinline[language=Verilog];1; & \lstinline[language=Verilog];1; & {\tt \$\_DFF\_NP1\_} \\
\lstinline[language=Verilog];posedge; & \lstinline[language=Verilog];0; & \lstinline[language=Verilog];0; & {\tt \$\_DFF\_PN0\_} \\
\lstinline[language=Verilog];posedge; & \lstinline[language=Verilog];0; & \lstinline[language=Verilog];1; & {\tt \$\_DFF\_PN1\_} \\
\lstinline[language=Verilog];posedge; & \lstinline[language=Verilog];1; & \lstinline[language=Verilog];0; & {\tt \$\_DFF\_PP0\_} \\
\lstinline[language=Verilog];posedge; & \lstinline[language=Verilog];1; & \lstinline[language=Verilog];1; & {\tt \$\_DFF\_PP1\_} \\
\end{tabular}
% FIXME: the layout of this is broken and I have no idea how to fix it
\hfil
\begin{tabular}[t]{lll}
$ClkEdge$ & $EnLvl$ & Cell Type \\
\hline
\lstinline[language=Verilog];negedge; & \lstinline[language=Verilog];0; & {\tt \$\_DFFE\_NN\_} \\
\lstinline[language=Verilog];negedge; & \lstinline[language=Verilog];1; & {\tt \$\_DFFE\_NP\_} \\
\lstinline[language=Verilog];posedge; & \lstinline[language=Verilog];0; & {\tt \$\_DFFE\_PN\_} \\
\lstinline[language=Verilog];posedge; & \lstinline[language=Verilog];1; & {\tt \$\_DFFE\_PP\_} \\
\end{tabular}
% FIXME: the layout of this is broken too
\hfil
\begin{tabular}[t]{llll}
$ClkEdge$ & $SetLvl$ & $RstLvl$ & Cell Type \\
\hline
\lstinline[language=Verilog];negedge; & \lstinline[language=Verilog];0; & \lstinline[language=Verilog];0; & {\tt \$\_DFFSR\_NNN\_} \\
\lstinline[language=Verilog];negedge; & \lstinline[language=Verilog];0; & \lstinline[language=Verilog];1; & {\tt \$\_DFFSR\_NNP\_} \\
\lstinline[language=Verilog];negedge; & \lstinline[language=Verilog];1; & \lstinline[language=Verilog];0; & {\tt \$\_DFFSR\_NPN\_} \\
\lstinline[language=Verilog];negedge; & \lstinline[language=Verilog];1; & \lstinline[language=Verilog];1; & {\tt \$\_DFFSR\_NPP\_} \\
\lstinline[language=Verilog];posedge; & \lstinline[language=Verilog];0; & \lstinline[language=Verilog];0; & {\tt \$\_DFFSR\_PNN\_} \\
\lstinline[language=Verilog];posedge; & \lstinline[language=Verilog];0; & \lstinline[language=Verilog];1; & {\tt \$\_DFFSR\_PNP\_} \\
\lstinline[language=Verilog];posedge; & \lstinline[language=Verilog];1; & \lstinline[language=Verilog];0; & {\tt \$\_DFFSR\_PPN\_} \\
\lstinline[language=Verilog];posedge; & \lstinline[language=Verilog];1; & \lstinline[language=Verilog];1; & {\tt \$\_DFFSR\_PPP\_} \\
\end{tabular}
\caption{Cell types for gate level logic networks}
\label{tab:CellLib_gates}
\end{table}

Table~\ref{tab:CellLib_gates} lists all cell types used for gate level logic. The cell types
{\tt \$\_NOT\_}, {\tt \$\_AND\_}, {\tt \$\_NAND\_}, {\tt \$\_ANDNOT\_}, {\tt \$\_OR\_}, {\tt \$\_NOR\_},
{\tt \$\_ORNOT\_}, {\tt \$\_XOR\_}, {\tt \$\_XNOR\_} and {\tt \$\_MUX\_} are used to model combinatorial logic.
The cell type {\tt \$\_TBUF\_} is used to model tristate logic.

The cell types {\tt \$\_DFF\_N\_} and {\tt \$\_DFF\_P\_} represent d-type flip-flops.

The cell types {\tt \$\_DFFE\_NN\_}, {\tt \$\_DFFE\_NP\_}, {\tt \$\_DFFE\_PN\_} and {\tt \$\_DFFE\_PP\_}
implement d-type flip-flops with enable. The values in the table for these cell types relate to the
following Verilog code template.

\begin{lstlisting}[mathescape,language=Verilog]
	always @($ClkEdge$ C)
		if (EN == $EnLvl$)
			Q <= D;
\end{lstlisting}

The cell types {\tt \$\_DFF\_NN0\_}, {\tt \$\_DFF\_NN1\_}, {\tt \$\_DFF\_NP0\_}, {\tt \$\_DFF\_NP1\_},
{\tt \$\_DFF\_PN0\_}, {\tt \$\_DFF\_PN1\_}, {\tt \$\_DFF\_PP0\_} and {\tt \$\_DFF\_PP1\_} implement
d-type flip-flops with asynchronous reset. The values in the table for these cell types relate to the
following Verilog code template, where \lstinline[mathescape,language=Verilog];$RstEdge$; is \lstinline[language=Verilog];posedge;
if \lstinline[mathescape,language=Verilog];$RstLvl$; if \lstinline[language=Verilog];1;, and \lstinline[language=Verilog];negedge;
otherwise.

\begin{lstlisting}[mathescape,language=Verilog]
	always @($ClkEdge$ C, $RstEdge$ R)
		if (R == $RstLvl$)
			Q <= $RstVal$;
		else
			Q <= D;
\end{lstlisting}

The cell types {\tt \$\_DFFSR\_NNN\_}, {\tt \$\_DFFSR\_NNP\_}, {\tt \$\_DFFSR\_NPN\_}, {\tt \$\_DFFSR\_NPP\_},
{\tt \$\_DFFSR\_PNN\_}, {\tt \$\_DFFSR\_PNP\_}, {\tt \$\_DFFSR\_PPN\_} and {\tt \$\_DFFSR\_PPP\_} implement
d-type flip-flops with asynchronous set and reset. The values in the table for these cell types relate to the
following Verilog code template, where \lstinline[mathescape,language=Verilog];$RstEdge$; is \lstinline[language=Verilog];posedge;
if \lstinline[mathescape,language=Verilog];$RstLvl$; if \lstinline[language=Verilog];1;, \lstinline[language=Verilog];negedge;
otherwise, and \lstinline[mathescape,language=Verilog];$SetEdge$; is \lstinline[language=Verilog];posedge;
if \lstinline[mathescape,language=Verilog];$SetLvl$; if \lstinline[language=Verilog];1;, \lstinline[language=Verilog];negedge;
otherwise.

\begin{lstlisting}[mathescape,language=Verilog]
	always @($ClkEdge$ C, $RstEdge$ R, $SetEdge$ S)
		if (R == $RstLvl$)
			Q <= 0;
		else if (S == $SetLvl$)
			Q <= 1;
		else
			Q <= D;
\end{lstlisting}

In most cases gate level logic networks are created from RTL networks using the {\tt techmap} pass. The flip-flop cells
from the gate level logic network can be mapped to physical flip-flop cells from a Liberty file using the {\tt dfflibmap}
pass. The combinatorial logic cells can be mapped to physical cells from a Liberty file via ABC \citeweblink{ABC}
using the {\tt abc} pass.

\begin{fixme}
Add information about {\tt \$assert}, {\tt \$assume}, {\tt \$live}, {\tt \$fair}, {\tt \$cover}, {\tt \$equiv},
{\tt \$initstate}, {\tt \$anyconst}, {\tt \$anyseq}, {\tt \$allconst}, {\tt \$allseq} cells.
\end{fixme}

\begin{fixme}
Add information about {\tt \$specify2}, {\tt \$specify3}, and {\tt \$specrule} cells.
\end{fixme}

\begin{fixme}
Add information about {\tt \$slice} and {\tt \$concat} cells.
\end{fixme}

\begin{fixme}
Add information about {\tt \$lut} and {\tt \$sop} cells.
\end{fixme}

\begin{fixme}
Add information about {\tt \$alu}, {\tt \$macc}, {\tt \$fa}, and {\tt \$lcu} cells.
\end{fixme}

\begin{fixme}
Add information about {\tt \$ff} and {\tt \$\_FF\_} cells.
\end{fixme}

\begin{fixme}
Add information about {\tt \$\_DLATCH\_?\_}, and {\tt \$\_DLATCHSR\_???\_} cells.
\end{fixme}

\begin{fixme}
Add information about {\tt \$\_AOI3\_}, {\tt \$\_OAI3\_}, {\tt \$\_AOI4\_}, {\tt \$\_OAI4\_}, and {\tt \$\_NMUX\_} cells.
\end{fixme}


\include{CHAPTER_Prog}


\chapter{The Verilog and AST Frontends}
\label{chapter:verilog}

This chapter provides an overview of the implementation of the Yosys Verilog
and AST frontends. The Verilog frontend reads Verilog-2005 code and creates
an abstract syntax tree (AST) representation of the input. This AST representation
is then passed to the AST frontend that converts it to RTLIL data, as illustrated
in Fig.~\ref{fig:Verilog_flow}.

\begin{figure}[b!]
	\hfil
	\begin{tikzpicture}
		\tikzstyle{process} = [draw, fill=green!10, rectangle, minimum height=3em, minimum width=10em, node distance=5em, font={\ttfamily}]
		\tikzstyle{data} = [draw, fill=blue!10, ellipse, minimum height=3em, minimum width=7em, node distance=5em, font={\ttfamily}]

		\node[data]    (n1)               {Verilog Source};
		\node[process] (n2) [below of=n1] {Verilog Frontend};
		\node[data]    (n3) [below of=n2] {AST};
		\node[process] (n4) [below of=n3] {AST Frontend};
		\node[data]    (n5) [below of=n4] {RTLIL};

		\draw[-latex] (n1) -- (n2);
		\draw[-latex] (n2) -- (n3);
		\draw[-latex] (n3) -- (n4);
		\draw[-latex] (n4) -- (n5);

		\tikzstyle{details} = [draw, fill=yellow!5, rectangle, node distance=6cm, font={\ttfamily}]

		\node[details] (d1) [right of=n2] {\begin{minipage}{5cm}
			\hfil
			\begin{tikzpicture}
				\tikzstyle{subproc} = [draw, fill=green!10, rectangle, minimum height=2em, minimum width=10em, node distance=3em, font={\ttfamily}]
				\node (s0) {};
				\node[subproc] (s1) [below of=s0] {Preprocessor};
				\node[subproc] (s2) [below of=s1] {Lexer};
				\node[subproc] (s3) [below of=s2] {Parser};
				\node[node distance=3em] (s4) [below of=s3] {};
				\draw[-latex] (s0) -- (s1);
				\draw[-latex] (s1) -- (s2);
				\draw[-latex] (s2) -- (s3);
				\draw[-latex] (s3) -- (s4);
			\end{tikzpicture}
		\end{minipage}};

		\draw[dashed] (n2.north east) -- (d1.north west);
		\draw[dashed] (n2.south east) -- (d1.south west);

		\node[details] (d2) [right of=n4] {\begin{minipage}{5cm}
			\hfil
			\begin{tikzpicture}
				\tikzstyle{subproc} = [draw, fill=green!10, rectangle, minimum height=2em, minimum width=10em, node distance=3em, font={\ttfamily}]
				\node (s0) {};
				\node[subproc] (s1) [below of=s0] {Simplifier};
				\node[subproc] (s2) [below of=s1] {RTLIL Generator};
				\node[node distance=3em] (s3) [below of=s2] {};
				\draw[-latex] (s0) -- (s1);
				\draw[-latex] (s1) -- (s2);
				\draw[-latex] (s2) -- (s3);
			\end{tikzpicture}
		\end{minipage}};

		\draw[dashed] (n4.north east) -- (d2.north west);
		\draw[dashed] (n4.south east) -- (d2.south west);

	\end{tikzpicture}
	\caption{Simplified Verilog to RTLIL data flow}
	\label{fig:Verilog_flow}
\end{figure}


\section{Transforming Verilog to AST}

The {\it Verilog frontend} converts the Verilog sources to an internal AST representation that closely resembles
the structure of the original Verilog code. The Verilog frontend consists of three components, the
{\it Preprocessor}, the {\it Lexer} and the {\it Parser}.

The source code to the Verilog frontend can be found in {\tt frontends/verilog/} in the Yosys source tree.

\subsection{The Verilog Preprocessor}

The Verilog preprocessor scans over the Verilog source code and interprets some of the Verilog compiler
directives such as \lstinline[language=Verilog]{`include}, \lstinline[language=Verilog]{`define} and
\lstinline[language=Verilog]{`ifdef}.

It is implemented as a C++ function that is passed a file descriptor as input and returns the
pre-processed Verilog code as a \lstinline[language=C++]{std::string}.

The source code to the Verilog Preprocessor can be found in {\tt
frontends/verilog/preproc.cc} in the Yosys source tree.

\subsection{The Verilog Lexer}

\begin{sloppypar}
The Verilog Lexer is written using the lexer generator {\it flex} \citeweblink{flex}. Its source code
can be found in {\tt frontends/verilog/verilog\_lexer.l} in the Yosys source tree.
The lexer does little more than identifying all keywords and literals
recognised by the Yosys Verilog frontend.
\end{sloppypar}

The lexer keeps track of the current location in the Verilog source code using
some global variables. These variables are used by the constructor of AST nodes
to annotate each node with the source code location it originated from.

\begin{sloppypar}
Finally the lexer identifies and handles special comments such as
``\lstinline[language=Verilog]{// synopsys translate_off}'' and
``\lstinline[language=Verilog]{// synopsys full_case}''. (It is recommended to
use \lstinline[language=Verilog]{`ifdef} constructs instead of the Synsopsys
translate\_on/off comments and attributes such as
\lstinline[language=Verilog]{(* full_case *)} over ``\lstinline[language=Verilog]{// synopsys full_case}''
whenever possible.)
\end{sloppypar}

\subsection{The Verilog Parser}

The Verilog Parser is written using the parser generator {\it bison} \citeweblink{bison}. Its source code
can be found in {\tt frontends/verilog/verilog\_parser.y} in the Yosys source tree.

It generates an AST using the \lstinline[language=C++]{AST::AstNode} data structure
defined in {\tt frontends/ast/ast.h}. An \lstinline[language=C++]{AST::AstNode} object has
the following properties:

%%%%%%%%%%%%%%%%%%%%%%%%%%%%%%%%%%%%%%%%%%%%%%%%%%%%%%%%%%%%%%%%%%

\begin{table}[b!]
\hfil
\begin{tabular}{>{\raggedright\arraybackslash}p{7cm}>{\raggedright\arraybackslash}p{8cm}}
AST Node Type & Corresponding Verilog Construct \\
\hline
\hline
\arrayrulecolor{gray}
{\tt AST\_NONE} & This Node type should never be used. \\
\hline
%
{\tt AST\_DESIGN} & This node type is used for the top node of the AST tree. It
has no corresponding Verilog construct. \\
\hline
%
{\tt AST\_MODULE},
{\tt AST\_TASK},
{\tt AST\_FUNCTION} &
\lstinline[language=Verilog];module;,
\lstinline[language=Verilog];task; and
\lstinline[language=Verilog];function; \\
\hline
%
{\tt AST\_WIRE} &
\lstinline[language=Verilog];input;,
\lstinline[language=Verilog];output;,
\lstinline[language=Verilog];wire;,
\lstinline[language=Verilog];reg; and
\lstinline[language=Verilog];integer; \\
\hline
%
{\tt AST\_MEMORY}  &
Verilog Arrays \\
\hline
%
{\tt AST\_AUTOWIRE} &
Created by the simplifier when an undeclared signal name is used. \\
\hline
%
{\tt AST\_PARAMETER},
{\tt AST\_LOCALPARAM} &
\lstinline[language=Verilog];parameter; and
\lstinline[language=Verilog];localparam; \\
\hline
%
{\tt AST\_PARASET} &
Parameter set in cell instantiation \\
\hline
%
{\tt AST\_ARGUMENT} &
Port connection in cell instantiation \\
\hline
%
{\tt AST\_RANGE} &
Bit-Index in a signal or element index in array \\
\hline
%
{\tt AST\_CONSTANT} &
A literal value \\
\hline
%
{\tt AST\_CELLTYPE} &
The type of cell in cell instantiation \\
\hline
%
{\tt AST\_IDENTIFIER} &
An Identifier (signal name in expression or cell/task/etc. name in other contexts) \\
\hline
%
{\tt AST\_PREFIX} &
Construct an identifier in the form {\tt <prefix>[<index>].<suffix>} (used only in
advanced generate constructs) \\
\hline
%
{\tt AST\_FCALL},
{\tt AST\_TCALL} &
Call to function or task \\
\hline
%
{\tt AST\_TO\_SIGNED},
{\tt AST\_TO\_UNSIGNED} &
The \lstinline[language=Verilog];$signed(); and
\lstinline[language=Verilog];$unsigned(); functions \\
\hline
\end{tabular}
\caption{AST node types with their corresponding Verilog constructs. \\ (continued on next page)}
\label{tab:Verilog_AstNodeType}
\end{table}

\begin{table}[t!]
\ContinuedFloat
\hfil
\begin{tabular}{>{\raggedright\arraybackslash}p{7cm}>{\raggedright\arraybackslash}p{8cm}}
AST Node Type & Corresponding Verilog Construct \\
\hline
\hline
\arrayrulecolor{gray}
{\tt AST\_CONCAT}
{\tt AST\_REPLICATE} &
The \lstinline[language=Verilog];{...}; and
\lstinline[language=Verilog];{...{...}}; operators \\
\hline
%
{\tt AST\_BIT\_NOT},
{\tt AST\_BIT\_AND},
{\tt AST\_BIT\_OR},
{\tt AST\_BIT\_XOR},
{\tt AST\_BIT\_XNOR} &
The bitwise operators \break
\lstinline[language=Verilog];~;,
\lstinline[language=Verilog];&;,
\lstinline[language=Verilog];|;,
\lstinline[language=Verilog];^; and
\lstinline[language=Verilog];~^; \\
\hline
%
{\tt AST\_REDUCE\_AND},
{\tt AST\_REDUCE\_OR},
{\tt AST\_REDUCE\_XOR},
{\tt AST\_REDUCE\_XNOR} &
The unary reduction operators \break
\lstinline[language=Verilog];~;,
\lstinline[language=Verilog];&;,
\lstinline[language=Verilog];|;,
\lstinline[language=Verilog];^; and
\lstinline[language=Verilog];~^; \\
\hline
%
{\tt AST\_REDUCE\_BOOL} &
Conversion from multi-bit value to boolean value
(equivalent to {\tt AST\_REDUCE\_OR}) \\
\hline
%
{\tt AST\_SHIFT\_LEFT},
{\tt AST\_SHIFT\_RIGHT},
{\tt AST\_SHIFT\_SLEFT},
{\tt AST\_SHIFT\_SRIGHT} &
The shift operators \break
\lstinline[language=Verilog];<<;,
\lstinline[language=Verilog];>>;,
\lstinline[language=Verilog];<<<; and
\lstinline[language=Verilog];>>>; \\
\hline
%
{\tt AST\_LT},
{\tt AST\_LE},
{\tt AST\_EQ},
{\tt AST\_NE},
{\tt AST\_GE},
{\tt AST\_GT} &
The relational operators \break
\lstinline[language=Verilog];<;,
\lstinline[language=Verilog];<=;,
\lstinline[language=Verilog];==;,
\lstinline[language=Verilog];!=;,
\lstinline[language=Verilog];>=; and
\lstinline[language=Verilog];>; \\
\hline
%
{\tt AST\_ADD},
{\tt AST\_SUB},
{\tt AST\_MUL},
{\tt AST\_DIV},
{\tt AST\_MOD},
{\tt AST\_POW} &
The binary operators \break
\lstinline[language=Verilog];+;,
\lstinline[language=Verilog];-;,
\lstinline[language=Verilog];*;,
\lstinline[language=Verilog];/;,
\lstinline[language=Verilog];%; and
\lstinline[language=Verilog];**; \\
\hline
%
{\tt AST\_POS},
{\tt AST\_NEG} &
The prefix operators
\lstinline[language=Verilog];+; and
\lstinline[language=Verilog];-; \\
\hline
%
{\tt AST\_LOGIC\_AND},
{\tt AST\_LOGIC\_OR},
{\tt AST\_LOGIC\_NOT} &
The logic operators
\lstinline[language=Verilog];&&;,
\lstinline[language=Verilog];||; and
\lstinline[language=Verilog];!; \\
\hline
%
{\tt AST\_TERNARY} &
The ternary \lstinline[language=Verilog];?:;-operator \\
\hline
%
{\tt AST\_MEMRD}
{\tt AST\_MEMWR} &
Read and write memories. These nodes are generated by
the AST simplifier for writes/reads to/from Verilog arrays. \\
\hline
%
{\tt AST\_ASSIGN} &
An \lstinline[language=Verilog];assign; statement \\
\hline
%
{\tt AST\_CELL} &
A cell instantiation \\
\hline
%
{\tt AST\_PRIMITIVE} &
A primitive cell (\lstinline[language=Verilog];and;,
\lstinline[language=Verilog];nand;,
\lstinline[language=Verilog];or;, etc.) \\
\hline
%
{\tt AST\_ALWAYS},
{\tt AST\_INITIAL} &
Verilog \lstinline[language=Verilog];always;- and \lstinline[language=Verilog];initial;-blocks \\
\hline
%
{\tt AST\_BLOCK} &
A \lstinline[language=Verilog];begin;-\lstinline[language=Verilog];end;-block \\
\hline
%
{\tt AST\_ASSIGN\_EQ}.
{\tt AST\_ASSIGN\_LE} &
Blocking (\lstinline[language=Verilog];=;) and nonblocking (\lstinline[language=Verilog];<=;)
assignments within an \lstinline[language=Verilog];always;- or \lstinline[language=Verilog];initial;-block \\
\hline
%
{\tt AST\_CASE}.
{\tt AST\_COND},
{\tt AST\_DEFAULT} &
The \lstinline[language=Verilog];case; (\lstinline[language=Verilog];if;) statements, conditions within a case
and the default case respectively \\
\hline
%
{\tt AST\_FOR} &
A \lstinline[language=Verilog];for;-loop with an
\lstinline[language=Verilog];always;- or
\lstinline[language=Verilog];initial;-block \\
\hline
%
{\tt AST\_GENVAR},
{\tt AST\_GENBLOCK},
{\tt AST\_GENFOR},
{\tt AST\_GENIF} &
The \lstinline[language=Verilog];genvar; and
\lstinline[language=Verilog];generate; keywords and
\lstinline[language=Verilog];for; and \lstinline[language=Verilog];if; within a
generate block. \\
\hline
%
{\tt AST\_POSEDGE},
{\tt AST\_NEGEDGE},
{\tt AST\_EDGE} &
Event conditions for \lstinline[language=Verilog];always; blocks. \\
\hline
\end{tabular}
\caption{AST node types with their corresponding Verilog constructs. \\ (continuation from previous page)}
\label{tab:Verilog_AstNodeTypeCont}
\end{table}

%%%%%%%%%%%%%%%%%%%%%%%%%%%%%%%%%%%%%%%%%%%%%%%%%%%%%%%%%%%%%%%%%%

\begin{itemize}
\item {\bf The node type} \\
This enum (\lstinline[language=C++]{AST::AstNodeType}) specifies the role of the node.
Table~\ref{tab:Verilog_AstNodeType} contains a list of all node types.
\item {\bf The child nodes} \\
This is a list of pointers to all children in the abstract syntax tree.
\item {\bf Attributes} \\
As almost every AST node might have Verilog attributes assigned to it, the
\lstinline[language=C++]{AST::AstNode} has direct support for attributes. Note that the
attribute values are again AST nodes.
\item {\bf Node content} \\
Each node might have additional content data. A series of member variables exist to hold such data.
For example the member \lstinline[language=C++]{std::string str} can hold a string value and is
used e.g.~in the {\tt AST\_IDENTIFIER} node type to store the identifier name.
\item {\bf Source code location} \\
Each \lstinline[language=C++]{AST::AstNode} is automatically annotated with the current
source code location by the \lstinline[language=C++]{AST::AstNode} constructor. It is
stored in the \lstinline[language=C++]{std::string filename} and \lstinline[language=C++]{int linenum}
member variables.
\end{itemize}

The \lstinline[language=C++]{AST::AstNode} constructor can be called with up to
two child nodes that are automatically added to the list of child nodes for the new object.
This simplifies the creation of AST nodes for simple expressions a bit. For example the bison
code for parsing multiplications:

\begin{lstlisting}[numbers=left,frame=single]
        basic_expr '*' attr basic_expr {
                $$ = new AstNode(AST_MUL, $1, $4);
                append_attr($$, $3);
        } |
\end{lstlisting}

The generated AST data structure is then passed directly to the AST frontend
that performs the actual conversion to RTLIL.

Note that the Yosys command {\tt read\_verilog} provides the options {\tt -yydebug}
and {\tt -dump\_ast} that can be used to print the parse tree or abstract syntax tree
respectively.

\section{Transforming AST to RTLIL}

The {\it AST Frontend} converts a set of modules in AST representation to
modules in RTLIL representation and adds them to the current design. This is done
in two steps: {\it simplification} and {\it RTLIL generation}.

The source code to the AST frontend can be found in {\tt frontends/ast/} in the Yosys source tree.

\subsection{AST Simplification}

A full-featured AST is too complex to be transformed into RTLIL directly. Therefore it must
first be brought into a simpler form. This is done by calling the \lstinline[language=C++]{AST::AstNode::simplify()}
method of all {\tt AST\_MODULE} nodes in the AST. This initiates a recursive process that performs the following transformations
on the AST data structure:

\begin{itemize}
\item Inline all task and function calls.
\item Evaluate all \lstinline[language=Verilog]{generate}-statements and unroll all \lstinline[language=Verilog]{for}-loops.
\item Perform const folding where it is necessary (e.g.~in the value part of {\tt AST\_PARAMETER}, {\tt AST\_LOCALPARAM},
{\tt AST\_PARASET} and {\tt AST\_RANGE} nodes).
\item Replace {\tt AST\_PRIMITIVE} nodes with appropriate {\tt AST\_ASSIGN} nodes.
\item Replace dynamic bit ranges in the left-hand-side of assignments with {\tt AST\_CASE} nodes with {\tt AST\_COND} children
for each possible case.
\item Detect array access patterns that are too complicated for the {\tt RTLIL::Memory} abstraction and replace them
with a set of signals and cases for all reads and/or writes.
\item Otherwise replace array accesses with {\tt AST\_MEMRD} and {\tt AST\_MEMWR} nodes.
\end{itemize}

In addition to these transformations, the simplifier also annotates the AST with additional information that is needed
for the RTLIL generator, namely:

\begin{itemize}
\item All ranges (width of signals and bit selections) are not only const folded but (when a constant value
is found) are also written to member variables in the {\tt AST\_RANGE} node.
\item All identifiers are resolved and all {\tt AST\_IDENTIFIER} nodes are annotated with a pointer to the AST node
that contains the declaration of the identifier. If no declaration has been found, an {\tt AST\_AUTOWIRE} node
is created and used for the annotation.
\end{itemize}

This produces an AST that is fairly easy to convert to the RTLIL format.

\subsection{Generating RTLIL}

After AST simplification, the \lstinline[language=C++]{AST::AstNode::genRTLIL()} method of each {\tt AST\_MODULE} node
in the AST is called. This initiates a recursive process that generates equivalent RTLIL data for the AST data.

The \lstinline[language=C++]{AST::AstNode::genRTLIL()} method returns an \lstinline[language=C++]{RTLIL::SigSpec} structure.
For nodes that represent expressions (operators, constants, signals, etc.), the cells needed to implement the calculation
described by the expression are created and the resulting signal is returned. That way it is easy to generate the circuits
for large expressions using depth-first recursion. For nodes that do not represent an expression (such as {\tt
AST\_CELL}), the corresponding circuit is generated and an empty \lstinline[language=C++]{RTLIL::SigSpec} is returned.

\section{Synthesizing Verilog always Blocks}

For behavioural Verilog code (code utilizing \lstinline[language=Verilog]{always}- and
\lstinline[language=Verilog]{initial}-blocks) it is necessary to also generate \lstinline[language=C++]{RTLIL::Process}
objects. This is done in the following way:

\begin{itemize}
\item Whenever \lstinline[language=C++]{AST::AstNode::genRTLIL()} encounters an \lstinline[language=Verilog]{always}-
or \lstinline[language=Verilog]{initial}-block, it creates an instance of
\lstinline[language=Verilog]{AST_INTERNAL::ProcessGenerator}. This object then generates the
\lstinline[language=C++]{RTLIL::Process} object for the block. It also calls \lstinline[language=C++]{AST::AstNode::genRTLIL()}
for all right-hand-side expressions contained within the block.
%
\begin{sloppypar}
\item First the  \lstinline[language=Verilog]{AST_INTERNAL::ProcessGenerator} creates a list of all signals assigned
within the block. It then creates a set of temporary signals using the naming scheme {\tt \$\it<number>\tt
\textbackslash\it <original\_name>} for each of the assigned signals.
\end{sloppypar}
%
\item Then an \lstinline[language=C++]{RTLIL::Process} is created that assigns all intermediate values for each left-hand-side
signal to the temporary signal in its \lstinline[language=C++]{RTLIL::CaseRule}/\lstinline[language=C++]{RTLIL::SwitchRule} tree.
%
\item Finally a \lstinline[language=C++]{RTLIL::SyncRule} is created for the \lstinline[language=C++]{RTLIL::Process} that
assigns the temporary signals for the final values to the actual signals.
%
\item Calls to \lstinline[language=C++]{AST::AstNode::genRTLIL()} are generated for right hand sides as needed. When blocking
assignments are used, \lstinline[language=C++]{AST::AstNode::genRTLIL()} is configured using global variables to use
the temporary signals that hold the correct intermediate values whenever one of the previously assigned signals is used
in an expression.
\end{itemize}

Unfortunately the generation of a correct \lstinline[language=C++]{RTLIL::CaseRule}/\lstinline[language=C++]{RTLIL::SwitchRule}
tree for behavioural code is a non-trivial task. The AST frontend solves the problem using the approach described on the following
pages. The following example illustrates what the algorithm is supposed to do. Consider the following Verilog code:

\begin{lstlisting}[numbers=left,frame=single,language=Verilog]
always @(posedge clock) begin
	out1 = in1;
	if (in2)
		out1 = !out1;
	out2 <= out1;
	if (in3)
		out2 <= out2;
	if (in4)
		if (in5)
			out3 <= in6;
		else
			out3 <= in7;
	out1 = out1 ^ out2;
end
\end{lstlisting}

This is translated by the Verilog and AST frontends into the following RTLIL code (attributes, cell parameters
and wire declarations not included):

\begin{lstlisting}[numbers=left,frame=single,language=rtlil]
cell $logic_not $logic_not$<input>:4$2
  connect \A \in1
  connect \Y $logic_not$<input>:4$2_Y
end
cell $xor $xor$<input>:13$3
  connect \A $1\out1[0:0]
  connect \B \out2
  connect \Y $xor$<input>:13$3_Y
end
process $proc$<input>:1$1
  assign $0\out3[0:0] \out3
  assign $0\out2[0:0] $1\out1[0:0]
  assign $0\out1[0:0] $xor$<input>:13$3_Y
  switch \in2
    case 1'1
      assign $1\out1[0:0] $logic_not$<input>:4$2_Y
    case
      assign $1\out1[0:0] \in1
  end
  switch \in3
    case 1'1
      assign $0\out2[0:0] \out2
    case
  end
  switch \in4
    case 1'1
      switch \in5
        case 1'1
          assign $0\out3[0:0] \in6
        case
          assign $0\out3[0:0] \in7
      end
    case
  end
  sync posedge \clock
    update \out1 $0\out1[0:0]
    update \out2 $0\out2[0:0]
    update \out3 $0\out3[0:0]
end
\end{lstlisting}

Note that the two operators are translated into separate cells outside the generated process. The signal
\lstinline[language=Verilog]{out1} is assigned using blocking assignments and therefore \lstinline[language=Verilog]{out1}
has been replaced with a different signal in all expressions after the initial assignment. The signal
\lstinline[language=Verilog]{out2} is assigned using nonblocking assignments and therefore is not substituted
on the right-hand-side expressions.

The \lstinline[language=C++]{RTLIL::CaseRule}/\lstinline[language=C++]{RTLIL::SwitchRule}
tree must be interpreted the following way:

\begin{itemize}
\item On each case level (the body of the process is the {\it root case}), first the actions on this level are
evaluated and then the switches within the case are evaluated. (Note that the last assignment on line 13 of the
Verilog code has been moved to the beginning of the RTLIL process to line 13 of the RTLIL listing.)

I.e.~the special cases deeper in the switch hierarchy override the defaults on the upper levels. The assignments
in lines 12 and 22 of the RTLIL code serve as an example for this.

Note that in contrast to this, the order within the \lstinline[language=C++]{RTLIL::SwitchRule} objects
within a \lstinline[language=C++]{RTLIL::CaseRule} is preserved with respect to the original AST and
Verilog code.
%
\item \begin{sloppypar}
The whole \lstinline[language=C++]{RTLIL::CaseRule}/\lstinline[language=C++]{RTLIL::SwitchRule} tree
describes an asynchronous circuit. I.e.~the decision tree formed by the switches can be seen independently for
each assigned signal. Whenever one assigned signal changes, all signals that depend on the changed signals
are to be updated. For example the assignments in lines 16 and 18 in the RTLIL code in fact influence the assignment
in line 12, even though they are in the ``wrong order''.
\end{sloppypar}
\end{itemize}

The only synchronous part of the process is in the \lstinline[language=C++]{RTLIL::SyncRule} object generated at line
35 in the RTLIL code. The sync rule is the only part of the process where the original signals are assigned. The
synchronization event from the original Verilog code has been translated into the synchronization type ({\tt posedge})
and signal ({\tt \textbackslash clock}) for the \lstinline[language=C++]{RTLIL::SyncRule} object. In the case of
this simple example the \lstinline[language=C++]{RTLIL::SyncRule} object is later simply transformed into a set of
d-type flip-flops and the \lstinline[language=C++]{RTLIL::CaseRule}/\lstinline[language=C++]{RTLIL::SwitchRule} tree
to a decision tree using multiplexers.

\begin{sloppypar}
In more complex examples (e.g.~asynchronous resets) the part of the
\lstinline[language=C++]{RTLIL::CaseRule}/\lstinline[language=C++]{RTLIL::SwitchRule}
tree that describes the asynchronous reset must first be transformed to the
correct \lstinline[language=C++]{RTLIL::SyncRule} objects. This is done by the {\tt proc\_adff} pass.
\end{sloppypar}

\subsection{The ProcessGenerator Algorithm}

The \lstinline[language=C++]{AST_INTERNAL::ProcessGenerator} uses the following internal state variables:

\begin{itemize}
\item \begin{sloppypar}
\lstinline[language=C++]{subst_rvalue_from} and \lstinline[language=C++]{subst_rvalue_to} \\
These two variables hold the replacement pattern that should be used by \lstinline[language=C++]{AST::AstNode::genRTLIL()}
for signals with blocking assignments. After initialization of \lstinline[language=C++]{AST_INTERNAL::ProcessGenerator}
these two variables are empty.
\end{sloppypar}
%
\item \lstinline[language=C++]{subst_lvalue_from} and \lstinline[language=C++]{subst_lvalue_to} \\
These two variables contain the mapping from left-hand-side signals ({\tt \textbackslash \it <name>}) to the current
temporary signal for the same thing (initially {\tt \$0\textbackslash \it <name>}).
%
\item \lstinline[language=C++]{current_case} \\
A pointer to a \lstinline[language=C++]{RTLIL::CaseRule} object. Initially this is the root case of the
generated \lstinline[language=C++]{RTLIL::Process}.
\end{itemize}

As the algorithm runs these variables are continuously modified as well as pushed
to the stack and later restored to their earlier values by popping from the stack.

On startup the ProcessGenerator generates a new
\lstinline[language=C++]{RTLIL::Process} object with an empty root case and
initializes its state variables as described above. Then the \lstinline[language=C++]{RTLIL::SyncRule} objects
are created using the synchronization events from the {\tt AST\_ALWAYS} node and the initial values of
\lstinline[language=C++]{subst_lvalue_from} and \lstinline[language=C++]{subst_lvalue_to}. Then the
AST for this process is evaluated recursively.

During this recursive evaluation, three different relevant types of AST nodes can be discovered:
{\tt AST\_ASSIGN\_LE} (nonblocking assignments), {\tt AST\_ASSIGN\_EQ} (blocking assignments) and
{\tt AST\_CASE} (\lstinline[language=Verilog]{if} or \lstinline[language=Verilog]{case} statement).

\subsubsection{Handling of Nonblocking Assignments}

When an {\tt AST\_ASSIGN\_LE} node is discovered, the following actions are performed by the
ProcessGenerator:

\begin{itemize}
\item The left-hand-side is evaluated using \lstinline[language=C++]{AST::AstNode::genRTLIL()} and mapped to
a temporary signal name using \lstinline[language=C++]{subst_lvalue_from} and \lstinline[language=C++]{subst_lvalue_to}.
%
\item The right-hand-side is evaluated using \lstinline[language=C++]{AST::AstNode::genRTLIL()}. For this call,
the values of \lstinline[language=C++]{subst_rvalue_from} and \lstinline[language=C++]{subst_rvalue_to} are used to
map blocking-assigned signals correctly.
%
\item Remove all assignments to the same left-hand-side as this assignment from the \lstinline[language=C++]{current_case}
and all cases within it.
%
\item Add the new assignment to the \lstinline[language=C++]{current_case}.
\end{itemize}

\subsubsection{Handling of Blocking Assignments}

When an {\tt AST\_ASSIGN\_EQ} node is discovered, the following actions are performed by
the ProcessGenerator:

\begin{itemize}
\item Perform all the steps that would be performed for a nonblocking assignment (see above).
%
\item Remove the found left-hand-side (before lvalue mapping) from
\lstinline[language=C++]{subst_rvalue_from} and also remove the respective
bits from \lstinline[language=C++]{subst_rvalue_to}.
%
\item Append the found left-hand-side (before lvalue mapping) to \lstinline[language=C++]{subst_rvalue_from}
and append the found right-hand-side to \lstinline[language=C++]{subst_rvalue_to}.
\end{itemize}

\subsubsection{Handling of Cases and if-Statements}

\begin{sloppypar}
When an {\tt AST\_CASE} node is discovered, the following actions are performed by
the ProcessGenerator:

\begin{itemize}
\item The values of \lstinline[language=C++]{subst_rvalue_from}, \lstinline[language=C++]{subst_rvalue_to},
\lstinline[language=C++]{subst_lvalue_from} and \lstinline[language=C++]{subst_lvalue_to} are pushed to the stack.
%
\item A new \lstinline[language=C++]{RTLIL::SwitchRule} object is generated, the selection expression is evaluated using
\lstinline[language=C++]{AST::AstNode::genRTLIL()} (with the use of \lstinline[language=C++]{subst_rvalue_from} and
\lstinline[language=C++]{subst_rvalue_to}) and added to the \lstinline[language=C++]{RTLIL::SwitchRule} object and the
object is added to the \lstinline[language=C++]{current_case}.
%
\item All lvalues assigned to within the {\tt AST\_CASE} node using blocking assignments are collected and
saved in the local variable \lstinline[language=C++]{this_case_eq_lvalue}.
%
\item New temporary signals are generated for all signals in \lstinline[language=C++]{this_case_eq_lvalue} and stored
in \lstinline[language=C++]{this_case_eq_ltemp}.
%
\item The signals in \lstinline[language=C++]{this_case_eq_lvalue} are mapped using \lstinline[language=C++]{subst_rvalue_from}
and \lstinline[language=C++]{subst_rvalue_to} and the resulting set of signals is stored in
\lstinline[language=C++]{this_case_eq_rvalue}.
\end{itemize}

Then the following steps are performed for each {\tt AST\_COND} node within the {\tt AST\_CASE} node:

\begin{itemize}
\item Set \lstinline[language=C++]{subst_rvalue_from}, \lstinline[language=C++]{subst_rvalue_to},
\lstinline[language=C++]{subst_lvalue_from} and \lstinline[language=C++]{subst_lvalue_to} to the values
that have been pushed to the stack.
%
\item Remove \lstinline[language=C++]{this_case_eq_lvalue} from
\lstinline[language=C++]{subst_lvalue_from}/\lstinline[language=C++]{subst_lvalue_to}.
%
\item Append \lstinline[language=C++]{this_case_eq_lvalue} to \lstinline[language=C++]{subst_lvalue_from} and append
\lstinline[language=C++]{this_case_eq_ltemp} to \lstinline[language=C++]{subst_lvalue_to}.
%
\item Push the value of \lstinline[language=C++]{current_case}.
%
\item Create a new \lstinline[language=C++]{RTLIL::CaseRule}. Set \lstinline[language=C++]{current_case} to the
new object and add the new object to the \lstinline[language=C++]{RTLIL::SwitchRule} created above.
%
\item Add an assignment from \lstinline[language=C++]{this_case_eq_rvalue} to \lstinline[language=C++]{this_case_eq_ltemp}
to the new \lstinline[language=C++]{current_case}.
%
\item Evaluate the compare value for this case using \lstinline[language=C++]{AST::AstNode::genRTLIL()} (with the use of
\lstinline[language=C++]{subst_rvalue_from} and \lstinline[language=C++]{subst_rvalue_to}) modify the new
\lstinline[language=C++]{current_case} accordingly.
%
\item Recursion into the children of the {\tt AST\_COND} node.
%
\item Restore \lstinline[language=C++]{current_case} by popping the old value from the stack.
\end{itemize}

Finally the following steps are performed:

\begin{itemize}
\item The values of \lstinline[language=C++]{subst_rvalue_from}, \lstinline[language=C++]{subst_rvalue_to},
\lstinline[language=C++]{subst_lvalue_from} and \lstinline[language=C++]{subst_lvalue_to} are popped from the stack.
%
\item The signals from \lstinline[language=C++]{this_case_eq_lvalue} are removed from the
\lstinline[language=C++]{subst_rvalue_from}/\lstinline[language=C++]{subst_rvalue_to}-pair.
%
\item The value of \lstinline[language=C++]{this_case_eq_lvalue} is appended to \lstinline[language=C++]{subst_rvalue_from}
and the value of \lstinline[language=C++]{this_case_eq_ltemp} is appended to \lstinline[language=C++]{subst_rvalue_to}.
%
\item Map the signals in \lstinline[language=C++]{this_case_eq_lvalue} using
\lstinline[language=C++]{subst_lvalue_from}/\lstinline[language=C++]{subst_lvalue_to}.
%
\item Remove all assignments to signals in \lstinline[language=C++]{this_case_eq_lvalue} in \lstinline[language=C++]{current_case}
and all cases within it.
%
\item Add an assignment from \lstinline[language=C++]{this_case_eq_ltemp} to \lstinline[language=C++]{this_case_eq_lvalue}
to \lstinline[language=C++]{current_case}.
\end{itemize}
\end{sloppypar}

\subsubsection{Further Analysis of the Algorithm for Cases and if-Statements}

With respect to nonblocking assignments the algorithm is easy: later assignments invalidate earlier assignments.
For each signal assigned using nonblocking assignments exactly one temporary variable is generated (with the
{\tt \$0}-prefix) and this variable is used for all assignments of the variable.

Note how all the \lstinline[language=C++]{_eq_}-variables become empty when no blocking assignments are used
and many of the steps in the algorithm can then be ignored as a result of this.

For a variable with blocking assignments the algorithm shows the following behaviour: First a new temporary variable
is created. This new temporary variable is then registered as the assignment target for all assignments for this
variable within the cases for this {\tt AST\_CASE} node. Then for each case the new temporary variable is first
assigned the old temporary variable. This assignment is overwritten if the variable is actually assigned in this
case and is kept as a default value otherwise.

This yields an \lstinline[language=C++]{RTLIL::CaseRule} that assigns the new temporary variable in all branches.
So when all cases have been processed a final assignment is added to the containing block that assigns the new
temporary variable to the old one. Note how this step always overrides a previous assignment to the old temporary
variable. Other than nonblocking assignments, the old assignment could still have an effect somewhere
in the design, as there have been calls to \lstinline[language=C++]{AST::AstNode::genRTLIL()} with a
\lstinline[language=C++]{subst_rvalue_from}/\lstinline[language=C++]{subst_rvalue_to}-tuple that contained
the right-hand-side of the old assignment.

\subsection{The proc pass}

The ProcessGenerator converts a behavioural model in AST representation to a behavioural model in
\lstinline[language=C++]{RTLIL::Process} representation. The actual conversion from a behavioural
model to an RTL representation is performed by the {\tt proc} pass and the passes it launches:

\begin{itemize}
\item {\tt proc\_clean} and {\tt proc\_rmdead} \\
These two passes just clean up the \lstinline[language=C++]{RTLIL::Process} structure. The {\tt proc\_clean}
pass removes empty parts (eg. empty assignments) from the process and {\tt proc\_rmdead} detects and removes
unreachable branches from the process's decision trees.
%
\item {\tt proc\_arst} \\
This pass detects processes that describe d-type flip-flops with asynchronous
resets and rewrites the process to better reflect what they are modelling:
Before this pass, an asynchronous reset has two edge-sensitive sync rules and
one top-level \C{RTLIL::SwitchRule} for the reset path. After this pass the
sync rule for the reset is level-sensitive and the top-level
\C{RTLIL::SwitchRule} has been removed.
%
\item {\tt proc\_mux} \\
This pass converts the \C{RTLIL::CaseRule}/\C{RTLIL::SwitchRule}-tree to a tree
of multiplexers per written signal. After this, the \C{RTLIL::Process} structure only contains
the \C{RTLIL::SyncRule}s that describe the output registers.
%
\item {\tt proc\_dff} \\
This pass replaces the \C{RTLIL::SyncRule}s to d-type flip-flops (with
asynchronous resets if necessary).
%
\item {\tt proc\_clean} \\
A final call to {\tt proc\_clean} removes the now empty \C{RTLIL::Process} objects.
\end{itemize}

Performing these last processing steps in passes instead of in the Verilog frontend has two important benefits:

First it improves the transparency of the process. Everything that happens in a separate pass is easier to debug,
as the RTLIL data structures can be easily investigated before and after each of the steps.

Second it improves flexibility. This scheme can easily be extended to support other types of storage-elements, such
as sr-latches or d-latches, without having to extend the actual Verilog frontend.

\section{Synthesizing Verilog Arrays}

\begin{fixme}
Add some information on the generation of {\tt \$memrd} and {\tt \$memwr} cells
and how they are processed in the {\tt memory} pass.
\end{fixme}

\section{Synthesizing Parametric Designs}

\begin{fixme}
Add some information on the \lstinline[language=C++]{RTLIL::Module::derive()} method and how it
is used to synthesize parametric modules via the {\tt hierarchy} pass.
\end{fixme}


\include{CHAPTER_Optimize}
\include{CHAPTER_Techmap}
% \include{CHAPTER_Eval}

\appendix

\include{CHAPTER_Auxlibs}
\include{CHAPTER_Auxprogs}

\chapter{Command Reference Manual}
\label{commandref}
% Generated using the yosys 'help -write-tex-command-reference-manual' command.

\section{abc -- use ABC for technology mapping}
\label{cmd:abc}
\begin{lstlisting}[numbers=left,frame=single]
    abc [options] [selection]

This pass uses the ABC tool [1] for technology mapping of yosys's internal gate
library to a target architecture.

    -exe <command>
        use the specified command instead of "<yosys-bindir>/yosys-abc" to execute ABC.
        This can e.g. be used to call a specific version of ABC or a wrapper.

    -script <file>
        use the specified ABC script file instead of the default script.

        if <file> starts with a plus sign (+), then the rest of the filename
        string is interpreted as the command string to be passed to ABC. The
        leading plus sign is removed and all commas (,) in the string are
        replaced with blanks before the string is passed to ABC.

        if no -script parameter is given, the following scripts are used:

        for -liberty without -constr:
          strash; ifraig; scorr; dc2; dretime; strash; &get -n; &dch -f;
               &nf {D}; &put

        for -liberty with -constr:
          strash; ifraig; scorr; dc2; dretime; strash; &get -n; &dch -f;
               &nf {D}; &put; buffer; upsize {D}; dnsize {D}; stime -p

        for -lut/-luts (only one LUT size):
          strash; ifraig; scorr; dc2; dretime; strash; dch -f; if; mfs2;
               lutpack {S}

        for -lut/-luts (different LUT sizes):
          strash; ifraig; scorr; dc2; dretime; strash; dch -f; if; mfs2

        for -sop:
          strash; ifraig; scorr; dc2; dretime; strash; dch -f;
               cover {I} {P}

        otherwise:
          strash; ifraig; scorr; dc2; dretime; strash; &get -n; &dch -f;
               &nf {D}; &put

    -fast
        use different default scripts that are slightly faster (at the cost
        of output quality):

        for -liberty without -constr:
          strash; dretime; map {D}

        for -liberty with -constr:
          strash; dretime; map {D}; buffer; upsize {D}; dnsize {D};
               stime -p

        for -lut/-luts:
          strash; dretime; if

        for -sop:
          strash; dretime; cover -I {I} -P {P}

        otherwise:
          strash; dretime; map

    -liberty <file>
        generate netlists for the specified cell library (using the liberty
        file format).

    -constr <file>
        pass this file with timing constraints to ABC. use with -liberty.

        a constr file contains two lines:
            set_driving_cell <cell_name>
            set_load <floating_point_number>

        the set_driving_cell statement defines which cell type is assumed to
        drive the primary inputs and the set_load statement sets the load in
        femtofarads for each primary output.

    -D <picoseconds>
        set delay target. the string {D} in the default scripts above is
        replaced by this option when used, and an empty string otherwise.
        this also replaces 'dretime' with 'dretime; retime -o {D}' in the
        default scripts above.

    -I <num>
        maximum number of SOP inputs.
        (replaces {I} in the default scripts above)

    -P <num>
        maximum number of SOP products.
        (replaces {P} in the default scripts above)

    -S <num>
        maximum number of LUT inputs shared.
        (replaces {S} in the default scripts above, default: -S 1)

    -lut <width>
        generate netlist using luts of (max) the specified width.

    -lut <w1>:<w2>
        generate netlist using luts of (max) the specified width <w2>. All
        luts with width <= <w1> have constant cost. for luts larger than <w1>
        the area cost doubles with each additional input bit. the delay cost
        is still constant for all lut widths.

    -luts <cost1>,<cost2>,<cost3>,<sizeN>:<cost4-N>,..
        generate netlist using luts. Use the specified costs for luts with 1,
        2, 3, .. inputs.

    -sop
        map to sum-of-product cells and inverters

    -g type1,type2,...
        Map to the specified list of gate types. Supported gates types are:
           AND, NAND, OR, NOR, XOR, XNOR, ANDNOT, ORNOT, MUX,
           NMUX, AOI3, OAI3, AOI4, OAI4.
        (The NOT gate is always added to this list automatically.)

        The following aliases can be used to reference common sets of gate types:
          simple: AND OR XOR MUX
          cmos2:  NAND NOR
          cmos3:  NAND NOR AOI3 OAI3
          cmos4:  NAND NOR AOI3 OAI3 AOI4 OAI4
          cmos:   NAND NOR AOI3 OAI3 AOI4 OAI4 NMUX MUX XOR XNOR
          gates:  AND NAND OR NOR XOR XNOR ANDNOT ORNOT
          aig:    AND NAND OR NOR ANDNOT ORNOT

        The alias 'all' represent the full set of all gate types.

        Prefix a gate type with a '-' to remove it from the list. For example
        the arguments 'AND,OR,XOR' and 'simple,-MUX' are equivalent.

        The default is 'all,-NMUX,-AOI3,-OAI3,-AOI4,-OAI4'.

    -dff
        also pass $_DFF_?_ and $_DFFE_??_ cells through ABC. modules with many
        clock domains are automatically partitioned in clock domains and each
        domain is passed through ABC independently.

    -clk [!]<clock-signal-name>[,[!]<enable-signal-name>]
        use only the specified clock domain. this is like -dff, but only FF
        cells that belong to the specified clock domain are used.

    -keepff
        set the "keep" attribute on flip-flop output wires. (and thus preserve
        them, for example for equivalence checking.)

    -nocleanup
        when this option is used, the temporary files created by this pass
        are not removed. this is useful for debugging.

    -showtmp
        print the temp dir name in log. usually this is suppressed so that the
        command output is identical across runs.

    -markgroups
        set a 'abcgroup' attribute on all objects created by ABC. The value of
        this attribute is a unique integer for each ABC process started. This
        is useful for debugging the partitioning of clock domains.

    -dress
        run the 'dress' command after all other ABC commands. This aims to
        preserve naming by an equivalence check between the original and post-ABC
        netlists (experimental).

When neither -liberty nor -lut is used, the Yosys standard cell library is
loaded into ABC before the ABC script is executed.

Note that this is a logic optimization pass within Yosys that is calling ABC
internally. This is not going to "run ABC on your design". It will instead run
ABC on logic snippets extracted from your design. You will not get any useful
output when passing an ABC script that writes a file. Instead write your full
design as BLIF file with write_blif and then load that into ABC externally if
you want to use ABC to convert your design into another format.

[1] http://www.eecs.berkeley.edu/~alanmi/abc/
\end{lstlisting}

\section{abc9 -- use ABC9 for technology mapping}
\label{cmd:abc9}
\begin{lstlisting}[numbers=left,frame=single]
    abc9 [options] [selection]

This script pass performs a sequence of commands to facilitate the use of the ABC
tool [1] for technology mapping of the current design to a target FPGA
architecture. Only fully-selected modules are supported.

    -run <from_label>:<to_label>
        only run the commands between the labels (see below). an empty
        from label is synonymous to 'begin', and empty to label is
        synonymous to the end of the command list.

    -exe <command>
        use the specified command instead of "<yosys-bindir>/yosys-abc" to execute ABC.
        This can e.g. be used to call a specific version of ABC or a wrapper.

    -script <file>
        use the specified ABC script file instead of the default script.

        if <file> starts with a plus sign (+), then the rest of the filename
        string is interpreted as the command string to be passed to ABC. The
        leading plus sign is removed and all commas (,) in the string are
        replaced with blanks before the string is passed to ABC.

        if no -script parameter is given, the following scripts are used:
          &scorr; &sweep; &dc2; &dch -f; &ps; &if {C} {W} {D} {R} -v; &mfs

    -fast
        use different default scripts that are slightly faster (at the cost
        of output quality):
          &if {C} {W} {D} {R} -v

    -D <picoseconds>
        set delay target. the string {D} in the default scripts above is
        replaced by this option when used, and an empty string otherwise
        (indicating best possible delay).

    -lut <width>
        generate netlist using luts of (max) the specified width.

    -lut <w1>:<w2>
        generate netlist using luts of (max) the specified width <w2>. All
        luts with width <= <w1> have constant cost. for luts larger than <w1>
        the area cost doubles with each additional input bit. the delay cost
        is still constant for all lut widths.

    -lut <file>
        pass this file with lut library to ABC.

    -luts <cost1>,<cost2>,<cost3>,<sizeN>:<cost4-N>,..
        generate netlist using luts. Use the specified costs for luts with 1,
        2, 3, .. inputs.

    -maxlut <width>
        when auto-generating the lut library, discard all luts equal to or
        greater than this size (applicable when neither -lut nor -luts is
        specified).

    -dff
        also pass $_ABC9_FF_ cells through to ABC. modules with many clock
        domains are marked as such and automatically partitioned by ABC.

    -nocleanup
        when this option is used, the temporary files created by this pass
        are not removed. this is useful for debugging.

    -showtmp
        print the temp dir name in log. usually this is suppressed so that the
        command output is identical across runs.

    -box <file>
        pass this file with box library to ABC.

Note that this is a logic optimization pass within Yosys that is calling ABC
internally. This is not going to "run ABC on your design". It will instead run
ABC on logic snippets extracted from your design. You will not get any useful
output when passing an ABC script that writes a file. Instead write your full
design as an XAIGER file with `write_xaiger' and then load that into ABC
externally if you want to use ABC to convert your design into another format.

[1] http://www.eecs.berkeley.edu/~alanmi/abc/


    pre:
        abc9_ops -check
        scc -set_attr abc9_scc_id {}
        abc9_ops -mark_scc -prep_delays -prep_xaiger [-dff]    (option for -dff)
        abc9_ops -prep_lut <maxlut>    (skip if -lut or -luts)
        abc9_ops -prep_box [-dff]    (skip if -box)
        select -set abc9_holes A:abc9_holes
        flatten -wb @abc9_holes
        techmap @abc9_holes
        abc9_ops -prep_dff    (only if -dff)
        opt -purge @abc9_holes
        aigmap
        wbflip @abc9_holes

    map:
        foreach module in selection
            abc9_ops -write_lut <abc-temp-dir>/input.lut    (skip if '-lut' or '-luts')
            abc9_ops -write_box <abc-temp-dir>/input.box
            write_xaiger -map <abc-temp-dir>/input.sym <abc-temp-dir>/input.xaig
            abc9_exe [options] -cwd <abc-temp-dir> [-lut <abc-temp-dir>/input.lut] -box <abc-temp-dir>/input.box
            read_aiger -xaiger -wideports -module_name <module-name>$abc9 -map <abc-temp-dir>/input.sym <abc-temp-dir>/output.aig
            abc9_ops -reintegrate
\end{lstlisting}

\section{abc9\_exe -- use ABC9 for technology mapping}
\label{cmd:abc9_exe}
\begin{lstlisting}[numbers=left,frame=single]
    abc9_exe [options]

 
This pass uses the ABC tool [1] for technology mapping of the top module
(according to the (* top *) attribute or if only one module is currently selected)
to a target FPGA architecture.

    -exe <command>
        use the specified command instead of "<yosys-bindir>/yosys-abc" to execute ABC.
        This can e.g. be used to call a specific version of ABC or a wrapper.

    -script <file>
        use the specified ABC script file instead of the default script.

        if <file> starts with a plus sign (+), then the rest of the filename
        string is interpreted as the command string to be passed to ABC. The
        leading plus sign is removed and all commas (,) in the string are
        replaced with blanks before the string is passed to ABC.

        if no -script parameter is given, the following scripts are used:
          &scorr; &sweep; &dc2; &dch -f; &ps; &if {C} {W} {D} {R} -v; &mfs

    -fast
        use different default scripts that are slightly faster (at the cost
        of output quality):
          &if {C} {W} {D} {R} -v

    -D <picoseconds>
        set delay target. the string {D} in the default scripts above is
        replaced by this option when used, and an empty string otherwise
        (indicating best possible delay).

    -lut <width>
        generate netlist using luts of (max) the specified width.

    -lut <w1>:<w2>
        generate netlist using luts of (max) the specified width <w2>. All
        luts with width <= <w1> have constant cost. for luts larger than <w1>
        the area cost doubles with each additional input bit. the delay cost
        is still constant for all lut widths.

    -lut <file>
        pass this file with lut library to ABC.

    -luts <cost1>,<cost2>,<cost3>,<sizeN>:<cost4-N>,..
        generate netlist using luts. Use the specified costs for luts with 1,
        2, 3, .. inputs.

    -showtmp
        print the temp dir name in log. usually this is suppressed so that the
        command output is identical across runs.

    -box <file>
        pass this file with box library to ABC.

    -cwd <dir>
        use this as the current working directory, inside which the 'input.xaig'
        file is expected. temporary files will be created in this directory, and
        the mapped result will be written to 'output.aig'.

Note that this is a logic optimization pass within Yosys that is calling ABC
internally. This is not going to "run ABC on your design". It will instead run
ABC on logic snippets extracted from your design. You will not get any useful
output when passing an ABC script that writes a file. Instead write your full
design as BLIF file with write_blif and then load that into ABC externally if
you want to use ABC to convert your design into another format.

[1] http://www.eecs.berkeley.edu/~alanmi/abc/
\end{lstlisting}

\section{abc9\_ops -- helper functions for ABC9}
\label{cmd:abc9_ops}
\begin{lstlisting}[numbers=left,frame=single]
    abc9_ops [options] [selection]

This pass contains a set of supporting operations for use during ABC technology
mapping, and is expected to be called in conjunction with other operations from
the `abc9' script pass. Only fully-selected modules are supported.

    -check
        check that the design is valid, e.g. (* abc9_box_id *) values are unique,
        (* abc9_carry *) is only given for one input/output port, etc.

    -prep_delays
        insert `$__ABC9_DELAY' blackbox cells into the design to account for
        certain required times.

    -mark_scc
        for an arbitrarily chosen cell in each unique SCC of each selected module
        (tagged with an (* abc9_scc_id = <int> *) attribute), temporarily mark all
        wires driven by this cell's outputs with a (* keep *) attribute in order
        to break the SCC. this temporary attribute will be removed on -reintegrate.

    -prep_xaiger
        prepare the design for XAIGER output. this includes computing the
        topological ordering of ABC9 boxes, as well as preparing the
        '<module-name>$holes' module that contains the logic behaviour of ABC9
        whiteboxes.

    -dff
        consider flop cells (those instantiating modules marked with (* abc9_flop *))
        during -prep_{delays,xaiger,box}.

    -prep_dff
        compute the clock domain and initial value of each flop in the design.
        process the '$holes' module to support clock-enable functionality.

    -prep_lut <maxlut>
        pre-compute the lut library by analysing all modules marked with
        (* abc9_lut=<area> *).

    -write_lut <dst>
        write the pre-computed lut library to <dst>.

    -prep_box
        pre-compute the box library by analysing all modules marked with
        (* abc9_box *).

    -write_box <dst>
        write the pre-computed box library to <dst>.

    -reintegrate
        for each selected module, re-intergrate the module '<module-name>$abc9'
        by first recovering ABC9 boxes, and then stitching in the remaining primary
        inputs and outputs.
\end{lstlisting}

\section{add -- add objects to the design}
\label{cmd:add}
\begin{lstlisting}[numbers=left,frame=single]
    add <command> [selection]

This command adds objects to the design. It operates on all fully selected
modules. So e.g. 'add -wire foo' will add a wire foo to all selected modules.


    add {-wire|-input|-inout|-output} <name> <width> [selection]

Add a wire (input, inout, output port) with the given name and width. The
command will fail if the object exists already and has different properties
than the object to be created.


    add -global_input <name> <width> [selection]

Like 'add -input', but also connect the signal between instances of the
selected modules.


    add {-assert|-assume|-live|-fair|-cover} <name1> [-if <name2>]

Add an $assert, $assume, etc. cell connected to a wire named name1, with its
enable signal optionally connected to a wire named name2 (default: 1'b1).


    add -mod <name[s]>

Add module[s] with the specified name[s].
\end{lstlisting}

\section{aigmap -- map logic to and-inverter-graph circuit}
\label{cmd:aigmap}
\begin{lstlisting}[numbers=left,frame=single]
    aigmap [options] [selection]

Replace all logic cells with circuits made of only $_AND_ and
$_NOT_ cells.

    -nand
        Enable creation of $_NAND_ cells

    -select
        Overwrite replaced cells in the current selection with new $_AND_,
        $_NOT_, and $_NAND_, cells
\end{lstlisting}

\section{alumacc -- extract ALU and MACC cells}
\label{cmd:alumacc}
\begin{lstlisting}[numbers=left,frame=single]
    alumacc [selection]

This pass translates arithmetic operations like $add, $mul, $lt, etc. to $alu
and $macc cells.
\end{lstlisting}

\section{anlogic\_eqn -- Anlogic: Calculate equations for luts}
\label{cmd:anlogic_eqn}
\begin{lstlisting}[numbers=left,frame=single]
    anlogic_eqn [selection]

Calculate equations for luts since bitstream generator depends on it.
\end{lstlisting}

\section{anlogic\_fixcarry -- Anlogic: fix carry chain}
\label{cmd:anlogic_fixcarry}
\begin{lstlisting}[numbers=left,frame=single]
    anlogic_fixcarry [options] [selection]

Add Anlogic adders to fix carry chain if needed.
\end{lstlisting}

\section{assertpmux -- adds asserts for parallel muxes}
\label{cmd:assertpmux}
\begin{lstlisting}[numbers=left,frame=single]
    assertpmux [options] [selection]

This command adds asserts to the design that assert that all parallel muxes
($pmux cells) have a maximum of one of their inputs enable at any time.

    -noinit
        do not enforce the pmux condition during the init state

    -always
        usually the $pmux condition is only checked when the $pmux output
        is used by the mux tree it drives. this option will deactivate this
        additional constraint and check the $pmux condition always.
\end{lstlisting}

\section{async2sync -- convert async FF inputs to sync circuits}
\label{cmd:async2sync}
\begin{lstlisting}[numbers=left,frame=single]
    async2sync [options] [selection]

This command replaces async FF inputs with sync circuits emulating the same
behavior for when the async signals are actually synchronized to the clock.

This pass assumes negative hold time for the async FF inputs. For example when
a reset deasserts with the clock edge, then the FF output will still drive the
reset value in the next cycle regardless of the data-in value at the time of
the clock edge.

Currently only $adff, $dffsr, and $dlatch cells are supported by this pass.
\end{lstlisting}

\section{attrmap -- renaming attributes}
\label{cmd:attrmap}
\begin{lstlisting}[numbers=left,frame=single]
    attrmap [options] [selection]

This command renames attributes and/or maps key/value pairs to
other key/value pairs.

    -tocase <name>
        Match attribute names case-insensitively and set it to the specified
        name.

    -rename <old_name> <new_name>
        Rename attributes as specified

    -map <old_name>=<old_value> <new_name>=<new_value>
        Map key/value pairs as indicated.

    -imap <old_name>=<old_value> <new_name>=<new_value>
        Like -map, but use case-insensitive match for <old_value> when
        it is a string value.

    -remove <name>=<value>
        Remove attributes matching this pattern.

    -modattr
        Operate on module attributes instead of attributes on wires and cells.

For example, mapping Xilinx-style "keep" attributes to Yosys-style:

    attrmap -tocase keep -imap keep="true" keep=1 \
            -imap keep="false" keep=0 -remove keep=0
\end{lstlisting}

\section{attrmvcp -- move or copy attributes from wires to driving cells}
\label{cmd:attrmvcp}
\begin{lstlisting}[numbers=left,frame=single]
    attrmvcp [options] [selection]

Move or copy attributes on wires to the cells driving them.

    -copy
        By default, attributes are moved. This will only add
        the attribute to the cell, without removing it from
        the wire.

    -purge
        If no selected cell consumes the attribute, then it is
        left on the wire by default. This option will cause the
        attribute to be removed from the wire, even if no selected
        cell takes it.

    -driven
        By default, attriburtes are moved to the cell driving the
        wire. With this option set it will be moved to the cell
        driven by the wire instead.

    -attr <attrname>
        Move or copy this attribute. This option can be used
        multiple times.
\end{lstlisting}

\section{autoname -- automatically assign names to objects}
\label{cmd:autoname}
\begin{lstlisting}[numbers=left,frame=single]
    autoname [selection]

Assign auto-generated public names to objects with private names (the ones
with $-prefix).
\end{lstlisting}

\section{blackbox -- convert modules into blackbox modules}
\label{cmd:blackbox}
\begin{lstlisting}[numbers=left,frame=single]
    blackbox [options] [selection]

Convert modules into blackbox modules (remove contents and set the blackbox
module attribute).
\end{lstlisting}

\section{bugpoint -- minimize testcases}
\label{cmd:bugpoint}
\begin{lstlisting}[numbers=left,frame=single]
    bugpoint [options]

This command minimizes testcases that crash Yosys. It removes an arbitrary part
of the design and recursively invokes Yosys with a given script, repeating these
steps while it can find a smaller design that still causes a crash. Once this
command finishes, it replaces the current design with the smallest testcase it
was able to produce.

It is possible to specify the kinds of design part that will be removed. If none
are specified, all parts of design will be removed.

    -yosys <filename>
        use this Yosys binary. if not specified, `yosys` is used.

    -script <filename>
        use this script to crash Yosys. required.

    -grep <string>
        only consider crashes that place this string in the log file.

    -fast
        run `proc_clean; clean -purge` after each minimization step. converges
        faster, but produces larger testcases, and may fail to produce any
        testcase at all if the crash is related to dangling wires.

    -clean
        run `proc_clean; clean -purge` before checking testcase and after
        finishing. produces smaller and more useful testcases, but may fail to
        produce any testcase at all if the crash is related to dangling wires.

    -modules
        try to remove modules.

    -ports
        try to remove module ports.

    -cells
        try to remove cells.

    -connections
        try to reconnect ports to 'x.

    -assigns
        try to remove process assigns from cases.

    -updates
        try to remove process updates from syncs.
\end{lstlisting}

\section{cd -- a shortcut for 'select -module <name>'}
\label{cmd:cd}
\begin{lstlisting}[numbers=left,frame=single]
    cd <modname>

This is just a shortcut for 'select -module <modname>'.


    cd <cellname>

When no module with the specified name is found, but there is a cell
with the specified name in the current module, then this is equivalent
to 'cd <celltype>'.

    cd ..

Remove trailing substrings that start with '.' in current module name until
the name of a module in the current design is generated, then switch to that
module. Otherwise clear the current selection.

    cd

This is just a shortcut for 'select -clear'.
\end{lstlisting}

\section{check -- check for obvious problems in the design}
\label{cmd:check}
\begin{lstlisting}[numbers=left,frame=single]
    check [options] [selection]

This pass identifies the following problems in the current design:

 - combinatorial loops

 - two or more conflicting drivers for one wire

 - used wires that do not have a driver

Options:

  -noinit
    Also check for wires which have the 'init' attribute set.

  -initdrv
    Also check for wires that have the 'init' attribute set and are not
    driven by an FF cell type.

  -mapped
    Also check for internal cells that have not been mapped to cells of the
    target architecture.

  -allow-tbuf
    Modify the -mapped behavior to still allow $_TBUF_ cells.

  -assert
    Produce a runtime error if any problems are found in the current design.
\end{lstlisting}

\section{chformal -- change formal constraints of the design}
\label{cmd:chformal}
\begin{lstlisting}[numbers=left,frame=single]
    chformal [types] [mode] [options] [selection]

Make changes to the formal constraints of the design. The [types] options
the type of constraint to operate on. If none of the following options are given,
the command will operate on all constraint types:

    -assert       $assert cells, representing assert(...) constraints
    -assume       $assume cells, representing assume(...) constraints
    -live         $live cells, representing assert(s_eventually ...)
    -fair         $fair cells, representing assume(s_eventually ...)
    -cover        $cover cells, representing cover() statements

Exactly one of the following modes must be specified:

    -remove
        remove the cells and thus constraints from the design

    -early
        bypass FFs that only delay the activation of a constraint

    -delay <N>
        delay activation of the constraint by <N> clock cycles

    -skip <N>
        ignore activation of the constraint in the first <N> clock cycles

    -assert2assume
    -assume2assert
    -live2fair
    -fair2live
        change the roles of cells as indicated. these options can be combined
\end{lstlisting}

\section{chparam -- re-evaluate modules with new parameters}
\label{cmd:chparam}
\begin{lstlisting}[numbers=left,frame=single]
    chparam [ -set name value ]... [selection]

Re-evaluate the selected modules with new parameters. String values must be
passed in double quotes (").


    chparam -list [selection]

List the available parameters of the selected modules.
\end{lstlisting}

\section{chtype -- change type of cells in the design}
\label{cmd:chtype}
\begin{lstlisting}[numbers=left,frame=single]
    chtype [options] [selection]

Change the types of cells in the design.

    -set <type>
        set the cell type to the given type

    -map <old_type> <new_type>
        change cells types that match <old_type> to <new_type>
\end{lstlisting}

\section{clean -- remove unused cells and wires}
\label{cmd:clean}
\begin{lstlisting}[numbers=left,frame=single]
    clean [options] [selection]

This is identical to 'opt_clean', but less verbose.

When commands are separated using the ';;' token, this command will be executed
between the commands.

When commands are separated using the ';;;' token, this command will be executed
in -purge mode between the commands.
\end{lstlisting}

\section{clk2fflogic -- convert clocked FFs to generic \$ff cells}
\label{cmd:clk2fflogic}
\begin{lstlisting}[numbers=left,frame=single]
    clk2fflogic [options] [selection]

This command replaces clocked flip-flops with generic $ff cells that use the
implicit global clock. This is useful for formal verification of designs with
multiple clocks.
\end{lstlisting}

\section{clkbufmap -- insert global buffers on clock networks}
\label{cmd:clkbufmap}
\begin{lstlisting}[numbers=left,frame=single]
    clkbufmap [options] [selection]

Inserts global buffers between nets connected to clock inputs and their drivers.

In the absence of any selection, all wires without the 'clkbuf_inhibit'
attribute will be considered for global buffer insertion.
Alternatively, to consider all wires without the 'buffer_type' attribute set to
'none' or 'bufr' one would specify:
  'w:* a:buffer_type=none a:buffer_type=bufr %u %d'
as the selection.

    -buf <celltype> <portname_out>:<portname_in>
        Specifies the cell type to use for the global buffers
        and its port names.  The first port will be connected to
        the clock network sinks, and the second will be connected
        to the actual clock source.  This option is required.

    -inpad <celltype> <portname_out>:<portname_in>
        If specified, a PAD cell of the given type is inserted on
        clock nets that are also top module's inputs (in addition
        to the global buffer).
\end{lstlisting}

\section{connect -- create or remove connections}
\label{cmd:connect}
\begin{lstlisting}[numbers=left,frame=single]
    connect [-nomap] [-nounset] -set <lhs-expr> <rhs-expr>

Create a connection. This is equivalent to adding the statement 'assign
<lhs-expr> = <rhs-expr>;' to the Verilog input. Per default, all existing
drivers for <lhs-expr> are unconnected. This can be overwritten by using
the -nounset option.


    connect [-nomap] -unset <expr>

Unconnect all existing drivers for the specified expression.


    connect [-nomap] -port <cell> <port> <expr>

Connect the specified cell port to the specified cell port.


Per default signal alias names are resolved and all signal names are mapped
the the signal name of the primary driver. Using the -nomap option deactivates
this behavior.

The connect command operates in one module only. Either only one module must
be selected or an active module must be set using the 'cd' command.

This command does not operate on module with processes.
\end{lstlisting}

\section{connect\_rpc -- connect to RPC frontend}
\label{cmd:connect_rpc}
\begin{lstlisting}[numbers=left,frame=single]
    connect_rpc -exec <command> [args...]
    connect_rpc -path <path>

Load modules using an out-of-process frontend.

    -exec <command> [args...]
        run <command> with arguments [args...]. send requests on stdin, read
        responses from stdout.

    -path <path>
        connect to Unix domain socket at <path>. (Unix)
        connect to bidirectional byte-type named pipe at <path>. (Windows)

A simple JSON-based, newline-delimited protocol is used for communicating with
the frontend. Yosys requests data from the frontend by sending exactly 1 line
of JSON. Frontend responds with data or error message by replying with exactly
1 line of JSON as well.

    -> {"method": "modules"}
    <- {"modules": ["<module-name>", ...]}
    <- {"error": "<error-message>"}
        request for the list of modules that can be derived by this frontend.
        the 'hierarchy' command will call back into this frontend if a cell
        with type <module-name> is instantiated in the design.

    -> {"method": "derive", "module": "<module-name">, "parameters": {
        "<param-name>": {"type": "[unsigned|signed|string|real]",
                           "value": "<param-value>"}, ...}}
    <- {"frontend": "[ilang|verilog|...]","source": "<source>"}}
    <- {"error": "<error-message>"}
        request for the module <module-name> to be derived for a specific set of
        parameters. <param-name> starts with \ for named parameters, and with $
        for unnamed parameters, which are numbered starting at 1.<param-value>
        for integer parameters is always specified as a binary string of unlimited
        precision. the <source> returned by the frontend is hygienically parsed
        by a built-in Yosys <frontend>, allowing the RPC frontend to return any
        convenient representation of the module. the derived module is cached,
        so the response should be the same whenever the same set of parameters
        is provided.
\end{lstlisting}

\section{connwrappers -- match width of input-output port pairs}
\label{cmd:connwrappers}
\begin{lstlisting}[numbers=left,frame=single]
    connwrappers [options] [selection]

Wrappers are used in coarse-grain synthesis to wrap cells with smaller ports
in wrapper cells with a (larger) constant port size. I.e. the upper bits
of the wrapper output are signed/unsigned bit extended. This command uses this
knowledge to rewire the inputs of the driven cells to match the output of
the driving cell.

    -signed <cell_type> <port_name> <width_param>
    -unsigned <cell_type> <port_name> <width_param>
        consider the specified signed/unsigned wrapper output

    -port <cell_type> <port_name> <width_param> <sign_param>
        use the specified parameter to decide if signed or unsigned

The options -signed, -unsigned, and -port can be specified multiple times.
\end{lstlisting}

\section{coolrunner2\_fixup -- insert necessary buffer cells for CoolRunner-II architecture}
\label{cmd:coolrunner2_fixup}
\begin{lstlisting}[numbers=left,frame=single]
    coolrunner2_fixup [options] [selection]

Insert necessary buffer cells for CoolRunner-II architecture.
\end{lstlisting}

\section{coolrunner2\_sop -- break \$sop cells into ANDTERM/ORTERM cells}
\label{cmd:coolrunner2_sop}
\begin{lstlisting}[numbers=left,frame=single]
    coolrunner2_sop [options] [selection]

Break $sop cells into ANDTERM/ORTERM cells.
\end{lstlisting}

\section{copy -- copy modules in the design}
\label{cmd:copy}
\begin{lstlisting}[numbers=left,frame=single]
    copy old_name new_name

Copy the specified module. Note that selection patterns are not supported
by this command.
\end{lstlisting}

\section{cover -- print code coverage counters}
\label{cmd:cover}
\begin{lstlisting}[numbers=left,frame=single]
    cover [options] [pattern]

Print the code coverage counters collected using the cover() macro in the Yosys
C++ code. This is useful to figure out what parts of Yosys are utilized by a
test bench.

    -q
        Do not print output to the normal destination (console and/or log file)

    -o file
        Write output to this file, truncate if exists.

    -a file
        Write output to this file, append if exists.

    -d dir
        Write output to a newly created file in the specified directory.

When one or more pattern (shell wildcards) are specified, then only counters
matching at least one pattern are printed.


It is also possible to instruct Yosys to print the coverage counters on program
exit to a file using environment variables:

    YOSYS_COVER_DIR="{dir-name}" yosys {args}

        This will create a file (with an auto-generated name) in this
        directory and write the coverage counters to it.

    YOSYS_COVER_FILE="{file-name}" yosys {args}

        This will append the coverage counters to the specified file.


Hint: Use the following AWK command to consolidate Yosys coverage files:

    gawk '{ p[$3] = $1; c[$3] += $2; } END { for (i in p)
      printf "%-60s %10d %s\n", p[i], c[i], i; }' {files} | sort -k3


Coverage counters are only available in Yosys for Linux.
\end{lstlisting}

\section{cutpoint -- adds formal cut points to the design}
\label{cmd:cutpoint}
\begin{lstlisting}[numbers=left,frame=single]
    cutpoint [options] [selection]

This command adds formal cut points to the design.

    -undef
        set cupoint nets to undef (x). the default behavior is to create a
        $anyseq cell and drive the cutpoint net from that
\end{lstlisting}

\section{debug -- run command with debug log messages enabled}
\label{cmd:debug}
\begin{lstlisting}[numbers=left,frame=single]
    debug cmd

Execute the specified command with debug log messages enabled
\end{lstlisting}

\section{delete -- delete objects in the design}
\label{cmd:delete}
\begin{lstlisting}[numbers=left,frame=single]
    delete [selection]

Deletes the selected objects. This will also remove entire modules, if the
whole module is selected.


    delete {-input|-output|-port} [selection]

Does not delete any object but removes the input and/or output flag on the
selected wires, thus 'deleting' module ports.
\end{lstlisting}

\section{deminout -- demote inout ports to input or output}
\label{cmd:deminout}
\begin{lstlisting}[numbers=left,frame=single]
    deminout [options] [selection]

"Demote" inout ports to input or output ports, if possible.
\end{lstlisting}

\section{design -- save, restore and reset current design}
\label{cmd:design}
\begin{lstlisting}[numbers=left,frame=single]
    design -reset

Clear the current design.


    design -save <name>

Save the current design under the given name.


    design -stash <name>

Save the current design under the given name and then clear the current design.


    design -push

Push the current design to the stack and then clear the current design.


    design -push-copy

Push the current design to the stack without clearing the current design.


    design -pop

Reset the current design and pop the last design from the stack.


    design -load <name>

Reset the current design and load the design previously saved under the given
name.


    design -copy-from <name> [-as <new_mod_name>] <selection>

Copy modules from the specified design into the current one. The selection is
evaluated in the other design.


    design -copy-to <name> [-as <new_mod_name>] [selection]

Copy modules from the current design into the specified one.


    design -import <name> [-as <new_top_name>] [selection]

Import the specified design into the current design. The source design must
either have a selected top module or the selection must contain exactly one
module that is then used as top module for this command.


    design -reset-vlog

The Verilog front-end remembers defined macros and top-level declarations
between calls to 'read_verilog'. This command resets this memory.
\end{lstlisting}

\section{determine\_init -- Determine the init value of cells}
\label{cmd:determine_init}
\begin{lstlisting}[numbers=left,frame=single]
    determine_init [selection]

Determine the init value of cells that doesn't allow unknown init value.
\end{lstlisting}

\section{dff2dffe -- transform \$dff cells to \$dffe cells}
\label{cmd:dff2dffe}
\begin{lstlisting}[numbers=left,frame=single]
    dff2dffe [options] [selection]

This pass transforms $dff cells driven by a tree of multiplexers with one or
more feedback paths to $dffe cells. It also works on gate-level cells such as
$_DFF_P_, $_DFF_N_ and $_MUX_.

    -unmap
        operate in the opposite direction: replace $dffe cells with combinations
        of $dff and $mux cells. the options below are ignored in unmap mode.

    -unmap-mince N
        Same as -unmap but only unmap $dffe where the clock enable port
        signal is used by less $dffe than the specified number

    -direct <internal_gate_type> <external_gate_type>
        map directly to external gate type. <internal_gate_type> can
        be any internal gate-level FF cell (except $_DFFE_??_). the
        <external_gate_type> is the cell type name for a cell with an
        identical interface to the <internal_gate_type>, except it
        also has an high-active enable port 'E'.
          Usually <external_gate_type> is an intermediate cell type
        that is then translated to the final type using 'techmap'.

    -direct-match <pattern>
        like -direct for all DFF cell types matching the expression.
        this will use $__DFFE_* as <external_gate_type> matching the
        internal gate type $_DFF_*_, and $__DFFSE_* for those matching
        $_DFFS_*_, except for $_DFF_[NP]_, which is converted to 
        $_DFFE_[NP]_.
\end{lstlisting}

\section{dff2dffs -- process sync set/reset with SR over CE priority}
\label{cmd:dff2dffs}
\begin{lstlisting}[numbers=left,frame=single]
    dff2dffs [options] [selection]

Merge synchronous set/reset $_MUX_ cells to create $__DFFS_[NP][NP][01], to be run before
dff2dffe for SR over CE priority.

    -match-init
        Disallow merging synchronous set/reset that has polarity opposite of the
        output wire's init attribute (if any).
\end{lstlisting}

\section{dffinit -- set INIT param on FF cells}
\label{cmd:dffinit}
\begin{lstlisting}[numbers=left,frame=single]
    dffinit [options] [selection]

This pass sets an FF cell parameter to the the initial value of the net it
drives. (This is primarily used in FPGA flows.)

    -ff <cell_name> <output_port> <init_param>
        operate on the specified cell type. this option can be used
        multiple times.

    -highlow
        use the string values "high" and "low" to represent a single-bit
        initial value of 1 or 0. (multi-bit values are not supported in this
        mode.)

    -strinit <string for high> <string for low> 
        use string values in the command line to represent a single-bit
        initial value of 1 or 0. (multi-bit values are not supported in this
        mode.)

    -noreinit
        fail if the FF cell has already a defined initial value set in other
        passes and the initial value of the net it drives is not equal to
        the already defined initial value.
\end{lstlisting}

\section{dfflibmap -- technology mapping of flip-flops}
\label{cmd:dfflibmap}
\begin{lstlisting}[numbers=left,frame=single]
    dfflibmap [-prepare] -liberty <file> [selection]

Map internal flip-flop cells to the flip-flop cells in the technology
library specified in the given liberty file.

This pass may add inverters as needed. Therefore it is recommended to
first run this pass and then map the logic paths to the target technology.

When called with -prepare, this command will convert the internal FF cells
to the internal cell types that best match the cells found in the given
liberty file.
\end{lstlisting}

\section{dump -- print parts of the design in ilang format}
\label{cmd:dump}
\begin{lstlisting}[numbers=left,frame=single]
    dump [options] [selection]

Write the selected parts of the design to the console or specified file in
ilang format.

    -m
        also dump the module headers, even if only parts of a single
        module is selected

    -n
        only dump the module headers if the entire module is selected

    -o <filename>
        write to the specified file.

    -a <filename>
        like -outfile but append instead of overwrite
\end{lstlisting}

\section{echo -- turning echoing back of commands on and off}
\label{cmd:echo}
\begin{lstlisting}[numbers=left,frame=single]
    echo on

Print all commands to log before executing them.


    echo off

Do not print all commands to log before executing them. (default)
\end{lstlisting}

\section{ecp5\_ffinit -- ECP5: handle FF init values}
\label{cmd:ecp5_ffinit}
\begin{lstlisting}[numbers=left,frame=single]
    ecp5_ffinit [options] [selection]

Remove init values for FF output signals when equal to reset value.
If reset is not used, set the reset value to the init value, otherwise
unmap out the reset (if not an async reset).
\end{lstlisting}

\section{ecp5\_gsr -- ECP5: handle GSR}
\label{cmd:ecp5_gsr}
\begin{lstlisting}[numbers=left,frame=single]
    ecp5_gsr [options] [selection]

Trim active low async resets connected to GSR and resolve GSR parameter,
if a GSR or SGSR primitive is used in the design.

If any cell has the GSR parameter set to "AUTO", this will be resolved
to "ENABLED" if a GSR primitive is present and the (* nogsr *) attribute
is not set, otherwise it will be resolved to "DISABLED".
\end{lstlisting}

\section{edgetypes -- list all types of edges in selection}
\label{cmd:edgetypes}
\begin{lstlisting}[numbers=left,frame=single]
    edgetypes [options] [selection]

This command lists all unique types of 'edges' found in the selection. An 'edge'
is a 4-tuple of source and sink cell type and port name.
\end{lstlisting}

\section{efinix\_fixcarry -- Efinix: fix carry chain}
\label{cmd:efinix_fixcarry}
\begin{lstlisting}[numbers=left,frame=single]
    efinix_fixcarry [options] [selection]

Add Efinix adders to fix carry chain if needed.
\end{lstlisting}

\section{efinix\_gbuf -- Efinix: insert global clock buffers}
\label{cmd:efinix_gbuf}
\begin{lstlisting}[numbers=left,frame=single]
    efinix_gbuf [options] [selection]

Add Efinix global clock buffers to top module as needed.
\end{lstlisting}

\section{equiv\_add -- add a \$equiv cell}
\label{cmd:equiv_add}
\begin{lstlisting}[numbers=left,frame=single]
    equiv_add [-try] gold_sig gate_sig

This command adds an $equiv cell for the specified signals.


    equiv_add [-try] -cell gold_cell gate_cell

This command adds $equiv cells for the ports of the specified cells.
\end{lstlisting}

\section{equiv\_induct -- proving \$equiv cells using temporal induction}
\label{cmd:equiv_induct}
\begin{lstlisting}[numbers=left,frame=single]
    equiv_induct [options] [selection]

Uses a version of temporal induction to prove $equiv cells.

Only selected $equiv cells are proven and only selected cells are used to
perform the proof.

    -undef
        enable modelling of undef states

    -seq <N>
        the max. number of time steps to be considered (default = 4)

This command is very effective in proving complex sequential circuits, when
the internal state of the circuit quickly propagates to $equiv cells.

However, this command uses a weak definition of 'equivalence': This command
proves that the two circuits will not diverge after they produce equal
outputs (observable points via $equiv) for at least <N> cycles (the <N>
specified via -seq).

Combined with simulation this is very powerful because simulation can give
you confidence that the circuits start out synced for at least <N> cycles
after reset.
\end{lstlisting}

\section{equiv\_make -- prepare a circuit for equivalence checking}
\label{cmd:equiv_make}
\begin{lstlisting}[numbers=left,frame=single]
    equiv_make [options] gold_module gate_module equiv_module

This creates a module annotated with $equiv cells from two presumably
equivalent modules. Use commands such as 'equiv_simple' and 'equiv_status'
to work with the created equivalent checking module.

    -inames
        Also match cells and wires with $... names.

    -blacklist <file>
        Do not match cells or signals that match the names in the file.

    -encfile <file>
        Match FSM encodings using the description from the file.
        See 'help fsm_recode' for details.

Note: The circuit created by this command is not a miter (with something like
a trigger output), but instead uses $equiv cells to encode the equivalence
checking problem. Use 'miter -equiv' if you want to create a miter circuit.
\end{lstlisting}

\section{equiv\_mark -- mark equivalence checking regions}
\label{cmd:equiv_mark}
\begin{lstlisting}[numbers=left,frame=single]
    equiv_mark [options] [selection]

This command marks the regions in an equivalence checking module. Region 0 is
the proven part of the circuit. Regions with higher numbers are connected
unproven subcricuits. The integer attribute 'equiv_region' is set on all
wires and cells.
\end{lstlisting}

\section{equiv\_miter -- extract miter from equiv circuit}
\label{cmd:equiv_miter}
\begin{lstlisting}[numbers=left,frame=single]
    equiv_miter [options] miter_module [selection]

This creates a miter module for further analysis of the selected $equiv cells.

    -trigger
        Create a trigger output

    -cmp
        Create cmp_* outputs for individual unproven $equiv cells

    -assert
        Create a $assert cell for each unproven $equiv cell

    -undef
        Create compare logic that handles undefs correctly
\end{lstlisting}

\section{equiv\_opt -- prove equivalence for optimized circuit}
\label{cmd:equiv_opt}
\begin{lstlisting}[numbers=left,frame=single]
    equiv_opt [options] [command]

This command uses temporal induction to check circuit equivalence before and
after an optimization pass.

    -run <from_label>:<to_label>
        only run the commands between the labels (see below). an empty
        from label is synonymous to the start of the command list, and empty to
        label is synonymous to the end of the command list.

    -map <filename>
        expand the modules in this file before proving equivalence. this is
        useful for handling architecture-specific primitives.

    -blacklist <file>
        Do not match cells or signals that match the names in the file
        (passed to equiv_make).

    -assert
        produce an error if the circuits are not equivalent.

    -multiclock
        run clk2fflogic before equivalence checking.

    -async2sync
        run async2sync before equivalence checking.

    -undef
        enable modelling of undef states during equiv_induct.

The following commands are executed by this verification command:

    run_pass:
        hierarchy -auto-top
        design -save preopt
        [command]
        design -stash postopt

    prepare:
        design -copy-from preopt  -as gold A:top
        design -copy-from postopt -as gate A:top

    techmap:    (only with -map)
        techmap -wb -D EQUIV -autoproc -map <filename> ...

    prove:
        clk2fflogic    (only with -multiclock)
        async2sync     (only with -async2sync)
        equiv_make -blacklist <filename> ... gold gate equiv
        equiv_induct [-undef] equiv
        equiv_status [-assert] equiv

    restore:
        design -load preopt
\end{lstlisting}

\section{equiv\_purge -- purge equivalence checking module}
\label{cmd:equiv_purge}
\begin{lstlisting}[numbers=left,frame=single]
    equiv_purge [options] [selection]

This command removes the proven part of an equivalence checking module, leaving
only the unproven segments in the design. This will also remove and add module
ports as needed.
\end{lstlisting}

\section{equiv\_remove -- remove \$equiv cells}
\label{cmd:equiv_remove}
\begin{lstlisting}[numbers=left,frame=single]
    equiv_remove [options] [selection]

This command removes the selected $equiv cells. If neither -gold nor -gate is
used then only proven cells are removed.

    -gold
        keep gold circuit

    -gate
        keep gate circuit
\end{lstlisting}

\section{equiv\_simple -- try proving simple \$equiv instances}
\label{cmd:equiv_simple}
\begin{lstlisting}[numbers=left,frame=single]
    equiv_simple [options] [selection]

This command tries to prove $equiv cells using a simple direct SAT approach.

    -v
        verbose output

    -undef
        enable modelling of undef states

    -short
        create shorter input cones that stop at shared nodes. This yields
        simpler SAT problems but sometimes fails to prove equivalence.

    -nogroup
        disabling grouping of $equiv cells by output wire

    -seq <N>
        the max. number of time steps to be considered (default = 1)
\end{lstlisting}

\section{equiv\_status -- print status of equivalent checking module}
\label{cmd:equiv_status}
\begin{lstlisting}[numbers=left,frame=single]
    equiv_status [options] [selection]

This command prints status information for all selected $equiv cells.

    -assert
        produce an error if any unproven $equiv cell is found
\end{lstlisting}

\section{equiv\_struct -- structural equivalence checking}
\label{cmd:equiv_struct}
\begin{lstlisting}[numbers=left,frame=single]
    equiv_struct [options] [selection]

This command adds additional $equiv cells based on the assumption that the
gold and gate circuit are structurally equivalent. Note that this can introduce
bad $equiv cells in cases where the netlists are not structurally equivalent,
for example when analyzing circuits with cells with commutative inputs. This
command will also de-duplicate gates.

    -fwd
        by default this command performans forward sweeps until nothing can
        be merged by forwards sweeps, then backward sweeps until forward
        sweeps are effective again. with this option set only forward sweeps
        are performed.

    -fwonly <cell_type>
        add the specified cell type to the list of cell types that are only
        merged in forward sweeps and never in backward sweeps. $equiv is in
        this list automatically.

    -icells
        by default, the internal RTL and gate cell types are ignored. add
        this option to also process those cell types with this command.

    -maxiter <N>
        maximum number of iterations to run before aborting
\end{lstlisting}

\section{eval -- evaluate the circuit given an input}
\label{cmd:eval}
\begin{lstlisting}[numbers=left,frame=single]
    eval [options] [selection]

This command evaluates the value of a signal given the value of all required
inputs.

    -set <signal> <value>
        set the specified signal to the specified value.

    -set-undef
        set all unspecified source signals to undef (x)

    -table <signal>
        create a truth table using the specified input signals

    -show <signal>
        show the value for the specified signal. if no -show option is passed
        then all output ports of the current module are used.
\end{lstlisting}

\section{exec -- execute commands in the operating system shell}
\label{cmd:exec}
\begin{lstlisting}[numbers=left,frame=single]
    exec [options] -- [command]

Execute a command in the operating system shell.  All supplied arguments are
concatenated and passed as a command to popen(3).  Whitespace is not guaranteed
to be preserved, even if quoted.  stdin and stderr are not connected, while stdout is
logged unless the "-q" option is specified.


    -q
        Suppress stdout and stderr from subprocess

    -expect-return <int>
        Generate an error if popen() does not return specified value.
        May only be specified once; the final specified value is controlling
        if specified multiple times.

    -expect-stdout <regex>
        Generate an error if the specified regex does not match any line
        in subprocess's stdout.  May be specified multiple times.

    -not-expect-stdout <regex>
        Generate an error if the specified regex matches any line
        in subprocess's stdout.  May be specified multiple times.


    Example: exec -q -expect-return 0 -- echo "bananapie" | grep "nana"
\end{lstlisting}

\section{expose -- convert internal signals to module ports}
\label{cmd:expose}
\begin{lstlisting}[numbers=left,frame=single]
    expose [options] [selection]

This command exposes all selected internal signals of a module as additional
outputs.

    -dff
        only consider wires that are directly driven by register cell.

    -cut
        when exposing a wire, create an input/output pair and cut the internal
        signal path at that wire.

    -input
        when exposing a wire, create an input port and disconnect the internal
        driver.

    -shared
        only expose those signals that are shared among the selected modules.
        this is useful for preparing modules for equivalence checking.

    -evert
        also turn connections to instances of other modules to additional
        inputs and outputs and remove the module instances.

    -evert-dff
        turn flip-flops to sets of inputs and outputs.

    -sep <separator>
        when creating new wire/port names, the original object name is suffixed
        with this separator (default: '.') and the port name or a type
        designator for the exposed signal.
\end{lstlisting}

\section{extract -- find subcircuits and replace them with cells}
\label{cmd:extract}
\begin{lstlisting}[numbers=left,frame=single]
    extract -map <map_file> [options] [selection]
    extract -mine <out_file> [options] [selection]

This pass looks for subcircuits that are isomorphic to any of the modules
in the given map file and replaces them with instances of this modules. The
map file can be a Verilog source file (*.v) or an ilang file (*.il).

    -map <map_file>
        use the modules in this file as reference. This option can be used
        multiple times.

    -map %<design-name>
        use the modules in this in-memory design as reference. This option can
        be used multiple times.

    -verbose
        print debug output while analyzing

    -constports
        also find instances with constant drivers. this may be much
        slower than the normal operation.

    -nodefaultswaps
        normally builtin port swapping rules for internal cells are used per
        default. This turns that off, so e.g. 'a^b' does not match 'b^a'
        when this option is used.

    -compat <needle_type> <haystack_type>
        Per default, the cells in the map file (needle) must have the
        type as the cells in the active design (haystack). This option
        can be used to register additional pairs of types that should
        match. This option can be used multiple times.

    -swap <needle_type> <port1>,<port2>[,...]
        Register a set of swappable ports for a needle cell type.
        This option can be used multiple times.

    -perm <needle_type> <port1>,<port2>[,...] <portA>,<portB>[,...]
        Register a valid permutation of swappable ports for a needle
        cell type. This option can be used multiple times.

    -cell_attr <attribute_name>
        Attributes on cells with the given name must match.

    -wire_attr <attribute_name>
        Attributes on wires with the given name must match.

    -ignore_parameters
        Do not use parameters when matching cells.

    -ignore_param <cell_type> <parameter_name>
        Do not use this parameter when matching cells.

This pass does not operate on modules with unprocessed processes in it.
(I.e. the 'proc' pass should be used first to convert processes to netlists.)

This pass can also be used for mining for frequent subcircuits. In this mode
the following options are to be used instead of the -map option.

    -mine <out_file>
        mine for frequent subcircuits and write them to the given ilang file

    -mine_cells_span <min> <max>
        only mine for subcircuits with the specified number of cells
        default value: 3 5

    -mine_min_freq <num>
        only mine for subcircuits with at least the specified number of matches
        default value: 10

    -mine_limit_matches_per_module <num>
        when calculating the number of matches for a subcircuit, don't count
        more than the specified number of matches per module

    -mine_max_fanout <num>
        don't consider internal signals with more than <num> connections

The modules in the map file may have the attribute 'extract_order' set to an
integer value. Then this value is used to determine the order in which the pass
tries to map the modules to the design (ascending, default value is 0).

See 'help techmap' for a pass that does the opposite thing.
\end{lstlisting}

\section{extract\_counter -- Extract GreenPak4 counter cells}
\label{cmd:extract_counter}
\begin{lstlisting}[numbers=left,frame=single]
    extract_counter [options] [selection]

This pass converts non-resettable or async resettable down counters to
counter cells. Use a target-specific 'techmap' map file to convert those cells
to the actual target cells.

    -maxwidth N
        Only extract counters up to N bits wide (default 64)

    -minwidth N
        Only extract counters at least N bits wide (default 2)

    -allow_arst yes|no
        Allow counters to have async reset (default yes)

    -dir up|down|both
        Look for up-counters, down-counters, or both (default down)

    -pout X,Y,...
        Only allow parallel output from the counter to the listed cell types
        (if not specified, parallel outputs are not restricted)
\end{lstlisting}

\section{extract\_fa -- find and extract full/half adders}
\label{cmd:extract_fa}
\begin{lstlisting}[numbers=left,frame=single]
    extract_fa [options] [selection]

This pass extracts full/half adders from a gate-level design.

    -fa, -ha
        Enable cell types (fa=full adder, ha=half adder)
        All types are enabled if none of this options is used

    -d <int>
        Set maximum depth for extracted logic cones (default=20)

    -b <int>
        Set maximum breadth for extracted logic cones (default=6)

    -v
        Verbose output
\end{lstlisting}

\section{extract\_reduce -- converts gate chains into \$reduce\_* cells}
\label{cmd:extract_reduce}
\begin{lstlisting}[numbers=left,frame=single]
    extract_reduce [options] [selection]

converts gate chains into $reduce_* cells

This command finds chains of $_AND_, $_OR_, and $_XOR_ cells and replaces them
with their corresponding $reduce_* cells. Because this command only operates on
these cell types, it is recommended to map the design to only these cell types
using the `abc -g` command. Note that, in some cases, it may be more effective
to map the design to only $_AND_ cells, run extract_reduce, map the remaining
parts of the design to AND/OR/XOR cells, and run extract_reduce a second time.

    -allow-off-chain
        Allows matching of cells that have loads outside the chain. These cells
        will be replicated and folded into the $reduce_* cell, but the original
        cell will remain, driving its original loads.
\end{lstlisting}

\section{extractinv -- extract explicit inverter cells for invertible cell pins}
\label{cmd:extractinv}
\begin{lstlisting}[numbers=left,frame=single]
    extractinv [options] [selection]

Searches the design for all cells with invertible pins controlled by a cell
parameter (eg. IS_CLK_INVERTED on many Xilinx cells) and removes the parameter.
If the parameter was set to 1, inserts an explicit inverter cell in front of
the pin instead.  Normally used for output to ISE, which does not support the
inversion parameters.

To mark a cell port as invertible, use (* invertible_pin = "param_name" *)
on the wire in the blackbox module.  The parameter value should have
the same width as the port, and will be effectively XORed with it.

    -inv <celltype> <portname_out>:<portname_in>
        Specifies the cell type to use for the inverters and its port names.
        This option is required.
\end{lstlisting}

\section{flatten -- flatten design}
\label{cmd:flatten}
\begin{lstlisting}[numbers=left,frame=single]
    flatten [options] [selection]

This pass flattens the design by replacing cells by their implementation. This
pass is very similar to the 'techmap' pass. The only difference is that this
pass is using the current design as mapping library.

Cells and/or modules with the 'keep_hierarchy' attribute set will not be
flattened by this command.

    -wb
        Ignore the 'whitebox' attribute on cell implementations.
\end{lstlisting}

\section{flowmap -- pack LUTs with FlowMap}
\label{cmd:flowmap}
\begin{lstlisting}[numbers=left,frame=single]
    flowmap [options] [selection]

This pass uses the FlowMap technology mapping algorithm to pack logic gates
into k-LUTs with optimal depth. It allows mapping any circuit elements that can
be evaluated with the `eval` pass, including cells with multiple output ports
and multi-bit input and output ports.

    -maxlut k
        perform technology mapping for a k-LUT architecture. if not specified,
        defaults to 3.

    -minlut n
        only produce n-input or larger LUTs. if not specified, defaults to 1.

    -cells <cell>[,<cell>,...]
        map specified cells. if not specified, maps $_NOT_, $_AND_, $_OR_,
        $_XOR_ and $_MUX_, which are the outputs of the `simplemap` pass.

    -relax
        perform depth relaxation and area minimization.

    -r-alpha n, -r-beta n, -r-gamma n
        parameters of depth relaxation heuristic potential function.
        if not specified, alpha=8, beta=2, gamma=1.

    -optarea n
        optimize for area by trading off at most n logic levels for fewer LUTs.
        n may be zero, to optimize for area without increasing depth.
        implies -relax.

    -debug
        dump intermediate graphs.

    -debug-relax
        explain decisions performed during depth relaxation.
\end{lstlisting}

\section{fmcombine -- combine two instances of a cell into one}
\label{cmd:fmcombine}
\begin{lstlisting}[numbers=left,frame=single]
    fmcombine [options] module_name gold_cell gate_cell

This pass takes two cells, which are instances of the same module, and replaces
them with one instance of a special 'combined' module, that effectively
contains two copies of the original module, plus some formal properties.

This is useful for formal test benches that check what differences in behavior
a slight difference in input causes in a module.

    -initeq
        Insert assumptions that initially all FFs in both circuits have the
        same initial values.

    -anyeq
        Do not duplicate $anyseq/$anyconst cells.

    -fwd
        Insert forward hint assumptions into the combined module.

    -bwd
        Insert backward hint assumptions into the combined module.
        (Backward hints are logically equivalend to fordward hits, but
        some solvers are faster with bwd hints, or even both -bwd and -fwd.)

    -nop
        Don't insert hint assumptions into the combined module.
        (This should not provide any speedup over the original design, but
        strangely sometimes it does.)

If none of -fwd, -bwd, and -nop is given, then -fwd is used as default.
\end{lstlisting}

\section{fminit -- set init values/sequences for formal}
\label{cmd:fminit}
\begin{lstlisting}[numbers=left,frame=single]
    fminit [options] <selection>

This pass creates init constraints (for example for reset sequences) in a formal
model.

    -seq <signal> <sequence>
        Set sequence using comma-separated list of values, use 'z for
        unconstrained bits. The last value is used for the remainder of the
        trace.

    -set <signal> <value>
        Add constant value constraint

    -posedge <signal>
    -negedge <signal>
        Set clock for init sequences
\end{lstlisting}

\section{freduce -- perform functional reduction}
\label{cmd:freduce}
\begin{lstlisting}[numbers=left,frame=single]
    freduce [options] [selection]

This pass performs functional reduction in the circuit. I.e. if two nodes are
equivalent, they are merged to one node and one of the redundant drivers is
disconnected. A subsequent call to 'clean' will remove the redundant drivers.

    -v, -vv
        enable verbose or very verbose output

    -inv
        enable explicit handling of inverted signals

    -stop <n>
        stop after <n> reduction operations. this is mostly used for
        debugging the freduce command itself.

    -dump <prefix>
        dump the design to <prefix>_<module>_<num>.il after each reduction
        operation. this is mostly used for debugging the freduce command.

This pass is undef-aware, i.e. it considers don't-care values for detecting
equivalent nodes.

All selected wires are considered for rewiring. The selected cells cover the
circuit that is analyzed.
\end{lstlisting}

\section{fsm -- extract and optimize finite state machines}
\label{cmd:fsm}
\begin{lstlisting}[numbers=left,frame=single]
    fsm [options] [selection]

This pass calls all the other fsm_* passes in a useful order. This performs
FSM extraction and optimization. It also calls opt_clean as needed:

    fsm_detect          unless got option -nodetect
    fsm_extract

    fsm_opt
    opt_clean
    fsm_opt

    fsm_expand          if got option -expand
    opt_clean           if got option -expand
    fsm_opt             if got option -expand

    fsm_recode          unless got option -norecode

    fsm_info

    fsm_export          if got option -export
    fsm_map             unless got option -nomap

Options:

    -expand, -norecode, -export, -nomap
        enable or disable passes as indicated above

    -fullexpand
        call expand with -full option

    -encoding type
    -fm_set_fsm_file file
    -encfile file
        passed through to fsm_recode pass
\end{lstlisting}

\section{fsm\_detect -- finding FSMs in design}
\label{cmd:fsm_detect}
\begin{lstlisting}[numbers=left,frame=single]
    fsm_detect [selection]

This pass detects finite state machines by identifying the state signal.
The state signal is then marked by setting the attribute 'fsm_encoding'
on the state signal to "auto".

Existing 'fsm_encoding' attributes are not changed by this pass.

Signals can be protected from being detected by this pass by setting the
'fsm_encoding' attribute to "none".
\end{lstlisting}

\section{fsm\_expand -- expand FSM cells by merging logic into it}
\label{cmd:fsm_expand}
\begin{lstlisting}[numbers=left,frame=single]
    fsm_expand [-full] [selection]

The fsm_extract pass is conservative about the cells that belong to a finite
state machine. This pass can be used to merge additional auxiliary gates into
the finite state machine.

By default, fsm_expand is still a bit conservative regarding merging larger
word-wide cells. Call with -full to consider all cells for merging.
\end{lstlisting}

\section{fsm\_export -- exporting FSMs to KISS2 files}
\label{cmd:fsm_export}
\begin{lstlisting}[numbers=left,frame=single]
    fsm_export [-noauto] [-o filename] [-origenc] [selection]

This pass creates a KISS2 file for every selected FSM. For FSMs with the
'fsm_export' attribute set, the attribute value is used as filename, otherwise
the module and cell name is used as filename. If the parameter '-o' is given,
the first exported FSM is written to the specified filename. This overwrites
the setting as specified with the 'fsm_export' attribute. All other FSMs are
exported to the default name as mentioned above.

    -noauto
        only export FSMs that have the 'fsm_export' attribute set

    -o filename
        filename of the first exported FSM

    -origenc
        use binary state encoding as state names instead of s0, s1, ...
\end{lstlisting}

\section{fsm\_extract -- extracting FSMs in design}
\label{cmd:fsm_extract}
\begin{lstlisting}[numbers=left,frame=single]
    fsm_extract [selection]

This pass operates on all signals marked as FSM state signals using the
'fsm_encoding' attribute. It consumes the logic that creates the state signal
and uses the state signal to generate control signal and replaces it with an
FSM cell.

The generated FSM cell still generates the original state signal with its
original encoding. The 'fsm_opt' pass can be used in combination with the
'opt_clean' pass to eliminate this signal.
\end{lstlisting}

\section{fsm\_info -- print information on finite state machines}
\label{cmd:fsm_info}
\begin{lstlisting}[numbers=left,frame=single]
    fsm_info [selection]

This pass dumps all internal information on FSM cells. It can be useful for
analyzing the synthesis process and is called automatically by the 'fsm'
pass so that this information is included in the synthesis log file.
\end{lstlisting}

\section{fsm\_map -- mapping FSMs to basic logic}
\label{cmd:fsm_map}
\begin{lstlisting}[numbers=left,frame=single]
    fsm_map [selection]

This pass translates FSM cells to flip-flops and logic.
\end{lstlisting}

\section{fsm\_opt -- optimize finite state machines}
\label{cmd:fsm_opt}
\begin{lstlisting}[numbers=left,frame=single]
    fsm_opt [selection]

This pass optimizes FSM cells. It detects which output signals are actually
not used and removes them from the FSM. This pass is usually used in
combination with the 'opt_clean' pass (see also 'help fsm').
\end{lstlisting}

\section{fsm\_recode -- recoding finite state machines}
\label{cmd:fsm_recode}
\begin{lstlisting}[numbers=left,frame=single]
    fsm_recode [options] [selection]

This pass reassign the state encodings for FSM cells. At the moment only
one-hot encoding and binary encoding is supported.
    -encoding <type>
        specify the encoding scheme used for FSMs without the
        'fsm_encoding' attribute or with the attribute set to `auto'.

    -fm_set_fsm_file <file>
        generate a file containing the mapping from old to new FSM encoding
        in form of Synopsys Formality set_fsm_* commands.

    -encfile <file>
        write the mappings from old to new FSM encoding to a file in the
        following format:

            .fsm <module_name> <state_signal>
            .map <old_bitpattern> <new_bitpattern>
\end{lstlisting}

\section{greenpak4\_dffinv -- merge greenpak4 inverters and DFF/latches}
\label{cmd:greenpak4_dffinv}
\begin{lstlisting}[numbers=left,frame=single]
    greenpak4_dffinv [options] [selection]

Merge GP_INV cells with GP_DFF* and GP_DLATCH* cells.
\end{lstlisting}

\section{help -- display help messages}
\label{cmd:help}
\begin{lstlisting}[numbers=left,frame=single]
    help  ................  list all commands
    help <command>  ......  print help message for given command
    help -all  ...........  print complete command reference

    help -cells ..........  list all cell types
    help <celltype>  .....  print help message for given cell type
    help <celltype>+  ....  print verilog code for given cell type
\end{lstlisting}

\section{hierarchy -- check, expand and clean up design hierarchy}
\label{cmd:hierarchy}
\begin{lstlisting}[numbers=left,frame=single]
    hierarchy [-check] [-top <module>]
    hierarchy -generate <cell-types> <port-decls>

In parametric designs, a module might exists in several variations with
different parameter values. This pass looks at all modules in the current
design an re-runs the language frontends for the parametric modules as
needed. It also resolves assignments to wired logic data types (wand/wor),
resolves positional module parameters, unroll array instances, and more.

    -check
        also check the design hierarchy. this generates an error when
        an unknown module is used as cell type.

    -simcheck
        like -check, but also throw an error if blackbox modules are
        instantiated, and throw an error if the design has no top module.

    -purge_lib
        by default the hierarchy command will not remove library (blackbox)
        modules. use this option to also remove unused blackbox modules.

    -libdir <directory>
        search for files named <module_name>.v in the specified directory
        for unknown modules and automatically run read_verilog for each
        unknown module.

    -keep_positionals
        per default this pass also converts positional arguments in cells
        to arguments using port names. This option disables this behavior.

    -keep_portwidths
        per default this pass adjusts the port width on cells that are
        module instances when the width does not match the module port. This
        option disables this behavior.

    -nodefaults
        do not resolve input port default values

    -nokeep_asserts
        per default this pass sets the "keep" attribute on all modules
        that directly or indirectly contain one or more formal properties.
        This option disables this behavior.

    -top <module>
        use the specified top module to build the design hierarchy. Modules
        outside this tree (unused modules) are removed.

        when the -top option is used, the 'top' attribute will be set on the
        specified top module. otherwise a module with the 'top' attribute set
        will implicitly be used as top module, if such a module exists.

    -auto-top
        automatically determine the top of the design hierarchy and mark it.

    -chparam name value 
       elaborate the top module using this parameter value. Modules on which
       this parameter does not exist may cause a warning message to be output.
       This option can be specified multiple times to override multiple
       parameters. String values must be passed in double quotes (").

In -generate mode this pass generates blackbox modules for the given cell
types (wildcards supported). For this the design is searched for cells that
match the given types and then the given port declarations are used to
determine the direction of the ports. The syntax for a port declaration is:

    {i|o|io}[@<num>]:<portname>

Input ports are specified with the 'i' prefix, output ports with the 'o'
prefix and inout ports with the 'io' prefix. The optional <num> specifies
the position of the port in the parameter list (needed when instantiated
using positional arguments). When <num> is not specified, the <portname> can
also contain wildcard characters.

This pass ignores the current selection and always operates on all modules
in the current design.
\end{lstlisting}

\section{hilomap -- technology mapping of constant hi- and/or lo-drivers}
\label{cmd:hilomap}
\begin{lstlisting}[numbers=left,frame=single]
    hilomap [options] [selection]

Map constants to 'tielo' and 'tiehi' driver cells.

    -hicell <celltype> <portname>
        Replace constant hi bits with this cell.

    -locell <celltype> <portname>
        Replace constant lo bits with this cell.

    -singleton
        Create only one hi/lo cell and connect all constant bits
        to that cell. Per default a separate cell is created for
        each constant bit.
\end{lstlisting}

\section{history -- show last interactive commands}
\label{cmd:history}
\begin{lstlisting}[numbers=left,frame=single]
    history

This command prints all commands in the shell history buffer. This are
all commands executed in an interactive session, but not the commands
from executed scripts.
\end{lstlisting}

\section{ice40\_braminit -- iCE40: perform SB\_RAM40\_4K initialization from file}
\label{cmd:ice40_braminit}
\begin{lstlisting}[numbers=left,frame=single]
    ice40_braminit

This command processes all SB_RAM40_4K blocks with a non-empty INIT_FILE
parameter and converts it into the required INIT_x attributes
\end{lstlisting}

\section{ice40\_dsp -- iCE40: map multipliers}
\label{cmd:ice40_dsp}
\begin{lstlisting}[numbers=left,frame=single]
    ice40_dsp [options] [selection]

Map multipliers ($mul/SB_MAC16) and multiply-accumulate ($mul/SB_MAC16 + $add)
cells into iCE40 DSP resources.
Currently, only the 16x16 multiply mode is supported and not the 2 x 8x8 mode.

Pack input registers (A, B, {C,D}; with optional hold), pipeline registers
({F,J,K,G}, H), output registers (O -- full 32-bits or lower 16-bits only; with
optional hold), and post-adder into into the SB_MAC16 resource.

Multiply-accumulate operations using the post-adder with feedback on the {C,D}
input will be folded into the DSP. In this scenario only, resetting the
the accumulator to an arbitrary value can be inferred to use the {C,D} input.
\end{lstlisting}

\section{ice40\_ffinit -- iCE40: handle FF init values}
\label{cmd:ice40_ffinit}
\begin{lstlisting}[numbers=left,frame=single]
    ice40_ffinit [options] [selection]

Remove zero init values for FF output signals. Add inverters to implement
nonzero init values.
\end{lstlisting}

\section{ice40\_ffssr -- iCE40: merge synchronous set/reset into FF cells}
\label{cmd:ice40_ffssr}
\begin{lstlisting}[numbers=left,frame=single]
    ice40_ffssr [options] [selection]

Merge synchronous set/reset $_MUX_ cells into iCE40 FFs.
\end{lstlisting}

\section{ice40\_opt -- iCE40: perform simple optimizations}
\label{cmd:ice40_opt}
\begin{lstlisting}[numbers=left,frame=single]
    ice40_opt [options] [selection]

This command executes the following script:

    do
        <ice40 specific optimizations>
        opt_expr -mux_undef -undriven [-full]
        opt_merge
        opt_rmdff
        opt_clean
    while <changed design>
\end{lstlisting}

\section{ice40\_wrapcarry -- iCE40: wrap carries}
\label{cmd:ice40_wrapcarry}
\begin{lstlisting}[numbers=left,frame=single]
    ice40_wrapcarry [selection]

Wrap manually instantiated SB_CARRY cells, along with their associated SB_LUT4s,
into an internal $__ICE40_CARRY_WRAPPER cell for preservation across technology
mapping.

Attributes on both cells will have their names prefixed with 'SB_CARRY.' or
'SB_LUT4.' and attached to the wrapping cell.
A (* keep *) attribute on either cell will be logically OR-ed together.

    -unwrap
        unwrap $__ICE40_CARRY_WRAPPER cells back into SB_CARRYs and SB_LUT4s,
        including restoring their attributes.
\end{lstlisting}

\section{insbuf -- insert buffer cells for connected wires}
\label{cmd:insbuf}
\begin{lstlisting}[numbers=left,frame=single]
    insbuf [options] [selection]

Insert buffer cells into the design for directly connected wires.

    -buf <celltype> <in-portname> <out-portname>
        Use the given cell type instead of $_BUF_. (Notice that the next
        call to "clean" will remove all $_BUF_ in the design.)
\end{lstlisting}

\section{iopadmap -- technology mapping of i/o pads (or buffers)}
\label{cmd:iopadmap}
\begin{lstlisting}[numbers=left,frame=single]
    iopadmap [options] [selection]

Map module inputs/outputs to PAD cells from a library. This pass
can only map to very simple PAD cells. Use 'techmap' to further map
the resulting cells to more sophisticated PAD cells.

    -inpad <celltype> <portname>[:<portname>]
        Map module input ports to the given cell type with the
        given output port name. if a 2nd portname is given, the
        signal is passed through the pad call, using the 2nd
        portname as the port facing the module port.

    -outpad <celltype> <portname>[:<portname>]
    -inoutpad <celltype> <portname>[:<portname>]
        Similar to -inpad, but for output and inout ports.

    -toutpad <celltype> <portname>:<portname>[:<portname>]
        Merges $_TBUF_ cells into the output pad cell. This takes precedence
        over the other -outpad cell. The first portname is the enable input
        of the tristate driver.

    -tinoutpad <celltype> <portname>:<portname>:<portname>[:<portname>]
        Merges $_TBUF_ cells into the inout pad cell. This takes precedence
        over the other -inoutpad cell. The first portname is the enable input
        of the tristate driver and the 2nd portname is the internal output
        buffering the external signal.

    -ignore <celltype> <portname>[:<portname>]*
        Skips mapping inputs/outputs that are already connected to given
        ports of the given cell.  Can be used multiple times.  This is in
        addition to the cells specified as mapping targets.

    -widthparam <param_name>
        Use the specified parameter name to set the port width.

    -nameparam <param_name>
        Use the specified parameter to set the port name.

    -bits
        create individual bit-wide buffers even for ports that
        are wider. (the default behavior is to create word-wide
        buffers using -widthparam to set the word size on the cell.)

Tristate PADS (-toutpad, -tinoutpad) always operate in -bits mode.
\end{lstlisting}

\section{json -- write design in JSON format}
\label{cmd:json}
\begin{lstlisting}[numbers=left,frame=single]
    json [options] [selection]

Write a JSON netlist of all selected objects.

    -o <filename>
        write to the specified file.

    -aig
        also include AIG models for the different gate types

    -compat-int
        emit 32-bit or smaller fully-defined parameter values directly
        as JSON numbers (for compatibility with old parsers)

See 'help write_json' for a description of the JSON format used.
\end{lstlisting}

\section{log -- print text and log files}
\label{cmd:log}
\begin{lstlisting}[numbers=left,frame=single]
    log string

Print the given string to the screen and/or the log file. This is useful for TCL
scripts, because the TCL command "puts" only goes to stdout but not to
logfiles.

    -stdout
        Print the output to stdout too. This is useful when all Yosys is executed
        with a script and the -q (quiet operation) argument to notify the user.

    -stderr
        Print the output to stderr too.

    -nolog
        Don't use the internal log() command. Use either -stdout or -stderr,
        otherwise no output will be generated at all.

    -n
        do not append a newline
\end{lstlisting}

\section{logger -- set logger properties}
\label{cmd:logger}
\begin{lstlisting}[numbers=left,frame=single]
    logger [options]

This command sets global logger properties, also available using command line
options.

    -[no]time
        enable/disable display of timestamp in log output.

    -[no]stderr
        enable/disable logging errors to stderr.

    -warn regex
        print a warning for all log messages matching the regex.

    -nowarn regex
        if a warning message matches the regex, it is printed as regular
        message instead.

    -werror regex
        if a warning message matches the regex, it is printed as error
        message instead and the tool terminates with a nonzero return code.

    -[no]debug
        globally enable/disable debug log messages.

    -experimental <feature>
        do not print warnings for the specified experimental feature

    -expect <type> <regex> <expected_count>
        expect log,warning or error to appear. In case of error return code is 0.

    -expect-no-warnings
        gives error in case there is at least one warning that is not expected.
\end{lstlisting}

\section{ls -- list modules or objects in modules}
\label{cmd:ls}
\begin{lstlisting}[numbers=left,frame=single]
    ls [selection]

When no active module is selected, this prints a list of modules.

When an active module is selected, this prints a list of objects in the module.
\end{lstlisting}

\section{ltp -- print longest topological path}
\label{cmd:ltp}
\begin{lstlisting}[numbers=left,frame=single]
    ltp [options] [selection]

This command prints the longest topological path in the design. (Only considers
paths within a single module, so the design must be flattened.)

    -noff
        automatically exclude FF cell types
\end{lstlisting}

\section{lut2mux -- convert \$lut to \$\_MUX\_}
\label{cmd:lut2mux}
\begin{lstlisting}[numbers=left,frame=single]
    lut2mux [options] [selection]

This pass converts $lut cells to $_MUX_ gates.
\end{lstlisting}

\section{maccmap -- mapping macc cells}
\label{cmd:maccmap}
\begin{lstlisting}[numbers=left,frame=single]
    maccmap [-unmap] [selection]

This pass maps $macc cells to yosys $fa and $alu cells. When the -unmap option
is used then the $macc cell is mapped to $add, $sub, etc. cells instead.
\end{lstlisting}

\section{memory -- translate memories to basic cells}
\label{cmd:memory}
\begin{lstlisting}[numbers=left,frame=single]
    memory [-nomap] [-nordff] [-memx] [-bram <bram_rules>] [selection]

This pass calls all the other memory_* passes in a useful order:

    opt_mem
    memory_dff [-nordff]                (-memx implies -nordff)
    opt_clean
    memory_share
    opt_clean
    memory_memx                         (when called with -memx)
    memory_collect
    memory_bram -rules <bram_rules>     (when called with -bram)
    memory_map                          (skipped if called with -nomap)

This converts memories to word-wide DFFs and address decoders
or multiport memory blocks if called with the -nomap option.
\end{lstlisting}

\section{memory\_bram -- map memories to block rams}
\label{cmd:memory_bram}
\begin{lstlisting}[numbers=left,frame=single]
    memory_bram -rules <rule_file> [selection]

This pass converts the multi-port $mem memory cells into block ram instances.
The given rules file describes the available resources and how they should be
used.

The rules file contains configuration options, a set of block ram description
and a sequence of match rules.

The option 'attr_icase' configures how attribute values are matched. The value 0
means case-sensitive, 1 means case-insensitive.

A block ram description looks like this:

    bram RAMB1024X32     # name of BRAM cell
      init 1             # set to '1' if BRAM can be initialized
      abits 10           # number of address bits
      dbits 32           # number of data bits
      groups 2           # number of port groups
      ports  1 1         # number of ports in each group
      wrmode 1 0         # set to '1' if this groups is write ports
      enable 4 1         # number of enable bits
      transp 0 2         # transparent (for read ports)
      clocks 1 2         # clock configuration
      clkpol 2 2         # clock polarity configuration
    endbram

For the option 'transp' the value 0 means non-transparent, 1 means transparent
and a value greater than 1 means configurable. All groups with the same
value greater than 1 share the same configuration bit.

For the option 'clocks' the value 0 means non-clocked, and a value greater
than 0 means clocked. All groups with the same value share the same clock
signal.

For the option 'clkpol' the value 0 means negative edge, 1 means positive edge
and a value greater than 1 means configurable. All groups with the same value
greater than 1 share the same configuration bit.

Using the same bram name in different bram blocks will create different variants
of the bram. Verilog configuration parameters for the bram are created as needed.

It is also possible to create variants by repeating statements in the bram block
and appending '@<label>' to the individual statements.

A match rule looks like this:

    match RAMB1024X32
      max waste 16384    # only use this bram if <= 16k ram bits are unused
      min efficiency 80  # only use this bram if efficiency is at least 80%
    endmatch

It is possible to match against the following values with min/max rules:

    words  ........  number of words in memory in design
    abits  ........  number of address bits on memory in design
    dbits  ........  number of data bits on memory in design
    wports  .......  number of write ports on memory in design
    rports  .......  number of read ports on memory in design
    ports  ........  number of ports on memory in design
    bits  .........  number of bits in memory in design
    dups ..........  number of duplications for more read ports

    awaste  .......  number of unused address slots for this match
    dwaste  .......  number of unused data bits for this match
    bwaste  .......  number of unused bram bits for this match
    waste  ........  total number of unused bram bits (bwaste*dups)
    efficiency  ...  total percentage of used and non-duplicated bits

    acells  .......  number of cells in 'address-direction'
    dcells  .......  number of cells in 'data-direction'
    cells  ........  total number of cells (acells*dcells*dups)

A match containing the command 'attribute' followed by a list of space
separated 'name[=string_value]' values requires that the memory contains any
one of the given attribute name and string values (where specified), or name
and integer 1 value (if no string_value given, since Verilog will interpret
'(* attr *)' as '(* attr=1 *)').
A name prefixed with '!' indicates that the attribute must not exist.

The interface for the created bram instances is derived from the bram
description. Use 'techmap' to convert the created bram instances into
instances of the actual bram cells of your target architecture.

A match containing the command 'or_next_if_better' is only used if it
has a higher efficiency than the next match (and the one after that if
the next also has 'or_next_if_better' set, and so forth).

A match containing the command 'make_transp' will add external circuitry
to simulate 'transparent read', if necessary.

A match containing the command 'make_outreg' will add external flip-flops
to implement synchronous read ports, if necessary.

A match containing the command 'shuffle_enable A' will re-organize
the data bits to accommodate the enable pattern of port A.
\end{lstlisting}

\section{memory\_collect -- creating multi-port memory cells}
\label{cmd:memory_collect}
\begin{lstlisting}[numbers=left,frame=single]
    memory_collect [selection]

This pass collects memories and memory ports and creates generic multiport
memory cells.
\end{lstlisting}

\section{memory\_dff -- merge input/output DFFs into memories}
\label{cmd:memory_dff}
\begin{lstlisting}[numbers=left,frame=single]
    memory_dff [options] [selection]

This pass detects DFFs at memory ports and merges them into the memory port.
I.e. it consumes an asynchronous memory port and the flip-flops at its
interface and yields a synchronous memory port.

    -nordfff
        do not merge registers on read ports
\end{lstlisting}

\section{memory\_map -- translate multiport memories to basic cells}
\label{cmd:memory_map}
\begin{lstlisting}[numbers=left,frame=single]
    memory_map [options] [selection]

This pass converts multiport memory cells as generated by the memory_collect
pass to word-wide DFFs and address decoders.

    -attr !<name>
        do not map memories that have attribute <name> set.

    -attr <name>[=<value>]
        for memories that have attribute <name> set, only map them if its value
        is a string <value> (if specified), or an integer 1 (otherwise). if this
        option is specified multiple times, map the memory if the attribute is
        to any of the values.

    -iattr
        for -attr, ignore case of <value>.
\end{lstlisting}

\section{memory\_memx -- emulate vlog sim behavior for mem ports}
\label{cmd:memory_memx}
\begin{lstlisting}[numbers=left,frame=single]
    memory_memx [selection]

This pass adds additional circuitry that emulates the Verilog simulation
behavior for out-of-bounds memory reads and writes.
\end{lstlisting}

\section{memory\_nordff -- extract read port FFs from memories}
\label{cmd:memory_nordff}
\begin{lstlisting}[numbers=left,frame=single]
    memory_nordff [options] [selection]

This pass extracts FFs from memory read ports. This results in a netlist
similar to what one would get from calling memory_dff with -nordff.
\end{lstlisting}

\section{memory\_share -- consolidate memory ports}
\label{cmd:memory_share}
\begin{lstlisting}[numbers=left,frame=single]
    memory_share [selection]

This pass merges share-able memory ports into single memory ports.

The following methods are used to consolidate the number of memory ports:

  - When write ports are connected to async read ports accessing the same
    address, then this feedback path is converted to a write port with
    byte/part enable signals.

  - When multiple write ports access the same address then this is converted
    to a single write port with a more complex data and/or enable logic path.

  - When multiple write ports are never accessed at the same time (a SAT
    solver is used to determine this), then the ports are merged into a single
    write port.

Note that in addition to the algorithms implemented in this pass, the $memrd
and $memwr cells are also subject to generic resource sharing passes (and other
optimizations) such as "share" and "opt_merge".
\end{lstlisting}

\section{memory\_unpack -- unpack multi-port memory cells}
\label{cmd:memory_unpack}
\begin{lstlisting}[numbers=left,frame=single]
    memory_unpack [selection]

This pass converts the multi-port $mem memory cells into individual $memrd and
$memwr cells. It is the counterpart to the memory_collect pass.
\end{lstlisting}

\section{miter -- automatically create a miter circuit}
\label{cmd:miter}
\begin{lstlisting}[numbers=left,frame=single]
    miter -equiv [options] gold_name gate_name miter_name

Creates a miter circuit for equivalence checking. The gold- and gate- modules
must have the same interfaces. The miter circuit will have all inputs of the
two source modules, prefixed with 'in_'. The miter circuit has a 'trigger'
output that goes high if an output mismatch between the two source modules is
detected.

    -ignore_gold_x
        a undef (x) bit in the gold module output will match any value in
        the gate module output.

    -make_outputs
        also route the gold- and gate-outputs to 'gold_*' and 'gate_*' outputs
        on the miter circuit.

    -make_outcmp
        also create a cmp_* output for each gold/gate output pair.

    -make_assert
        also create an 'assert' cell that checks if trigger is always low.

    -flatten
        call 'flatten -wb; opt_expr -keepdc -undriven;;' on the miter circuit.


    miter -assert [options] module [miter_name]

Creates a miter circuit for property checking. All input ports are kept,
output ports are discarded. An additional output 'trigger' is created that
goes high when an assert is violated. Without a miter_name, the existing
module is modified.

    -make_outputs
        keep module output ports.

    -flatten
        call 'flatten -wb; opt_expr -keepdc -undriven;;' on the miter circuit.
\end{lstlisting}

\section{mutate -- generate or apply design mutations}
\label{cmd:mutate}
\begin{lstlisting}[numbers=left,frame=single]
    mutate -list N [options] [selection]

Create a list of N mutations using an even sampling.

    -o filename
        Write list to this file instead of console output

    -s filename
        Write a list of all src tags found in the design to the specified file

    -seed N
        RNG seed for selecting mutations

    -none
        Include a "none" mutation in the output

    -ctrl name width value
        Add -ctrl options to the output. Use 'value' for first mutation, then
        simply count up from there.

    -mode name
    -module name
    -cell name
    -port name
    -portbit int
    -ctrlbit int
    -wire name
    -wirebit int
    -src string
        Filter list of mutation candidates to those matching
        the given parameters.

    -cfg option int
        Set a configuration option. Options available:
          weight_pq_w weight_pq_b weight_pq_c weight_pq_s
          weight_pq_mw weight_pq_mb weight_pq_mc weight_pq_ms
          weight_cover pick_cover_prcnt


    mutate -mode MODE [options]

Apply the given mutation.

    -ctrl name width value
        Add a control signal with the given name and width. The mutation is
        activated if the control signal equals the given value.

    -module name
    -cell name
    -port name
    -portbit int
    -ctrlbit int
        Mutation parameters, as generated by 'mutate -list N'.

    -wire name
    -wirebit int
    -src string
        Ignored. (They are generated by -list for documentation purposes.)
\end{lstlisting}

\section{muxcover -- cover trees of MUX cells with wider MUXes}
\label{cmd:muxcover}
\begin{lstlisting}[numbers=left,frame=single]
    muxcover [options] [selection]

Cover trees of $_MUX_ cells with $_MUX{4,8,16}_ cells

    -mux4[=cost], -mux8[=cost], -mux16[=cost]
        Cover $_MUX_ trees using the specified types of MUXes (with optional
        integer costs). If none of these options are given, the effect is the
        same as if all of them are.
        Default costs: $_MUX4_ = 220, $_MUX8_ = 460, 
                       $_MUX16_ = 940

    -mux2=cost
        Use the specified cost for $_MUX_ cells when making covering decisions.
        Default cost: $_MUX_ = 100

    -dmux=cost
        Use the specified cost for $_MUX_ cells used in decoders.
        Default cost: 90

    -nodecode
        Do not insert decoder logic. This reduces the number of possible
        substitutions, but guarantees that the resulting circuit is not
        less efficient than the original circuit.

    -nopartial
        Do not consider mappings that use $_MUX<N>_ to select from less
        than <N> different signals.
\end{lstlisting}

\section{muxpack -- \$mux/\$pmux cascades to \$pmux}
\label{cmd:muxpack}
\begin{lstlisting}[numbers=left,frame=single]
    muxpack [selection]

This pass converts cascaded chains of $pmux cells (e.g. those create from case
constructs) and $mux cells (e.g. those created by if-else constructs) into
$pmux cells.

This optimisation is conservative --- it will only pack $mux or $pmux cells
whose select lines are driven by '$eq' cells with other such cells if it can be
certain that their select inputs are mutually exclusive.
\end{lstlisting}

\section{nlutmap -- map to LUTs of different sizes}
\label{cmd:nlutmap}
\begin{lstlisting}[numbers=left,frame=single]
    nlutmap [options] [selection]

This pass uses successive calls to 'abc' to map to an architecture. That
provides a small number of differently sized LUTs.

    -luts N_1,N_2,N_3,...
        The number of LUTs with 1, 2, 3, ... inputs that are
        available in the target architecture.

    -assert
        Create an error if not all logic can be mapped

Excess logic that does not fit into the specified LUTs is mapped back
to generic logic gates ($_AND_, etc.).
\end{lstlisting}

\section{onehot -- optimize \$eq cells for onehot signals}
\label{cmd:onehot}
\begin{lstlisting}[numbers=left,frame=single]
    onehot [options] [selection]

This pass optimizes $eq cells that compare one-hot signals against constants

    -v, -vv
        verbose output
\end{lstlisting}

\section{opt -- perform simple optimizations}
\label{cmd:opt}
\begin{lstlisting}[numbers=left,frame=single]
    opt [options] [selection]

This pass calls all the other opt_* passes in a useful order. This performs
a series of trivial optimizations and cleanups. This pass executes the other
passes in the following order:

    opt_expr [-mux_undef] [-mux_bool] [-undriven] [-clkinv] [-fine] [-full] [-keepdc]
    opt_merge [-share_all] -nomux

    do
        opt_muxtree
        opt_reduce [-fine] [-full]
        opt_merge [-share_all]
        opt_share (-full only)
        opt_rmdff [-keepdc] [-sat]
        opt_clean [-purge]
        opt_expr [-mux_undef] [-mux_bool] [-undriven] [-clkinv] [-fine] [-full] [-keepdc]
    while <changed design>

When called with -fast the following script is used instead:

    do
        opt_expr [-mux_undef] [-mux_bool] [-undriven] [-clkinv] [-fine] [-full] [-keepdc]
        opt_merge [-share_all]
        opt_rmdff [-keepdc] [-sat]
        opt_clean [-purge]
    while <changed design in opt_rmdff>

Note: Options in square brackets (such as [-keepdc]) are passed through to
the opt_* commands when given to 'opt'.
\end{lstlisting}

\section{opt\_clean -- remove unused cells and wires}
\label{cmd:opt_clean}
\begin{lstlisting}[numbers=left,frame=single]
    opt_clean [options] [selection]

This pass identifies wires and cells that are unused and removes them. Other
passes often remove cells but leave the wires in the design or reconnect the
wires but leave the old cells in the design. This pass can be used to clean up
after the passes that do the actual work.

This pass only operates on completely selected modules without processes.

    -purge
        also remove internal nets if they have a public name
\end{lstlisting}

\section{opt\_demorgan -- Optimize reductions with DeMorgan equivalents}
\label{cmd:opt_demorgan}
\begin{lstlisting}[numbers=left,frame=single]
    opt_demorgan [selection]

This pass pushes inverters through $reduce_* cells if this will reduce the
overall gate count of the circuit
\end{lstlisting}

\section{opt\_expr -- perform const folding and simple expression rewriting}
\label{cmd:opt_expr}
\begin{lstlisting}[numbers=left,frame=single]
    opt_expr [options] [selection]

This pass performs const folding on internal cell types with constant inputs.
It also performs some simple expression rewriting.

    -mux_undef
        remove 'undef' inputs from $mux, $pmux and $_MUX_ cells

    -mux_bool
        replace $mux cells with inverters or buffers when possible

    -undriven
        replace undriven nets with undef (x) constants

    -clkinv
        optimize clock inverters by changing FF types

    -fine
        perform fine-grain optimizations

    -full
        alias for -mux_undef -mux_bool -undriven -fine

    -keepdc
        some optimizations change the behavior of the circuit with respect to
        don't-care bits. for example in 'a+0' a single x-bit in 'a' will cause
        all result bits to be set to x. this behavior changes when 'a+0' is
        replaced by 'a'. the -keepdc option disables all such optimizations.
\end{lstlisting}

\section{opt\_lut -- optimize LUT cells}
\label{cmd:opt_lut}
\begin{lstlisting}[numbers=left,frame=single]
    opt_lut [options] [selection]

This pass combines cascaded $lut cells with unused inputs.

    -dlogic <type>:<cell-port>=<LUT-input>[:<cell-port>=<LUT-input>...]
        preserve connections to dedicated logic cell <type> that has ports
        <cell-port> connected to LUT inputs <LUT-input>. this includes
        the case where both LUT and dedicated logic input are connected to
        the same constant.

    -limit N
        only perform the first N combines, then stop. useful for debugging.
\end{lstlisting}

\section{opt\_lut\_ins -- discard unused LUT inputs}
\label{cmd:opt_lut_ins}
\begin{lstlisting}[numbers=left,frame=single]
    opt_lut_ins [options] [selection]

This pass removes unused inputs from LUT cells (that is, inputs that can not
influence the output signal given this LUT's value).  While such LUTs cannot
be directly emitted by ABC, they can be a result of various post-ABC
transformations, such as mapping wide LUTs (not all sub-LUTs will use the
full set of inputs) or optimizations such as xilinx_dffopt.

    -tech <technology>
        Instead of generic $lut cells, operate on LUT cells specific
        to the given technology.  Valid values are: xilinx, ecp5, gowin.
\end{lstlisting}

\section{opt\_mem -- optimize memories}
\label{cmd:opt_mem}
\begin{lstlisting}[numbers=left,frame=single]
    opt_mem [options] [selection]

This pass performs various optimizations on memories in the design.
\end{lstlisting}

\section{opt\_merge -- consolidate identical cells}
\label{cmd:opt_merge}
\begin{lstlisting}[numbers=left,frame=single]
    opt_merge [options] [selection]

This pass identifies cells with identical type and input signals. Such cells
are then merged to one cell.

    -nomux
        Do not merge MUX cells.

    -share_all
        Operate on all cell types, not just built-in types.
\end{lstlisting}

\section{opt\_muxtree -- eliminate dead trees in multiplexer trees}
\label{cmd:opt_muxtree}
\begin{lstlisting}[numbers=left,frame=single]
    opt_muxtree [selection]

This pass analyzes the control signals for the multiplexer trees in the design
and identifies inputs that can never be active. It then removes this dead
branches from the multiplexer trees.

This pass only operates on completely selected modules without processes.
\end{lstlisting}

\section{opt\_reduce -- simplify large MUXes and AND/OR gates}
\label{cmd:opt_reduce}
\begin{lstlisting}[numbers=left,frame=single]
    opt_reduce [options] [selection]

This pass performs two interlinked optimizations:

1. it consolidates trees of large AND gates or OR gates and eliminates
duplicated inputs.

2. it identifies duplicated inputs to MUXes and replaces them with a single
input with the original control signals OR'ed together.

    -fine
      perform fine-grain optimizations

    -full
      alias for -fine
\end{lstlisting}

\section{opt\_rmdff -- remove DFFs with constant inputs}
\label{cmd:opt_rmdff}
\begin{lstlisting}[numbers=left,frame=single]
    opt_rmdff [-keepdc] [-sat] [selection]

This pass identifies flip-flops with constant inputs and replaces them with
a constant driver.

    -sat
        additionally invoke SAT solver to detect and remove flip-flops (with 
        non-constant inputs) that can also be replaced with a constant driver
\end{lstlisting}

\section{opt\_share -- merge mutually exclusive cells of the same type that share an input signal}
\label{cmd:opt_share}
\begin{lstlisting}[numbers=left,frame=single]
    opt_share [selection]

This pass identifies mutually exclusive cells of the same type that:
    (a) share an input signal,
    (b) drive the same $mux, $_MUX_, or $pmux multiplexing cell,

allowing the cell to be merged and the multiplexer to be moved from
multiplexing its output to multiplexing the non-shared input signals.
\end{lstlisting}

\section{paramap -- renaming cell parameters}
\label{cmd:paramap}
\begin{lstlisting}[numbers=left,frame=single]
    paramap [options] [selection]

This command renames cell parameters and/or maps key/value pairs to
other key/value pairs.

    -tocase <name>
        Match attribute names case-insensitively and set it to the specified
        name.

    -rename <old_name> <new_name>
        Rename attributes as specified

    -map <old_name>=<old_value> <new_name>=<new_value>
        Map key/value pairs as indicated.

    -imap <old_name>=<old_value> <new_name>=<new_value>
        Like -map, but use case-insensitive match for <old_value> when
        it is a string value.

    -remove <name>=<value>
        Remove attributes matching this pattern.

For example, mapping Diamond-style ECP5 "init" attributes to Yosys-style:

    paramap -tocase INIT t:LUT4
\end{lstlisting}

\section{peepopt -- collection of peephole optimizers}
\label{cmd:peepopt}
\begin{lstlisting}[numbers=left,frame=single]
    peepopt [options] [selection]

This pass applies a collection of peephole optimizers to the current design.
\end{lstlisting}

\section{plugin -- load and list loaded plugins}
\label{cmd:plugin}
\begin{lstlisting}[numbers=left,frame=single]
    plugin [options]

Load and list loaded plugins.

    -i <plugin_filename>
        Load (install) the specified plugin.

    -a <alias_name>
        Register the specified alias name for the loaded plugin

    -l
        List loaded plugins
\end{lstlisting}

\section{pmux2shiftx -- transform \$pmux cells to \$shiftx cells}
\label{cmd:pmux2shiftx}
\begin{lstlisting}[numbers=left,frame=single]
    pmux2shiftx [options] [selection]

This pass transforms $pmux cells to $shiftx cells.

    -v, -vv
        verbose output

    -min_density <percentage>
        specifies the minimum density for the shifter
        default: 50

    -min_choices <int>
        specified the minimum number of choices for a control signal
        default: 3

    -onehot ignore|pmux|shiftx
        select strategy for one-hot encoded control signals
        default: pmux

    -norange
        disable $sub inference for "range decoders"
\end{lstlisting}

\section{pmuxtree -- transform \$pmux cells to trees of \$mux cells}
\label{cmd:pmuxtree}
\begin{lstlisting}[numbers=left,frame=single]
    pmuxtree [selection]

This pass transforms $pmux cells to trees of $mux cells.
\end{lstlisting}

\section{portlist -- list (top-level) ports}
\label{cmd:portlist}
\begin{lstlisting}[numbers=left,frame=single]
    portlist [options] [selection]

This command lists all module ports found in the selected modules.

If no selection is provided then it lists the ports on the top module.

  -m
    print verilog blackbox module definitions instead of port lists
\end{lstlisting}

\section{prep -- generic synthesis script}
\label{cmd:prep}
\begin{lstlisting}[numbers=left,frame=single]
    prep [options]

This command runs a conservative RTL synthesis. A typical application for this
is the preparation stage of a verification flow. This command does not operate
on partly selected designs.

    -top <module>
        use the specified module as top module (default='top')

    -auto-top
        automatically determine the top of the design hierarchy

    -flatten
        flatten the design before synthesis. this will pass '-auto-top' to
        'hierarchy' if no top module is specified.

    -ifx
        passed to 'proc'. uses verilog simulation behavior for verilog if/case
        undef handling. this also prevents 'wreduce' from being run.

    -memx
        simulate verilog simulation behavior for out-of-bounds memory accesses
        using the 'memory_memx' pass.

    -nomem
        do not run any of the memory_* passes

    -rdff
        do not pass -nordff to 'memory_dff'. This enables merging of FFs into
        memory read ports.

    -nokeepdc
        do not call opt_* with -keepdc

    -run <from_label>[:<to_label>]
        only run the commands between the labels (see below). an empty
        from label is synonymous to 'begin', and empty to label is
        synonymous to the end of the command list.


The following commands are executed by this synthesis command:

    begin:
        hierarchy -check [-top <top> | -auto-top]

    coarse:
        proc [-ifx]
        flatten    (if -flatten)
        opt_expr -keepdc
        opt_clean
        check
        opt -keepdc
        wreduce -keepdc [-memx]
        memory_dff [-nordff]
        memory_memx    (if -memx)
        opt_clean
        memory_collect
        opt -keepdc -fast

    check:
        stat
        check
\end{lstlisting}

\section{proc -- translate processes to netlists}
\label{cmd:proc}
\begin{lstlisting}[numbers=left,frame=single]
    proc [options] [selection]

This pass calls all the other proc_* passes in the most common order.

    proc_clean
    proc_rmdead
    proc_prune
    proc_init
    proc_arst
    proc_mux
    proc_dlatch
    proc_dff
    proc_clean

This replaces the processes in the design with multiplexers,
flip-flops and latches.

The following options are supported:

    -global_arst [!]<netname>
        This option is passed through to proc_arst.

    -ifx
        This option is passed through to proc_mux. proc_rmdead is not
        executed in -ifx mode.
\end{lstlisting}

\section{proc\_arst -- detect asynchronous resets}
\label{cmd:proc_arst}
\begin{lstlisting}[numbers=left,frame=single]
    proc_arst [-global_arst [!]<netname>] [selection]

This pass identifies asynchronous resets in the processes and converts them
to a different internal representation that is suitable for generating
flip-flop cells with asynchronous resets.

    -global_arst [!]<netname>
        In modules that have a net with the given name, use this net as async
        reset for registers that have been assign initial values in their
        declaration ('reg foobar = constant_value;'). Use the '!' modifier for
        active low reset signals. Note: the frontend stores the default value
        in the 'init' attribute on the net.
\end{lstlisting}

\section{proc\_clean -- remove empty parts of processes}
\label{cmd:proc_clean}
\begin{lstlisting}[numbers=left,frame=single]
    proc_clean [options] [selection]

    -quiet
        do not print any messages.

This pass removes empty parts of processes and ultimately removes a process
if it contains only empty structures.
\end{lstlisting}

\section{proc\_dff -- extract flip-flops from processes}
\label{cmd:proc_dff}
\begin{lstlisting}[numbers=left,frame=single]
    proc_dff [selection]

This pass identifies flip-flops in the processes and converts them to
d-type flip-flop cells.
\end{lstlisting}

\section{proc\_dlatch -- extract latches from processes}
\label{cmd:proc_dlatch}
\begin{lstlisting}[numbers=left,frame=single]
    proc_dlatch [selection]

This pass identifies latches in the processes and converts them to
d-type latches.
\end{lstlisting}

\section{proc\_init -- convert initial block to init attributes}
\label{cmd:proc_init}
\begin{lstlisting}[numbers=left,frame=single]
    proc_init [selection]

This pass extracts the 'init' actions from processes (generated from Verilog
'initial' blocks) and sets the initial value to the 'init' attribute on the
respective wire.
\end{lstlisting}

\section{proc\_mux -- convert decision trees to multiplexers}
\label{cmd:proc_mux}
\begin{lstlisting}[numbers=left,frame=single]
    proc_mux [options] [selection]

This pass converts the decision trees in processes (originating from if-else
and case statements) to trees of multiplexer cells.

    -ifx
        Use Verilog simulation behavior with respect to undef values in
        'case' expressions and 'if' conditions.
\end{lstlisting}

\section{proc\_prune -- remove redundant assignments}
\label{cmd:proc_prune}
\begin{lstlisting}[numbers=left,frame=single]
    proc_prune [selection]

This pass identifies assignments in processes that are always overwritten by
a later assignment to the same signal and removes them.
\end{lstlisting}

\section{proc\_rmdead -- eliminate dead trees in decision trees}
\label{cmd:proc_rmdead}
\begin{lstlisting}[numbers=left,frame=single]
    proc_rmdead [selection]

This pass identifies unreachable branches in decision trees and removes them.
\end{lstlisting}

\section{qwp -- quadratic wirelength placer}
\label{cmd:qwp}
\begin{lstlisting}[numbers=left,frame=single]
    qwp [options] [selection]

This command runs quadratic wirelength placement on the selected modules and
annotates the cells in the design with 'qwp_position' attributes.

    -ltr
        Add left-to-right constraints: constrain all inputs on the left border
        outputs to the right border.

    -alpha
        Add constraints for inputs/outputs to be placed in alphanumerical
        order along the y-axis (top-to-bottom).

    -grid N
        Number of grid divisions in x- and y-direction. (default=16)

    -dump <html_file_name>
        Dump a protocol of the placement algorithm to the html file.

    -v
        Verbose solver output for profiling or debugging

Note: This implementation of a quadratic wirelength placer uses exact
dense matrix operations. It is only a toy-placer for small circuits.
\end{lstlisting}

\section{read -- load HDL designs}
\label{cmd:read}
\begin{lstlisting}[numbers=left,frame=single]
    read {-vlog95|-vlog2k|-sv2005|-sv2009|-sv2012|-sv|-formal} <verilog-file>..

Load the specified Verilog/SystemVerilog files. (Full SystemVerilog support
is only available via Verific.)

Additional -D<macro>[=<value>] options may be added after the option indicating
the language version (and before file names) to set additional verilog defines.


    read {-vhdl87|-vhdl93|-vhdl2k|-vhdl2008|-vhdl} <vhdl-file>..

Load the specified VHDL files. (Requires Verific.)


    read -define <macro>[=<value>]..

Set global Verilog/SystemVerilog defines.


    read -undef <macro>..

Unset global Verilog/SystemVerilog defines.


    read -incdir <directory>

Add directory to global Verilog/SystemVerilog include directories.


    read -verific
    read -noverific

Subsequent calls to 'read' will either use or not use Verific. Calling 'read'
with -verific will result in an error on Yosys binaries that are built without
Verific support. The default is to use Verific if it is available.
\end{lstlisting}

\section{read\_aiger -- read AIGER file}
\label{cmd:read_aiger}
\begin{lstlisting}[numbers=left,frame=single]
    read_aiger [options] [filename]

Load module from an AIGER file into the current design.

    -module_name <module_name>
        name of module to be created (default: <filename>)

    -clk_name <wire_name>
        if specified, AIGER latches to be transformed into $_DFF_P_ cells
        clocked by wire of this name. otherwise, $_FF_ cells will be used

    -map <filename>
        read file with port and latch symbols

    -wideports
        merge ports that match the pattern 'name[int]' into a single
        multi-bit port 'name'

    -xaiger
        read XAIGER extensions
\end{lstlisting}

\section{read\_blif -- read BLIF file}
\label{cmd:read_blif}
\begin{lstlisting}[numbers=left,frame=single]
    read_blif [options] [filename]

Load modules from a BLIF file into the current design.

    -sop
        Create $sop cells instead of $lut cells

    -wideports
        Merge ports that match the pattern 'name[int]' into a single
        multi-bit port 'name'.
\end{lstlisting}

\section{read\_ilang -- read modules from ilang file}
\label{cmd:read_ilang}
\begin{lstlisting}[numbers=left,frame=single]
    read_ilang [filename]

Load modules from an ilang file to the current design. (ilang is a text
representation of a design in yosys's internal format.)

    -nooverwrite
        ignore re-definitions of modules. (the default behavior is to
        create an error message if the existing module is not a blackbox
        module, and overwrite the existing module if it is a blackbox module.)

    -overwrite
        overwrite existing modules with the same name

    -lib
        only create empty blackbox modules
\end{lstlisting}

\section{read\_json -- read JSON file}
\label{cmd:read_json}
\begin{lstlisting}[numbers=left,frame=single]
    read_json [filename]

Load modules from a JSON file into the current design See "help write_json"
for a description of the file format.
\end{lstlisting}

\section{read\_liberty -- read cells from liberty file}
\label{cmd:read_liberty}
\begin{lstlisting}[numbers=left,frame=single]
    read_liberty [filename]

Read cells from liberty file as modules into current design.

    -lib
        only create empty blackbox modules

    -nooverwrite
        ignore re-definitions of modules. (the default behavior is to
        create an error message if the existing module is not a blackbox
        module, and overwrite the existing module if it is  a blackbox module.)

    -overwrite
        overwrite existing modules with the same name

    -ignore_miss_func
        ignore cells with missing function specification of outputs

    -ignore_miss_dir
        ignore cells with a missing or invalid direction
        specification on a pin

    -ignore_miss_data_latch
        ignore latches with missing data and/or enable pins

    -setattr <attribute_name>
        set the specified attribute (to the value 1) on all loaded modules
\end{lstlisting}

\section{read\_verilog -- read modules from Verilog file}
\label{cmd:read_verilog}
\begin{lstlisting}[numbers=left,frame=single]
    read_verilog [options] [filename]

Load modules from a Verilog file to the current design. A large subset of
Verilog-2005 is supported.

    -sv
        enable support for SystemVerilog features. (only a small subset
        of SystemVerilog is supported)

    -formal
        enable support for SystemVerilog assertions and some Yosys extensions
        replace the implicit -D SYNTHESIS with -D FORMAL

    -noassert
        ignore assert() statements

    -noassume
        ignore assume() statements

    -norestrict
        ignore restrict() statements

    -assume-asserts
        treat all assert() statements like assume() statements

    -assert-assumes
        treat all assume() statements like assert() statements

    -debug
        alias for -dump_ast1 -dump_ast2 -dump_vlog1 -dump_vlog2 -yydebug

    -dump_ast1
        dump abstract syntax tree (before simplification)

    -dump_ast2
        dump abstract syntax tree (after simplification)

    -no_dump_ptr
        do not include hex memory addresses in dump (easier to diff dumps)

    -dump_vlog1
        dump ast as Verilog code (before simplification)

    -dump_vlog2
        dump ast as Verilog code (after simplification)

    -dump_rtlil
        dump generated RTLIL netlist

    -yydebug
        enable parser debug output

    -nolatches
        usually latches are synthesized into logic loops
        this option prohibits this and sets the output to 'x'
        in what would be the latches hold condition

        this behavior can also be achieved by setting the
        'nolatches' attribute on the respective module or
        always block.

    -nomem2reg
        under certain conditions memories are converted to registers
        early during simplification to ensure correct handling of
        complex corner cases. this option disables this behavior.

        this can also be achieved by setting the 'nomem2reg'
        attribute on the respective module or register.

        This is potentially dangerous. Usually the front-end has good
        reasons for converting an array to a list of registers.
        Prohibiting this step will likely result in incorrect synthesis
        results.

    -mem2reg
        always convert memories to registers. this can also be
        achieved by setting the 'mem2reg' attribute on the respective
        module or register.

    -nomeminit
        do not infer $meminit cells and instead convert initialized
        memories to registers directly in the front-end.

    -ppdump
        dump Verilog code after pre-processor

    -nopp
        do not run the pre-processor

    -nodpi
        disable DPI-C support

    -noblackbox
        do not automatically add a (* blackbox *) attribute to an
        empty module.

    -lib
        only create empty blackbox modules. This implies -DBLACKBOX.
        modules with the (* whitebox *) attribute will be preserved.
        (* lib_whitebox *) will be treated like (* whitebox *).

    -nowb
        delete (* whitebox *) and (* lib_whitebox *) attributes from
        all modules.

    -specify
        parse and import specify blocks

    -noopt
        don't perform basic optimizations (such as const folding) in the
        high-level front-end.

    -icells
        interpret cell types starting with '$' as internal cell types

    -pwires
        add a wire for each module parameter

    -nooverwrite
        ignore re-definitions of modules. (the default behavior is to
        create an error message if the existing module is not a black box
        module, and overwrite the existing module otherwise.)

    -overwrite
        overwrite existing modules with the same name

    -defer
        only read the abstract syntax tree and defer actual compilation
        to a later 'hierarchy' command. Useful in cases where the default
        parameters of modules yield invalid or not synthesizable code.

    -noautowire
        make the default of `default_nettype be "none" instead of "wire".

    -setattr <attribute_name>
        set the specified attribute (to the value 1) on all loaded modules

    -Dname[=definition]
        define the preprocessor symbol 'name' and set its optional value
        'definition'

    -Idir
        add 'dir' to the directories which are used when searching include
        files

The command 'verilog_defaults' can be used to register default options for
subsequent calls to 'read_verilog'.

Note that the Verilog frontend does a pretty good job of processing valid
verilog input, but has not very good error reporting. It generally is
recommended to use a simulator (for example Icarus Verilog) for checking
the syntax of the code, rather than to rely on read_verilog for that.

Depending on if read_verilog is run in -formal mode, either the macro
SYNTHESIS or FORMAL is defined automatically. In addition, read_verilog
always defines the macro YOSYS.

See the Yosys README file for a list of non-standard Verilog features
supported by the Yosys Verilog front-end.
\end{lstlisting}

\section{rename -- rename object in the design}
\label{cmd:rename}
\begin{lstlisting}[numbers=left,frame=single]
    rename old_name new_name

Rename the specified object. Note that selection patterns are not supported
by this command.



    rename -output old_name new_name

Like above, but also make the wire an output. This will fail if the object is
not a wire.


    rename -src [selection]

Assign names auto-generated from the src attribute to all selected wires and
cells with private names.


    rename -wire [selection]

Assign auto-generated names based on the wires they drive to all selected
cells with private names. Ignores cells driving privatly named wires.


    rename -enumerate [-pattern <pattern>] [selection]

Assign short auto-generated names to all selected wires and cells with private
names. The -pattern option can be used to set the pattern for the new names.
The character % in the pattern is replaced with a integer number. The default
pattern is '_%_'.


    rename -hide [selection]

Assign private names (the ones with $-prefix) to all selected wires and cells
with public names. This ignores all selected ports.


    rename -top new_name

Rename top module.
\end{lstlisting}

\section{rmports -- remove module ports with no connections}
\label{cmd:rmports}
\begin{lstlisting}[numbers=left,frame=single]
    rmports [selection]

This pass identifies ports in the selected modules which are not used or
driven and removes them.
\end{lstlisting}

\section{sat -- solve a SAT problem in the circuit}
\label{cmd:sat}
\begin{lstlisting}[numbers=left,frame=single]
    sat [options] [selection]

This command solves a SAT problem defined over the currently selected circuit
and additional constraints passed as parameters.

    -all
        show all solutions to the problem (this can grow exponentially, use
        -max <N> instead to get <N> solutions)

    -max <N>
        like -all, but limit number of solutions to <N>

    -enable_undef
        enable modeling of undef value (aka 'x-bits')
        this option is implied by -set-def, -set-undef et. cetera

    -max_undef
        maximize the number of undef bits in solutions, giving a better
        picture of which input bits are actually vital to the solution.

    -set <signal> <value>
        set the specified signal to the specified value.

    -set-def <signal>
        add a constraint that all bits of the given signal must be defined

    -set-any-undef <signal>
        add a constraint that at least one bit of the given signal is undefined

    -set-all-undef <signal>
        add a constraint that all bits of the given signal are undefined

    -set-def-inputs
        add -set-def constraints for all module inputs

    -show <signal>
        show the model for the specified signal. if no -show option is
        passed then a set of signals to be shown is automatically selected.

    -show-inputs, -show-outputs, -show-ports
        add all module (input/output) ports to the list of shown signals

    -show-regs, -show-public, -show-all
        show all registers, show signals with 'public' names, show all signals

    -ignore_div_by_zero
        ignore all solutions that involve a division by zero

    -ignore_unknown_cells
        ignore all cells that can not be matched to a SAT model

The following options can be used to set up a sequential problem:

    -seq <N>
        set up a sequential problem with <N> time steps. The steps will
        be numbered from 1 to N.

        note: for large <N> it can be significantly faster to use
        -tempinduct-baseonly -maxsteps <N> instead of -seq <N>.

    -set-at <N> <signal> <value>
    -unset-at <N> <signal>
        set or unset the specified signal to the specified value in the
        given timestep. this has priority over a -set for the same signal.

    -set-assumes
        set all assumptions provided via $assume cells

    -set-def-at <N> <signal>
    -set-any-undef-at <N> <signal>
    -set-all-undef-at <N> <signal>
        add undef constraints in the given timestep.

    -set-init <signal> <value>
        set the initial value for the register driving the signal to the value

    -set-init-undef
        set all initial states (not set using -set-init) to undef

    -set-init-def
        do not force a value for the initial state but do not allow undef

    -set-init-zero
        set all initial states (not set using -set-init) to zero

    -dump_vcd <vcd-file-name>
        dump SAT model (counter example in proof) to VCD file

    -dump_json <json-file-name>
        dump SAT model (counter example in proof) to a WaveJSON file.

    -dump_cnf <cnf-file-name>
        dump CNF of SAT problem (in DIMACS format). in temporal induction
        proofs this is the CNF of the first induction step.

The following additional options can be used to set up a proof. If also -seq
is passed, a temporal induction proof is performed.

    -tempinduct
        Perform a temporal induction proof. In a temporal induction proof it is
        proven that the condition holds forever after the number of time steps
        specified using -seq.

    -tempinduct-def
        Perform a temporal induction proof. Assume an initial state with all
        registers set to defined values for the induction step.

    -tempinduct-baseonly
        Run only the basecase half of temporal induction (requires -maxsteps)

    -tempinduct-inductonly
        Run only the induction half of temporal induction

    -tempinduct-skip <N>
        Skip the first <N> steps of the induction proof.

        note: this will assume that the base case holds for <N> steps.
        this must be proven independently with "-tempinduct-baseonly
        -maxsteps <N>". Use -initsteps if you just want to set a
        minimal induction length.

    -prove <signal> <value>
        Attempt to proof that <signal> is always <value>.

    -prove-x <signal> <value>
        Like -prove, but an undef (x) bit in the lhs matches any value on
        the right hand side. Useful for equivalence checking.

    -prove-asserts
        Prove that all asserts in the design hold.

    -prove-skip <N>
        Do not enforce the prove-condition for the first <N> time steps.

    -maxsteps <N>
        Set a maximum length for the induction.

    -initsteps <N>
        Set initial length for the induction.
        This will speed up the search of the right induction length
        for deep induction proofs.

    -stepsize <N>
        Increase the size of the induction proof in steps of <N>.
        This will speed up the search of the right induction length
        for deep induction proofs.

    -timeout <N>
        Maximum number of seconds a single SAT instance may take.

    -verify
        Return an error and stop the synthesis script if the proof fails.

    -verify-no-timeout
        Like -verify but do not return an error for timeouts.

    -falsify
        Return an error and stop the synthesis script if the proof succeeds.

    -falsify-no-timeout
        Like -falsify but do not return an error for timeouts.
\end{lstlisting}

\section{scatter -- add additional intermediate nets}
\label{cmd:scatter}
\begin{lstlisting}[numbers=left,frame=single]
    scatter [selection]

This command adds additional intermediate nets on all cell ports. This is used
for testing the correct use of the SigMap helper in passes. If you don't know
what this means: don't worry -- you only need this pass when testing your own
extensions to Yosys.

Use the opt_clean command to get rid of the additional nets.
\end{lstlisting}

\section{scc -- detect strongly connected components (logic loops)}
\label{cmd:scc}
\begin{lstlisting}[numbers=left,frame=single]
    scc [options] [selection]

This command identifies strongly connected components (aka logic loops) in the
design.

    -expect <num>
        expect to find exactly <num> SSCs. A different number of SSCs will
        produce an error.

    -max_depth <num>
        limit to loops not longer than the specified number of cells. This
        can e.g. be useful in identifying small local loops in a module that
        implements one large SCC.

    -nofeedback
        do not count cells that have their output fed back into one of their
        inputs as single-cell scc.

    -all_cell_types
        Usually this command only considers internal non-memory cells. With
        this option set, all cells are considered. For unknown cells all ports
        are assumed to be bidirectional 'inout' ports.

    -set_attr <name> <value>
        set the specified attribute on all cells that are part of a logic
        loop. the special token {} in the value is replaced with a unique
        identifier for the logic loop.

    -select
        replace the current selection with a selection of all cells and wires
        that are part of a found logic loop
\end{lstlisting}

\section{scratchpad -- get/set values in the scratchpad}
\label{cmd:scratchpad}
\begin{lstlisting}[numbers=left,frame=single]
    scratchpad [options]

This pass allows to read and modify values from the scratchpad of the current
design. Options:

    -get <identifier>
        print the value saved in the scratchpad under the given identifier.

    -set <identifier> <value>
        save the given value in the scratchpad under the given identifier.

    -unset <identifier>
        remove the entry for the given identifier from the scratchpad.

    -copy <identifier_from> <identifier_to>
        copy the value of the first identifier to the second identifier.

    -assert <identifier> <value>
        assert that the entry for the given identifier is set to the given value.

    -assert-set <identifier>
        assert that the entry for the given identifier exists.

    -assert-unset <identifier>
        assert that the entry for the given identifier does not exist.

The identifier may not contain whitespace. By convention, it is usually prefixed
by the name of the pass that uses it, e.g. 'opt.did_something'. If the value
contains whitespace, it must be enclosed in double quotes.
\end{lstlisting}

\section{script -- execute commands from file or wire}
\label{cmd:script}
\begin{lstlisting}[numbers=left,frame=single]
    script <filename> [<from_label>:<to_label>]
    script -scriptwire [selection]

This command executes the yosys commands in the specified file (default
behaviour), or commands embedded in the constant text value connected to the
selected wires.

In the default (file) case, the 2nd argument can be used to only execute the
section of the file between the specified labels. An empty from label is
synonymous with the beginning of the file and an empty to label is synonymous
with the end of the file.

If only one label is specified (without ':') then only the block
marked with that label (until the next label) is executed.

In "-scriptwire" mode, the commands on the selected wire(s) will be executed
in the scope of (and thus, relative to) the wires' owning module(s). This
'-module' mode can be exited by using the 'cd' command.
\end{lstlisting}

\section{select -- modify and view the list of selected objects}
\label{cmd:select}
\begin{lstlisting}[numbers=left,frame=single]
    select [ -add | -del | -set <name> ] {-read <filename> | <selection>}
    select [ <assert_option> ] {-read <filename> | <selection>}
    select [ -list | -write <filename> | -count | -clear ]
    select -module <modname>

Most commands use the list of currently selected objects to determine which part
of the design to operate on. This command can be used to modify and view this
list of selected objects.

Note that many commands support an optional [selection] argument that can be
used to override the global selection for the command. The syntax of this
optional argument is identical to the syntax of the <selection> argument
described here.

    -add, -del
        add or remove the given objects to the current selection.
        without this options the current selection is replaced.

    -set <name>
        do not modify the current selection. instead save the new selection
        under the given name (see @<name> below). to save the current selection,
        use "select -set <name> %"

    -assert-none
        do not modify the current selection. instead assert that the given
        selection is empty. i.e. produce an error if any object matching the
        selection is found.

    -assert-any
        do not modify the current selection. instead assert that the given
        selection is non-empty. i.e. produce an error if no object matching
        the selection is found.

    -assert-count N
        do not modify the current selection. instead assert that the given
        selection contains exactly N objects.

    -assert-max N
        do not modify the current selection. instead assert that the given
        selection contains less than or exactly N objects.

    -assert-min N
        do not modify the current selection. instead assert that the given
        selection contains at least N objects.

    -list
        list all objects in the current selection

    -write <filename>
        like -list but write the output to the specified file

    -read <filename>
        read the specified file (written by -write)

    -count
        count all objects in the current selection

    -clear
        clear the current selection. this effectively selects the whole
        design. it also resets the selected module (see -module). use the
        command 'select *' to select everything but stay in the current module.

    -none
        create an empty selection. the current module is unchanged.

    -module <modname>
        limit the current scope to the specified module.
        the difference between this and simply selecting the module
        is that all object names are interpreted relative to this
        module after this command until the selection is cleared again.

When this command is called without an argument, the current selection
is displayed in a compact form (i.e. only the module name when a whole module
is selected).

The <selection> argument itself is a series of commands for a simple stack
machine. Each element on the stack represents a set of selected objects.
After this commands have been executed, the union of all remaining sets
on the stack is computed and used as selection for the command.

Pushing (selecting) object when not in -module mode:

    <mod_pattern>
        select the specified module(s)

    <mod_pattern>/<obj_pattern>
        select the specified object(s) from the module(s)

Pushing (selecting) object when in -module mode:

    <obj_pattern>
        select the specified object(s) from the current module

A <mod_pattern> can be a module name, wildcard expression (*, ?, [..])
matching module names, or one of the following:

    A:<pattern>, A:<pattern>=<pattern>
        all modules with an attribute matching the given pattern
        in addition to = also <, <=, >=, and > are supported

    N:<pattern>
        all modules with a name matching the given pattern
        (i.e. 'N:' is optional as it is the default matching rule)

An <obj_pattern> can be an object name, wildcard expression, or one of
the following:

    w:<pattern>
        all wires with a name matching the given wildcard pattern

    i:<pattern>, o:<pattern>, x:<pattern>
        all inputs (i:), outputs (o:) or any ports (x:) with matching names

    s:<size>, s:<min>:<max>
        all wires with a matching width

    m:<pattern>
        all memories with a name matching the given pattern

    c:<pattern>
        all cells with a name matching the given pattern

    t:<pattern>
        all cells with a type matching the given pattern

    p:<pattern>
        all processes with a name matching the given pattern

    a:<pattern>
        all objects with an attribute name matching the given pattern

    a:<pattern>=<pattern>
        all objects with a matching attribute name-value-pair.
        in addition to = also <, <=, >=, and > are supported

    r:<pattern>, r:<pattern>=<pattern>
        cells with matching parameters. also with <, <=, >= and >.

    n:<pattern>
        all objects with a name matching the given pattern
        (i.e. 'n:' is optional as it is the default matching rule)

    @<name>
        push the selection saved prior with 'select -set <name> ...'

The following actions can be performed on the top sets on the stack:

    %
        push a copy of the current selection to the stack

    %%
        replace the stack with a union of all elements on it

    %n
        replace top set with its invert

    %u
        replace the two top sets on the stack with their union

    %i
        replace the two top sets on the stack with their intersection

    %d
        pop the top set from the stack and subtract it from the new top

    %D
        like %d but swap the roles of two top sets on the stack

    %c
        create a copy of the top set from the stack and push it

    %x[<num1>|*][.<num2>][:<rule>[:<rule>..]]
        expand top set <num1> num times according to the specified rules.
        (i.e. select all cells connected to selected wires and select all
        wires connected to selected cells) The rules specify which cell
        ports to use for this. the syntax for a rule is a '-' for exclusion
        and a '+' for inclusion, followed by an optional comma separated
        list of cell types followed by an optional comma separated list of
        cell ports in square brackets. a rule can also be just a cell or wire
        name that limits the expansion (is included but does not go beyond).
        select at most <num2> objects. a warning message is printed when this
        limit is reached. When '*' is used instead of <num1> then the process
        is repeated until no further object are selected.

    %ci[<num1>|*][.<num2>][:<rule>[:<rule>..]]
    %co[<num1>|*][.<num2>][:<rule>[:<rule>..]]
        similar to %x, but only select input (%ci) or output cones (%co)

    %xe[...] %cie[...] %coe
        like %x, %ci, and %co but only consider combinatorial cells

    %a
        expand top set by selecting all wires that are (at least in part)
        aliases for selected wires.

    %s
        expand top set by adding all modules that implement cells in selected
        modules

    %m
        expand top set by selecting all modules that contain selected objects

    %M
        select modules that implement selected cells

    %C
        select cells that implement selected modules

    %R[<num>]
        select <num> random objects from top selection (default 1)

Example: the following command selects all wires that are connected to a
'GATE' input of a 'SWITCH' cell:

    select */t:SWITCH %x:+[GATE] */t:SWITCH %d
\end{lstlisting}

\section{setattr -- set/unset attributes on objects}
\label{cmd:setattr}
\begin{lstlisting}[numbers=left,frame=single]
    setattr [ -mod ] [ -set name value | -unset name ]... [selection]

Set/unset the given attributes on the selected objects. String values must be
passed in double quotes (").

When called with -mod, this command will set and unset attributes on modules
instead of objects within modules.
\end{lstlisting}

\section{setparam -- set/unset parameters on objects}
\label{cmd:setparam}
\begin{lstlisting}[numbers=left,frame=single]
    setparam [ -type cell_type ] [ -set name value | -unset name ]... [selection]

Set/unset the given parameters on the selected cells. String values must be
passed in double quotes (").

The -type option can be used to change the cell type of the selected cells.
\end{lstlisting}

\section{setundef -- replace undef values with defined constants}
\label{cmd:setundef}
\begin{lstlisting}[numbers=left,frame=single]
    setundef [options] [selection]

This command replaces undef (x) constants with defined (0/1) constants.

    -undriven
        also set undriven nets to constant values

    -expose
        also expose undriven nets as inputs (use with -undriven)

    -zero
        replace with bits cleared (0)

    -one
        replace with bits set (1)

    -undef
        replace with undef (x) bits, may be used with -undriven

    -anyseq
        replace with $anyseq drivers (for formal)

    -anyconst
        replace with $anyconst drivers (for formal)

    -random <seed>
        replace with random bits using the specified integer as seed
        value for the random number generator.

    -init
        also create/update init values for flip-flops

    -params
        replace undef in cell parameters
\end{lstlisting}

\section{sf2\_iobs -- SF2: insert IO buffers}
\label{cmd:sf2_iobs}
\begin{lstlisting}[numbers=left,frame=single]
    sf2_iobs [options] [selection]

Add SF2 I/O buffers and global buffers to top module as needed.

    -clkbuf
        Insert PAD->global_net clock buffers
\end{lstlisting}

\section{share -- perform sat-based resource sharing}
\label{cmd:share}
\begin{lstlisting}[numbers=left,frame=single]
    share [options] [selection]

This pass merges shareable resources into a single resource. A SAT solver
is used to determine if two resources are share-able.

  -force
    Per default the selection of cells that is considered for sharing is
    narrowed using a list of cell types. With this option all selected
    cells are considered for resource sharing.

    IMPORTANT NOTE: If the -all option is used then no cells with internal
    state must be selected!

  -aggressive
    Per default some heuristics are used to reduce the number of cells
    considered for resource sharing to only large resources. This options
    turns this heuristics off, resulting in much more cells being considered
    for resource sharing.

  -fast
    Only consider the simple part of the control logic in SAT solving, resulting
    in much easier SAT problems at the cost of maybe missing some opportunities
    for resource sharing.

  -limit N
    Only perform the first N merges, then stop. This is useful for debugging.
\end{lstlisting}

\section{shell -- enter interactive command mode}
\label{cmd:shell}
\begin{lstlisting}[numbers=left,frame=single]
    shell

This command enters the interactive command mode. This can be useful
in a script to interrupt the script at a certain point and allow for
interactive inspection or manual synthesis of the design at this point.

The command prompt of the interactive shell indicates the current
selection (see 'help select'):

    yosys>
        the entire design is selected

    yosys*>
        only part of the design is selected

    yosys [modname]>
        the entire module 'modname' is selected using 'select -module modname'

    yosys [modname]*>
        only part of current module 'modname' is selected

When in interactive shell, some errors (e.g. invalid command arguments)
do not terminate yosys but return to the command prompt.

This command is the default action if nothing else has been specified
on the command line.

Press Ctrl-D or type 'exit' to leave the interactive shell.
\end{lstlisting}

\section{show -- generate schematics using graphviz}
\label{cmd:show}
\begin{lstlisting}[numbers=left,frame=single]
    show [options] [selection]

Create a graphviz DOT file for the selected part of the design and compile it
to a graphics file (usually SVG or PostScript).

    -viewer <viewer>
        Run the specified command with the graphics file as parameter.
        On Windows, this pauses yosys until the viewer exits.

    -format <format>
        Generate a graphics file in the specified format. Use 'dot' to just
        generate a .dot file, or other <format> strings such as 'svg' or 'ps'
        to generate files in other formats (this calls the 'dot' command).

    -lib <verilog_or_ilang_file>
        Use the specified library file for determining whether cell ports are
        inputs or outputs. This option can be used multiple times to specify
        more than one library.

        note: in most cases it is better to load the library before calling
        show with 'read_verilog -lib <filename>'. it is also possible to
        load liberty files with 'read_liberty -lib <filename>'.

    -prefix <prefix>
        generate <prefix>.* instead of ~/.yosys_show.*

    -color <color> <object>
        assign the specified color to the specified object. The object can be
        a single selection wildcard expressions or a saved set of objects in
        the @<name> syntax (see "help select" for details).

    -label <text> <object>
        assign the specified label text to the specified object. The object can
        be a single selection wildcard expressions or a saved set of objects in
        the @<name> syntax (see "help select" for details).

    -colors <seed>
        Randomly assign colors to the wires. The integer argument is the seed
        for the random number generator. Change the seed value if the colored
        graph still is ambiguous. A seed of zero deactivates the coloring.

    -colorattr <attribute_name>
        Use the specified attribute to assign colors. A unique color is
        assigned to each unique value of this attribute.

    -width
        annotate buses with a label indicating the width of the bus.

    -signed
        mark ports (A, B) that are declared as signed (using the [AB]_SIGNED
        cell parameter) with an asterisk next to the port name.

    -stretch
        stretch the graph so all inputs are on the left side and all outputs
        (including inout ports) are on the right side.

    -pause
        wait for the use to press enter to before returning

    -enum
        enumerate objects with internal ($-prefixed) names

    -long
        do not abbreviate objects with internal ($-prefixed) names

    -notitle
        do not add the module name as graph title to the dot file

    -nobg
        don't run viewer in the background, IE wait for the viewer tool to
        exit before returning

When no <format> is specified, 'dot' is used. When no <format> and <viewer> is
specified, 'xdot' is used to display the schematic (POSIX systems only).

The generated output files are '~/.yosys_show.dot' and '~/.yosys_show.<format>',
unless another prefix is specified using -prefix <prefix>.

Yosys on Windows and YosysJS use different defaults: The output is written
to 'show.dot' in the current directory and new viewer is launched each time
the 'show' command is executed.
\end{lstlisting}

\section{shregmap -- map shift registers}
\label{cmd:shregmap}
\begin{lstlisting}[numbers=left,frame=single]
    shregmap [options] [selection]

This pass converts chains of $_DFF_[NP]_ gates to target specific shift register
primitives. The generated shift register will be of type $__SHREG_DFF_[NP]_ and
will use the same interface as the original $_DFF_*_ cells. The cell parameter
'DEPTH' will contain the depth of the shift register. Use a target-specific
'techmap' map file to convert those cells to the actual target cells.

    -minlen N
        minimum length of shift register (default = 2)
        (this is the length after -keep_before and -keep_after)

    -maxlen N
        maximum length of shift register (default = no limit)
        larger chains will be mapped to multiple shift register instances

    -keep_before N
        number of DFFs to keep before the shift register (default = 0)

    -keep_after N
        number of DFFs to keep after the shift register (default = 0)

    -clkpol pos|neg|any
        limit match to only positive or negative edge clocks. (default = any)

    -enpol pos|neg|none|any_or_none|any
        limit match to FFs with the specified enable polarity. (default = none)

    -match <cell_type>[:<d_port_name>:<q_port_name>]
        match the specified cells instead of $_DFF_N_ and $_DFF_P_. If
        ':<d_port_name>:<q_port_name>' is omitted then 'D' and 'Q' is used
        by default. E.g. the option '-clkpol pos' is just an alias for
        '-match $_DFF_P_', which is an alias for '-match $_DFF_P_:D:Q'.

    -params
        instead of encoding the clock and enable polarity in the cell name by
        deriving from the original cell name, simply name all generated cells
        $__SHREG_ and use CLKPOL and ENPOL parameters. An ENPOL value of 2 is
        used to denote cells without enable input. The ENPOL parameter is
        omitted when '-enpol none' (or no -enpol option) is passed.

    -zinit
        assume the shift register is automatically zero-initialized, so it
        becomes legal to merge zero initialized FFs into the shift register.

    -init
        map initialized registers to the shift reg, add an INIT parameter to
        generated cells with the initialization value. (first bit to shift out
        in LSB position)

    -tech greenpak4
        map to greenpak4 shift registers.
\end{lstlisting}

\section{sim -- simulate the circuit}
\label{cmd:sim}
\begin{lstlisting}[numbers=left,frame=single]
    sim [options] [top-level]

This command simulates the circuit using the given top-level module.

    -vcd <filename>
        write the simulation results to the given VCD file

    -clock <portname>
        name of top-level clock input

    -clockn <portname>
        name of top-level clock input (inverse polarity)

    -reset <portname>
        name of top-level reset input (active high)

    -resetn <portname>
        name of top-level inverted reset input (active low)

    -rstlen <integer>
        number of cycles reset should stay active (default: 1)

    -zinit
        zero-initialize all uninitialized regs and memories

    -n <integer>
        number of cycles to simulate (default: 20)

    -a
        include all nets in VCD output, not just those with public names

    -w
        writeback mode: use final simulation state as new init state

    -d
        enable debug output
\end{lstlisting}

\section{simplemap -- mapping simple coarse-grain cells}
\label{cmd:simplemap}
\begin{lstlisting}[numbers=left,frame=single]
    simplemap [selection]

This pass maps a small selection of simple coarse-grain cells to yosys gate
primitives. The following internal cell types are mapped by this pass:

  $not, $pos, $and, $or, $xor, $xnor
  $reduce_and, $reduce_or, $reduce_xor, $reduce_xnor, $reduce_bool
  $logic_not, $logic_and, $logic_or, $mux, $tribuf
  $sr, $ff, $dff, $dffsr, $adff, $dlatch
\end{lstlisting}

\section{splice -- create explicit splicing cells}
\label{cmd:splice}
\begin{lstlisting}[numbers=left,frame=single]
    splice [options] [selection]

This command adds $slice and $concat cells to the design to make the splicing
of multi-bit signals explicit. This for example is useful for coarse grain
synthesis, where dedicated hardware is needed to splice signals.

    -sel_by_cell
        only select the cell ports to rewire by the cell. if the selection
        contains a cell, than all cell inputs are rewired, if necessary.

    -sel_by_wire
        only select the cell ports to rewire by the wire. if the selection
        contains a wire, than all cell ports driven by this wire are wired,
        if necessary.

    -sel_any_bit
        it is sufficient if the driver of any bit of a cell port is selected.
        by default all bits must be selected.

    -wires
        also add $slice and $concat cells to drive otherwise unused wires.

    -no_outputs
        do not rewire selected module outputs.

    -port <name>
        only rewire cell ports with the specified name. can be used multiple
        times. implies -no_output.

    -no_port <name>
        do not rewire cell ports with the specified name. can be used multiple
        times. can not be combined with -port <name>.

By default selected output wires and all cell ports of selected cells driven
by selected wires are rewired.
\end{lstlisting}

\section{splitnets -- split up multi-bit nets}
\label{cmd:splitnets}
\begin{lstlisting}[numbers=left,frame=single]
    splitnets [options] [selection]

This command splits multi-bit nets into single-bit nets.

    -format char1[char2[char3]]
        the first char is inserted between the net name and the bit index, the
        second char is appended to the netname. e.g. -format () creates net
        names like 'mysignal(42)'. the 3rd character is the range separation
        character when creating multi-bit wires. the default is '[]:'.

    -ports
        also split module ports. per default only internal signals are split.

    -driver
        don't blindly split nets in individual bits. instead look at the driver
        and split nets so that no driver drives only part of a net.
\end{lstlisting}

\section{stat -- print some statistics}
\label{cmd:stat}
\begin{lstlisting}[numbers=left,frame=single]
    stat [options] [selection]

Print some statistics (number of objects) on the selected portion of the
design.

    -top <module>
        print design hierarchy with this module as top. if the design is fully
        selected and a module has the 'top' attribute set, this module is used
        default value for this option.

    -liberty <liberty_file>
        use cell area information from the provided liberty file

    -tech <technology>
        print area estemate for the specified technology. Currently supported
        values for <technology>: xilinx, cmos

    -width
        annotate internal cell types with their word width.
        e.g. $add_8 for an 8 bit wide $add cell.
\end{lstlisting}

\section{submod -- moving part of a module to a new submodule}
\label{cmd:submod}
\begin{lstlisting}[numbers=left,frame=single]
    submod [options] [selection]

This pass identifies all cells with the 'submod' attribute and moves them to
a newly created module. The value of the attribute is used as name for the
cell that replaces the group of cells with the same attribute value.

This pass can be used to create a design hierarchy in flat design. This can
be useful for analyzing or reverse-engineering a design.

This pass only operates on completely selected modules with no processes
or memories.

    -copy
        by default the cells are 'moved' from the source module and the source
        module will use an instance of the new module after this command is
        finished. call with -copy to not modify the source module.

    -name <name>
        don't use the 'submod' attribute but instead use the selection. only
        objects from one module might be selected. the value of the -name option
        is used as the value of the 'submod' attribute instead.

    -hidden
        instead of creating submodule ports with public names, create ports with
        private names so that a subsequent 'flatten; clean' call will restore the
        original module with original public names.
\end{lstlisting}

\section{supercover -- add hi/lo cover cells for each wire bit}
\label{cmd:supercover}
\begin{lstlisting}[numbers=left,frame=single]
    supercover [options] [selection]

This command adds two cover cells for each bit of each selected wire, one
checking for a hi signal level and one checking for lo level.
\end{lstlisting}

\section{synth -- generic synthesis script}
\label{cmd:synth}
\begin{lstlisting}[numbers=left,frame=single]
    synth [options]

This command runs the default synthesis script. This command does not operate
on partly selected designs.

    -top <module>
        use the specified module as top module (default='top')

    -auto-top
        automatically determine the top of the design hierarchy

    -flatten
        flatten the design before synthesis. this will pass '-auto-top' to
        'hierarchy' if no top module is specified.

    -encfile <file>
        passed to 'fsm_recode' via 'fsm'

    -lut <k>
        perform synthesis for a k-LUT architecture.

    -nofsm
        do not run FSM optimization

    -noabc
        do not run abc (as if yosys was compiled without ABC support)

    -noalumacc
        do not run 'alumacc' pass. i.e. keep arithmetic operators in
        their direct form ($add, $sub, etc.).

    -nordff
        passed to 'memory'. prohibits merging of FFs into memory read ports

    -noshare
        do not run SAT-based resource sharing

    -run <from_label>[:<to_label>]
        only run the commands between the labels (see below). an empty
        from label is synonymous to 'begin', and empty to label is
        synonymous to the end of the command list.

    -abc9
        use new ABC9 flow (EXPERIMENTAL)

    -flowmap
        use FlowMap LUT techmapping instead of ABC


The following commands are executed by this synthesis command:

    begin:
        hierarchy -check [-top <top> | -auto-top]

    coarse:
        proc
        flatten      (if -flatten)
        opt_expr
        opt_clean
        check
        opt
        wreduce
        peepopt
        opt_clean
        techmap -map +/cmp2lut.v -map +/cmp2lcu.v     (if -lut)
        alumacc      (unless -noalumacc)
        share        (unless -noshare)
        opt
        fsm          (unless -nofsm)
        opt -fast
        memory -nomap
        opt_clean

    fine:
        opt -fast -full
        memory_map
        opt -full
        techmap
        techmap -map +/gate2lut.v    (if -noabc and -lut)
        clean; opt_lut               (if -noabc and -lut)
        flowmap -maxlut K            (if -flowmap and -lut)
        opt -fast
        abc -fast           (unless -noabc, unless -lut)
        abc -fast -lut k    (unless -noabc, if -lut)
        opt -fast           (unless -noabc)

    check:
        hierarchy -check
        stat
        check
\end{lstlisting}

\section{synth\_achronix -- synthesis for Acrhonix Speedster22i FPGAs.}
\label{cmd:synth_achronix}
\begin{lstlisting}[numbers=left,frame=single]
    synth_achronix [options]

This command runs synthesis for Achronix Speedster eFPGAs. This work is still experimental.

    -top <module>
        use the specified module as top module (default='top')

    -vout <file>
        write the design to the specified Verilog netlist file. writing of an
        output file is omitted if this parameter is not specified.

    -run <from_label>:<to_label>
        only run the commands between the labels (see below). an empty
        from label is synonymous to 'begin', and empty to label is
        synonymous to the end of the command list.

    -noflatten
        do not flatten design before synthesis

    -retime
        run 'abc' with '-dff -D 1' options


The following commands are executed by this synthesis command:

    begin:
        read_verilog -sv -lib +/achronix/speedster22i/cells_sim.v
        hierarchy -check -top <top>

    flatten:    (unless -noflatten)
        proc
        flatten
        tribuf -logic
        deminout

    coarse:
        synth -run coarse

    fine:
        opt -fast -mux_undef -undriven -fine -full
        memory_map
        opt -undriven -fine
        dff2dffe -direct-match $_DFF_*
        opt -fine
        techmap -map +/techmap.v
        opt -full
        clean -purge
        setundef -undriven -zero
        abc -markgroups -dff -D 1    (only if -retime)

    map_luts:
        abc -lut 4
        clean

    map_cells:
        iopadmap -bits -outpad $__outpad I:O -inpad $__inpad O:I
        techmap -map +/achronix/speedster22i/cells_map.v
        clean -purge

    check:
        hierarchy -check
        stat
        check -noinit

    vout:
        write_verilog -nodec -attr2comment -defparam -renameprefix syn_ <file-name>
\end{lstlisting}

\section{synth\_anlogic -- synthesis for Anlogic FPGAs}
\label{cmd:synth_anlogic}
\begin{lstlisting}[numbers=left,frame=single]
    synth_anlogic [options]

This command runs synthesis for Anlogic FPGAs.

    -top <module>
        use the specified module as top module

    -edif <file>
        write the design to the specified EDIF file. writing of an output file
        is omitted if this parameter is not specified.

    -json <file>
        write the design to the specified JSON file. writing of an output file
        is omitted if this parameter is not specified.

    -run <from_label>:<to_label>
        only run the commands between the labels (see below). an empty
        from label is synonymous to 'begin', and empty to label is
        synonymous to the end of the command list.

    -noflatten
        do not flatten design before synthesis

    -retime
        run 'abc' with '-dff -D 1' options

    -nolutram
        do not use EG_LOGIC_DRAM16X4 cells in output netlist


The following commands are executed by this synthesis command:

    begin:
        read_verilog -lib +/anlogic/cells_sim.v +/anlogic/eagle_bb.v
        hierarchy -check -top <top>

    flatten:    (unless -noflatten)
        proc
        flatten
        tribuf -logic
        deminout

    coarse:
        synth -run coarse

    map_lutram:    (skip if -nolutram)
        memory_bram -rules +/anlogic/lutrams.txt
        techmap -map +/anlogic/lutrams_map.v
        setundef -zero -params t:EG_LOGIC_DRAM16X4

    map_ffram:
        opt -fast -mux_undef -undriven -fine
        memory_map
        opt -undriven -fine

    map_gates:
        techmap -map +/techmap.v -map +/anlogic/arith_map.v
        opt -fast
        abc -dff -D 1    (only if -retime)

    map_ffs:
        techmap -D NO_LUT -map +/anlogic/cells_map.v
        dffinit -strinit SET RESET -ff AL_MAP_SEQ q REGSET -noreinit
        opt_expr -mux_undef
        simplemap

    map_luts:
        abc -lut 4:6
        clean

    map_cells:
        techmap -map +/anlogic/cells_map.v
        clean

    map_anlogic:
        anlogic_fixcarry
        anlogic_eqn

    check:
        hierarchy -check
        stat
        check -noinit

    edif:
        write_edif <file-name>

    json:
        write_json <file-name>
\end{lstlisting}

\section{synth\_coolrunner2 -- synthesis for Xilinx Coolrunner-II CPLDs}
\label{cmd:synth_coolrunner2}
\begin{lstlisting}[numbers=left,frame=single]
    synth_coolrunner2 [options]

This command runs synthesis for Coolrunner-II CPLDs. This work is experimental.
It is intended to be used with https://github.com/azonenberg/openfpga as the
place-and-route.

    -top <module>
        use the specified module as top module (default='top')

    -json <file>
        write the design to the specified JSON file. writing of an output file
        is omitted if this parameter is not specified.

    -run <from_label>:<to_label>
        only run the commands between the labels (see below). an empty
        from label is synonymous to 'begin', and empty to label is
        synonymous to the end of the command list.

    -noflatten
        do not flatten design before synthesis

    -retime
        run 'abc' with '-dff -D 1' options


The following commands are executed by this synthesis command:

    begin:
        read_verilog -lib +/coolrunner2/cells_sim.v
        hierarchy -check -top <top>

    flatten:    (unless -noflatten)
        proc
        flatten
        tribuf -logic

    coarse:
        synth -run coarse

    fine:
        extract_counter -dir up -allow_arst no
        techmap -map +/coolrunner2/cells_counter_map.v
        clean
        opt -fast -full
        techmap -map +/techmap.v -map +/coolrunner2/cells_latch.v
        opt -fast
        dfflibmap -prepare -liberty +/coolrunner2/xc2_dff.lib

    map_tff:
        abc -g AND,XOR
        clean
        extract -map +/coolrunner2/tff_extract.v

    map_pla:
        abc -sop -I 40 -P 56
        clean

    map_cells:
        dfflibmap -liberty +/coolrunner2/xc2_dff.lib
        dffinit -ff FDCP Q INIT
        dffinit -ff FDCP_N Q INIT
        dffinit -ff FTCP Q INIT
        dffinit -ff FTCP_N Q INIT
        dffinit -ff LDCP Q INIT
        dffinit -ff LDCP_N Q INIT
        coolrunner2_sop
        clean
        iopadmap -bits -inpad IBUF O:I -outpad IOBUFE I:IO -inoutpad IOBUFE O:IO -toutpad IOBUFE E:I:IO -tinoutpad IOBUFE E:O:I:IO
        attrmvcp -attr src -attr LOC t:IOBUFE n:*
        attrmvcp -attr src -attr LOC -driven t:IBUF n:*
        coolrunner2_fixup
        splitnets
        clean

    check:
        hierarchy -check
        stat
        check -noinit

    json:
        write_json <file-name>
\end{lstlisting}

\section{synth\_easic -- synthesis for eASIC platform}
\label{cmd:synth_easic}
\begin{lstlisting}[numbers=left,frame=single]
    synth_easic [options]

This command runs synthesis for eASIC platform.

    -top <module>
        use the specified module as top module

    -vlog <file>
        write the design to the specified structural Verilog file. writing of
        an output file is omitted if this parameter is not specified.

    -etools <path>
        set path to the eTools installation. (default=/opt/eTools)

    -run <from_label>:<to_label>
        only run the commands between the labels (see below). an empty
        from label is synonymous to 'begin', and empty to label is
        synonymous to the end of the command list.

    -noflatten
        do not flatten design before synthesis

    -retime
        run 'abc' with '-dff -D 1' options


The following commands are executed by this synthesis command:

    begin:
        read_liberty -lib <etools_phys_clk_lib>
        read_liberty -lib <etools_logic_lut_lib>
        hierarchy -check -top <top>

    flatten:    (unless -noflatten)
        proc
        flatten

    coarse:
        synth -run coarse

    fine:
        opt -fast -mux_undef -undriven -fine
        memory_map
        opt -undriven -fine
        techmap
        opt -fast
        abc -dff -D 1     (only if -retime)
        opt_clean    (only if -retime)

    map:
        dfflibmap -liberty <etools_phys_clk_lib>
        abc -liberty <etools_logic_lut_lib>
        opt_clean

    check:
        hierarchy -check
        stat
        check -noinit

    vlog:
        write_verilog -noexpr -attr2comment <file-name>
\end{lstlisting}

\section{synth\_ecp5 -- synthesis for ECP5 FPGAs}
\label{cmd:synth_ecp5}
\begin{lstlisting}[numbers=left,frame=single]
    synth_ecp5 [options]

This command runs synthesis for ECP5 FPGAs.

    -top <module>
        use the specified module as top module

    -blif <file>
        write the design to the specified BLIF file. writing of an output file
        is omitted if this parameter is not specified.

    -edif <file>
        write the design to the specified EDIF file. writing of an output file
        is omitted if this parameter is not specified.

    -json <file>
        write the design to the specified JSON file. writing of an output file
        is omitted if this parameter is not specified.

    -run <from_label>:<to_label>
        only run the commands between the labels (see below). an empty
        from label is synonymous to 'begin', and empty to label is
        synonymous to the end of the command list.

    -noflatten
        do not flatten design before synthesis

    -retime
        run 'abc' with '-dff -D 1' options

    -noccu2
        do not use CCU2 cells in output netlist

    -nodffe
        do not use flipflops with CE in output netlist

    -nobram
        do not use block RAM cells in output netlist

    -nolutram
        do not use LUT RAM cells in output netlist

    -nowidelut
        do not use PFU muxes to implement LUTs larger than LUT4s

    -asyncprld
        use async PRLD mode to implement DLATCH and DFFSR (EXPERIMENTAL)

    -abc2
        run two passes of 'abc' for slightly improved logic density

    -abc9
        use new ABC9 flow (EXPERIMENTAL)

    -vpr
        generate an output netlist (and BLIF file) suitable for VPR
        (this feature is experimental and incomplete)

    -nodsp
        do not map multipliers to MULT18X18D


The following commands are executed by this synthesis command:

    begin:
        read_verilog -lib -specify +/ecp5/cells_sim.v +/ecp5/cells_bb.v
        hierarchy -check -top <top>

    coarse:
        proc
        flatten
        tribuf -logic
        deminout
        opt_expr
        opt_clean
        check
        opt
        wreduce
        peepopt
        opt_clean
        share
        techmap -map +/cmp2lut.v -D LUT_WIDTH=4
        opt_expr
        opt_clean
        techmap -map +/mul2dsp.v -map +/ecp5/dsp_map.v -D DSP_A_MAXWIDTH=18 -D DSP_B_MAXWIDTH=18  -D DSP_A_MINWIDTH=2 -D DSP_B_MINWIDTH=2  -D DSP_NAME=$__MUL18X18    (unless -nodsp)
        chtype -set $mul t:$__soft_mul    (unless -nodsp)
        alumacc
        opt
        fsm
        opt -fast
        memory -nomap
        opt_clean

    map_bram:    (skip if -nobram)
        memory_bram -rules +/ecp5/brams.txt
        techmap -map +/ecp5/brams_map.v

    map_lutram:    (skip if -nolutram)
        memory_bram -rules +/ecp5/lutrams.txt
        techmap -map +/ecp5/lutrams_map.v

    map_ffram:
        opt -fast -mux_undef -undriven -fine
        memory_map -iattr -attr !ram_block -attr !rom_block -attr logic_block -attr syn_ramstyle=auto -attr syn_ramstyle=registers -attr syn_romstyle=auto -attr syn_romstyle=logic
        opt -undriven -fine

    map_gates:
        techmap -map +/techmap.v -map +/ecp5/arith_map.v
        opt -fast
        abc -dff -D 1    (only if -retime)

    map_ffs:
        dff2dffs
        opt_clean
        dff2dffe -direct-match $_DFF_* -direct-match $__DFFS_*
        techmap -D NO_LUT [-D ASYNC_PRLD] -map +/ecp5/cells_map.v
        opt_expr -undriven -mux_undef
        simplemap
        ecp5_ffinit
        ecp5_gsr
        attrmvcp -copy -attr syn_useioff
        opt_clean

    map_luts:
        abc          (only if -abc2)
        techmap -map +/ecp5/latches_map.v
        abc -lut 4:7 -dress
        clean

    map_cells:
        techmap -map +/ecp5/cells_map.v    (with -D NO_LUT in vpr mode)
        opt_lut_ins -tech ecp5
        clean

    check:
        autoname
        hierarchy -check
        stat
        check -noinit

    blif:
        opt_clean -purge                                     (vpr mode)
        write_blif -attr -cname -conn -param <file-name>     (vpr mode)
        write_blif -gates -attr -param <file-name>           (non-vpr mode)

    edif:
        write_edif <file-name>

    json:
        write_json <file-name>
\end{lstlisting}

\section{synth\_efinix -- synthesis for Efinix FPGAs}
\label{cmd:synth_efinix}
\begin{lstlisting}[numbers=left,frame=single]
    synth_efinix [options]

This command runs synthesis for Efinix FPGAs.

    -top <module>
        use the specified module as top module

    -edif <file>
        write the design to the specified EDIF file. writing of an output file
        is omitted if this parameter is not specified.

    -json <file>
        write the design to the specified JSON file. writing of an output file
        is omitted if this parameter is not specified.

    -run <from_label>:<to_label>
        only run the commands between the labels (see below). an empty
        from label is synonymous to 'begin', and empty to label is
        synonymous to the end of the command list.

    -noflatten
        do not flatten design before synthesis

    -retime
        run 'abc' with '-dff -D 1' options

    -nobram
        do not use EFX_RAM_5K cells in output netlist


The following commands are executed by this synthesis command:

    begin:
        read_verilog -lib +/efinix/cells_sim.v
        hierarchy -check -top <top>

    flatten:    (unless -noflatten)
        proc
        flatten
        tribuf -logic
        deminout

    coarse:
        synth -run coarse
        memory_bram -rules +/efinix/brams.txt
        techmap -map +/efinix/brams_map.v
        setundef -zero -params t:EFX_RAM_5K

    map_ffram:
        opt -fast -mux_undef -undriven -fine
        memory_map
        opt -undriven -fine

    map_gates:
        techmap -map +/techmap.v -map +/efinix/arith_map.v
        opt -fast
        abc -dff -D 1    (only if -retime)

    map_ffs:
        techmap -D NO_LUT -map +/efinix/cells_map.v
        dffinit -strinit SET RESET -ff AL_MAP_SEQ q REGSET -noreinit
        opt_expr -mux_undef
        simplemap

    map_luts:
        abc -lut 4
        clean

    map_cells:
        techmap -map +/efinix/cells_map.v
        clean

    map_gbuf:
        efinix_gbuf
        efinix_fixcarry
        clean

    check:
        hierarchy -check
        stat
        check -noinit

    edif:
        write_edif <file-name>

    json:
        write_json <file-name>
\end{lstlisting}

\section{synth\_gowin -- synthesis for Gowin FPGAs}
\label{cmd:synth_gowin}
\begin{lstlisting}[numbers=left,frame=single]
    synth_gowin [options]

This command runs synthesis for Gowin FPGAs. This work is experimental.

    -top <module>
        use the specified module as top module (default='top')

    -vout <file>
        write the design to the specified Verilog netlist file. writing of an
        output file is omitted if this parameter is not specified.

    -run <from_label>:<to_label>
        only run the commands between the labels (see below). an empty
        from label is synonymous to 'begin', and empty to label is
        synonymous to the end of the command list.

    -nodffe
        do not use flipflops with CE in output netlist

    -nobram
        do not use BRAM cells in output netlist

    -nolutram
        do not use distributed RAM cells in output netlist

    -noflatten
        do not flatten design before synthesis

    -retime
        run 'abc' with '-dff -D 1' options

    -nowidelut
        do not use muxes to implement LUTs larger than LUT4s

    -noiopads
        do not emit IOB at top level ports


The following commands are executed by this synthesis command:

    begin:
        read_verilog -lib +/gowin/cells_sim.v
        hierarchy -check -top <top>

    flatten:    (unless -noflatten)
        proc
        flatten
        tribuf -logic
        deminout

    coarse:
        synth -run coarse

    map_bram:    (skip if -nobram)
        memory_bram -rules +/gowin/brams.txt
        techmap -map +/gowin/brams_map.v

    map_lutram:    (skip if -nolutram)
        memory_bram -rules +/gowin/lutrams.txt
        techmap -map +/gowin/lutrams_map.v
        determine_init

    map_ffram:
        opt -fast -mux_undef -undriven -fine
        memory_map
        opt -undriven -fine

    map_gates:
        techmap -map +/techmap.v -map +/gowin/arith_map.v
        opt -fast
        abc -dff -D 1    (only if -retime)
        splitnets

    map_ffs:
        dff2dffs -match-init
        opt_clean
        dff2dffe -direct-match $_DFF_* -direct-match $__DFFS_*
        techmap -map +/gowin/cells_map.v
        opt_expr -mux_undef
        simplemap

    map_luts:
        abc -lut 4:8
        clean

    map_cells:
        techmap -map +/gowin/cells_map.v
        opt_lut_ins -tech gowin
        setundef -undriven -params -zero
        hilomap -singleton -hicell VCC V -locell GND G
        iopadmap -bits -inpad IBUF O:I -outpad OBUF I:O -toutpad TBUF OEN:I:O -tinoutpad IOBUF OEN:O:I:IO    (unless -noiopads)
        clean

    check:
        hierarchy -check
        stat
        check -noinit

    vout:
        write_verilog -decimal -attr2comment -defparam -renameprefix gen <file-name>
\end{lstlisting}

\section{synth\_greenpak4 -- synthesis for GreenPAK4 FPGAs}
\label{cmd:synth_greenpak4}
\begin{lstlisting}[numbers=left,frame=single]
    synth_greenpak4 [options]

This command runs synthesis for GreenPAK4 FPGAs. This work is experimental.
It is intended to be used with https://github.com/azonenberg/openfpga as the
place-and-route.

    -top <module>
        use the specified module as top module (default='top')

    -part <part>
        synthesize for the specified part. Valid values are SLG46140V,
        SLG46620V, and SLG46621V (default).

    -json <file>
        write the design to the specified JSON file. writing of an output file
        is omitted if this parameter is not specified.

    -run <from_label>:<to_label>
        only run the commands between the labels (see below). an empty
        from label is synonymous to 'begin', and empty to label is
        synonymous to the end of the command list.

    -noflatten
        do not flatten design before synthesis

    -retime
        run 'abc' with '-dff -D 1' options


The following commands are executed by this synthesis command:

    begin:
        read_verilog -lib +/greenpak4/cells_sim.v
        hierarchy -check -top <top>

    flatten:    (unless -noflatten)
        proc
        flatten
        tribuf -logic

    coarse:
        synth -run coarse

    fine:
        extract_counter -pout GP_DCMP,GP_DAC -maxwidth 14
        clean
        opt -fast -mux_undef -undriven -fine
        memory_map
        opt -undriven -fine
        techmap -map +/techmap.v -map +/greenpak4/cells_latch.v
        dfflibmap -prepare -liberty +/greenpak4/gp_dff.lib
        opt -fast
        abc -dff -D 1    (only if -retime)

    map_luts:
        nlutmap -assert -luts 0,6,8,2     (for -part SLG46140V)
        nlutmap -assert -luts 2,8,16,2    (for -part SLG46620V)
        nlutmap -assert -luts 2,8,16,2    (for -part SLG46621V)
        clean

    map_cells:
        shregmap -tech greenpak4
        dfflibmap -liberty +/greenpak4/gp_dff.lib
        dffinit -ff GP_DFF Q INIT
        dffinit -ff GP_DFFR Q INIT
        dffinit -ff GP_DFFS Q INIT
        dffinit -ff GP_DFFSR Q INIT
        iopadmap -bits -inpad GP_IBUF OUT:IN -outpad GP_OBUF IN:OUT -inoutpad GP_OBUF OUT:IN -toutpad GP_OBUFT OE:IN:OUT -tinoutpad GP_IOBUF OE:OUT:IN:IO
        attrmvcp -attr src -attr LOC t:GP_OBUF t:GP_OBUFT t:GP_IOBUF n:*
        attrmvcp -attr src -attr LOC -driven t:GP_IBUF n:*
        techmap -map +/greenpak4/cells_map.v
        greenpak4_dffinv
        clean

    check:
        hierarchy -check
        stat
        check -noinit

    json:
        write_json <file-name>
\end{lstlisting}

\section{synth\_ice40 -- synthesis for iCE40 FPGAs}
\label{cmd:synth_ice40}
\begin{lstlisting}[numbers=left,frame=single]
    synth_ice40 [options]

This command runs synthesis for iCE40 FPGAs.

    -device < hx | lp | u >
        relevant only for '-abc9' flow, optimise timing for the specified device.
        default: hx

    -top <module>
        use the specified module as top module

    -blif <file>
        write the design to the specified BLIF file. writing of an output file
        is omitted if this parameter is not specified.

    -edif <file>
        write the design to the specified EDIF file. writing of an output file
        is omitted if this parameter is not specified.

    -json <file>
        write the design to the specified JSON file. writing of an output file
        is omitted if this parameter is not specified.

    -run <from_label>:<to_label>
        only run the commands between the labels (see below). an empty
        from label is synonymous to 'begin', and empty to label is
        synonymous to the end of the command list.

    -noflatten
        do not flatten design before synthesis

    -retime
        run 'abc' with '-dff -D 1' options

    -nocarry
        do not use SB_CARRY cells in output netlist

    -nodffe
        do not use SB_DFFE* cells in output netlist

    -dffe_min_ce_use <min_ce_use>
        do not use SB_DFFE* cells if the resulting CE line would go to less
        than min_ce_use SB_DFFE* in output netlist

    -nobram
        do not use SB_RAM40_4K* cells in output netlist

    -dsp
        use iCE40 UltraPlus DSP cells for large arithmetic

    -noabc
        use built-in Yosys LUT techmapping instead of abc

    -abc2
        run two passes of 'abc' for slightly improved logic density

    -vpr
        generate an output netlist (and BLIF file) suitable for VPR
        (this feature is experimental and incomplete)

    -abc9
        use new ABC9 flow (EXPERIMENTAL)

    -flowmap
        use FlowMap LUT techmapping instead of abc (EXPERIMENTAL)


The following commands are executed by this synthesis command:

    begin:
        read_verilog -D ICE40_HX -lib -specify +/ice40/cells_sim.v
        hierarchy -check -top <top>
        proc

    flatten:    (unless -noflatten)
        flatten
        tribuf -logic
        deminout

    coarse:
        opt_expr
        opt_clean
        check
        opt
        wreduce
        peepopt
        opt_clean
        share
        techmap -map +/cmp2lut.v -D LUT_WIDTH=4
        opt_expr
        opt_clean
        memory_dff
        wreduce t:$mul
        techmap -map +/mul2dsp.v -map +/ice40/dsp_map.v -D DSP_A_MAXWIDTH=16 -D DSP_B_MAXWIDTH=16 -D DSP_A_MINWIDTH=2 -D DSP_B_MINWIDTH=2 -D DSP_Y_MINWIDTH=11 -D DSP_NAME=$__MUL16X16    (if -dsp)
        select a:mul2dsp                  (if -dsp)
        setattr -unset mul2dsp            (if -dsp)
        opt_expr -fine                    (if -dsp)
        wreduce                           (if -dsp)
        select -clear                     (if -dsp)
        ice40_dsp                         (if -dsp)
        chtype -set $mul t:$__soft_mul    (if -dsp)
        alumacc
        opt
        fsm
        opt -fast
        memory -nomap
        opt_clean

    map_bram:    (skip if -nobram)
        memory_bram -rules +/ice40/brams.txt
        techmap -map +/ice40/brams_map.v
        ice40_braminit

    map_ffram:
        opt -fast -mux_undef -undriven -fine
        memory_map -iattr -attr !ram_block -attr !rom_block -attr logic_block -attr syn_ramstyle=auto -attr syn_ramstyle=registers -attr syn_romstyle=auto -attr syn_romstyle=logic
        opt -undriven -fine

    map_gates:
        ice40_wrapcarry
        techmap -map +/techmap.v -map +/ice40/arith_map.v
        opt -fast
        abc -dff -D 1    (only if -retime)
        ice40_opt

    map_ffs:
        dff2dffe -direct-match $_DFF_*
        techmap -D NO_LUT -D NO_ADDER -map +/ice40/cells_map.v
        opt_expr -mux_undef
        simplemap
        ice40_ffinit
        ice40_ffssr
        ice40_opt -full

    map_luts:
        abc          (only if -abc2)
        ice40_opt    (only if -abc2)
        techmap -map +/ice40/latches_map.v
        simplemap                                   (if -noabc or -flowmap)
        techmap -map +/gate2lut.v -D LUT_WIDTH=4    (only if -noabc)
        flowmap -maxlut 4    (only if -flowmap)
        abc -dress -lut 4    (skip if -noabc)
        ice40_wrapcarry -unwrap
        techmap -D NO_LUT -map +/ice40/cells_map.v
        clean
        opt_lut -dlogic SB_CARRY:I0=2:I1=1:CI=0

    map_cells:
        techmap -map +/ice40/cells_map.v    (with -D NO_LUT in vpr mode)
        clean

    check:
        autoname
        hierarchy -check
        stat
        check -noinit

    blif:
        opt_clean -purge                                     (vpr mode)
        write_blif -attr -cname -conn -param <file-name>     (vpr mode)
        write_blif -gates -attr -param <file-name>           (non-vpr mode)

    edif:
        write_edif <file-name>

    json:
        write_json <file-name>
\end{lstlisting}

\section{synth\_intel -- synthesis for Intel (Altera) FPGAs.}
\label{cmd:synth_intel}
\begin{lstlisting}[numbers=left,frame=single]
    synth_intel [options]

This command runs synthesis for Intel FPGAs.

    -family <max10 | arria10gx | cyclone10lp | cyclonev | cycloneiv | cycloneive>
        generate the synthesis netlist for the specified family.
        MAX10 is the default target if no family argument specified.
        For Cyclone IV GX devices, use cycloneiv argument; for Cyclone IV E, use cycloneive.
        Cyclone V and Arria 10 GX devices are experimental.

    -top <module>
        use the specified module as top module (default='top')

    -vqm <file>
        write the design to the specified Verilog Quartus Mapping File. Writing of an
        output file is omitted if this parameter is not specified.
        Note that this backend has not been tested and is likely incompatible
        with recent versions of Quartus.

    -vpr <file>
        write BLIF files for VPR flow experiments. The synthesized BLIF output file is not
        compatible with the Quartus flow. Writing of an
        output file is omitted if this parameter is not specified.

    -run <from_label>:<to_label>
        only run the commands between the labels (see below). an empty
        from label is synonymous to 'begin', and empty to label is
        synonymous to the end of the command list.

    -iopads
        use IO pad cells in output netlist

    -nobram
        do not use block RAM cells in output netlist

    -noflatten
        do not flatten design before synthesis

    -retime
        run 'abc' with '-dff -D 1' options

The following commands are executed by this synthesis command:

    begin:

    family:
        read_verilog -sv -lib +/intel/max10/cells_sim.v
        read_verilog -sv -lib +/intel/common/m9k_bb.v
        read_verilog -sv -lib +/intel/common/altpll_bb.v
        hierarchy -check -top <top>

    flatten:    (unless -noflatten)
        proc
        flatten
        tribuf -logic
        deminout

    coarse:
        synth -run coarse

    map_bram:    (skip if -nobram)
        memory_bram -rules +/intel/common/brams_m9k.txt    (if applicable for family)
        techmap -map +/intel/common/brams_map_m9k.v    (if applicable for family)

    map_ffram:
        opt -fast -mux_undef -undriven -fine -full
        memory_map
        opt -undriven -fine
        dff2dffe -direct-match $_DFF_*
        opt -fine
        techmap -map +/techmap.v
        opt -full
        clean -purge
        setundef -undriven -zero
        abc -markgroups -dff -D 1    (only if -retime)

    map_luts:
        abc -lut 4
        clean

    map_cells:
        iopadmap -bits -outpad $__outpad I:O -inpad $__inpad O:I    (if -iopads)
        techmap -map +/intel/max10/cells_map.v
        dffinit -highlow -ff dffeas q power_up
        clean -purge

    check:
        hierarchy -check
        stat
        check -noinit

    vqm:
        write_verilog -attr2comment -defparam -nohex -decimal -renameprefix syn_ <file-name>

    vpr:
        opt_clean -purge
        write_blif <file-name>


WARNING: THE 'synth_intel' COMMAND IS EXPERIMENTAL.
\end{lstlisting}

\section{synth\_sf2 -- synthesis for SmartFusion2 and IGLOO2 FPGAs}
\label{cmd:synth_sf2}
\begin{lstlisting}[numbers=left,frame=single]
    synth_sf2 [options]

This command runs synthesis for SmartFusion2 and IGLOO2 FPGAs.

    -top <module>
        use the specified module as top module

    -edif <file>
        write the design to the specified EDIF file. writing of an output file
        is omitted if this parameter is not specified.

    -vlog <file>
        write the design to the specified Verilog file. writing of an output file
        is omitted if this parameter is not specified.

    -json <file>
        write the design to the specified JSON file. writing of an output file
        is omitted if this parameter is not specified.

    -run <from_label>:<to_label>
        only run the commands between the labels (see below). an empty
        from label is synonymous to 'begin', and empty to label is
        synonymous to the end of the command list.

    -noflatten
        do not flatten design before synthesis

    -noiobs
        run synthesis in "block mode", i.e. do not insert IO buffers

    -clkbuf
        insert direct PAD->global_net buffers

    -retime
        run 'abc' with '-dff -D 1' options


The following commands are executed by this synthesis command:

    begin:
        read_verilog -lib +/sf2/cells_sim.v
        hierarchy -check -top <top>

    flatten:    (unless -noflatten)
        proc
        flatten
        tribuf -logic
        deminout

    coarse:
        synth -run coarse

    fine:
        opt -fast -mux_undef -undriven -fine
        memory_map
        opt -undriven -fine
        techmap -map +/techmap.v -map +/sf2/arith_map.v
        opt -fast
        abc -dff -D 1    (only if -retime)

    map_ffs:
        techmap -D NO_LUT -map +/sf2/cells_map.v
        opt_expr -mux_undef
        simplemap

    map_luts:
        abc -lut 4
        clean

    map_cells:
        techmap -map +/sf2/cells_map.v
        clean

    map_iobs:
        sf2_iobs [-clkbuf]    (unless -noiobs)
        clean

    check:
        hierarchy -check
        stat
        check -noinit

    edif:
        write_edif -gndvccy <file-name>

    vlog:
        write_verilog <file-name>

    json:
        write_json <file-name>
\end{lstlisting}

\section{synth\_xilinx -- synthesis for Xilinx FPGAs}
\label{cmd:synth_xilinx}
\begin{lstlisting}[numbers=left,frame=single]
    synth_xilinx [options]

This command runs synthesis for Xilinx FPGAs. This command does not operate on
partly selected designs. At the moment this command creates netlists that are
compatible with 7-Series Xilinx devices.

    -top <module>
        use the specified module as top module

    -family <family>
        run synthesis for the specified Xilinx architecture
        generate the synthesis netlist for the specified family.
        supported values:
        - xcup: Ultrascale Plus
        - xcu: Ultrascale
        - xc7: Series 7 (default)
        - xc6s: Spartan 6
        - xc6v: Virtex 6
        - xc5v: Virtex 5 (EXPERIMENTAL)
        - xc4v: Virtex 4 (EXPERIMENTAL)
        - xc3sda: Spartan 3A DSP (EXPERIMENTAL)
        - xc3sa: Spartan 3A (EXPERIMENTAL)
        - xc3se: Spartan 3E (EXPERIMENTAL)
        - xc3s: Spartan 3 (EXPERIMENTAL)
        - xc2vp: Virtex 2 Pro (EXPERIMENTAL)
        - xc2v: Virtex 2 (EXPERIMENTAL)
        - xcve: Virtex E, Spartan 2E (EXPERIMENTAL)
        - xcv: Virtex, Spartan 2 (EXPERIMENTAL)

    -edif <file>
        write the design to the specified edif file. writing of an output file
        is omitted if this parameter is not specified.

    -blif <file>
        write the design to the specified BLIF file. writing of an output file
        is omitted if this parameter is not specified.

    -vpr
        generate an output netlist (and BLIF file) suitable for VPR
        (this feature is experimental and incomplete)

    -ise
        generate an output netlist suitable for ISE

    -nobram
        do not use block RAM cells in output netlist

    -nolutram
        do not use distributed RAM cells in output netlist

    -nosrl
        do not use distributed SRL cells in output netlist

    -nocarry
        do not use XORCY/MUXCY/CARRY4 cells in output netlist

    -nowidelut
        do not use MUXF[5-9] resources to implement LUTs larger than native for the target

    -nodsp
        do not use DSP48*s to implement multipliers and associated logic

    -noiopad
        disable I/O buffer insertion (useful for hierarchical or 
        out-of-context flows)

    -noclkbuf
        disable automatic clock buffer insertion

    -uram
        infer URAM288s for large memories (xcup only)

    -widemux <int>
        enable inference of hard multiplexer resources (MUXF[78]) for muxes at or
        above this number of inputs (minimum value 2, recommended value >= 5).
        default: 0 (no inference)

    -run <from_label>:<to_label>
        only run the commands between the labels (see below). an empty
        from label is synonymous to 'begin', and empty to label is
        synonymous to the end of the command list.

    -flatten
        flatten design before synthesis

    -dff
        run 'abc'/'abc9' with -dff option

    -retime
        run 'abc' with '-D 1' option to enable flip-flop retiming.
        implies -dff.

    -abc9
        use new ABC9 flow (EXPERIMENTAL)


The following commands are executed by this synthesis command:

    begin:
        read_verilog -lib -specify +/xilinx/cells_sim.v
        read_verilog -lib +/xilinx/cells_xtra.v
        hierarchy -check -auto-top

    prepare:
        proc
        flatten    (with '-flatten')
        tribuf -logic
        deminout
        opt_expr
        opt_clean
        check
        opt
        wreduce [-keepdc]    (option for '-widemux')
        peepopt
        opt_clean
        muxpack        ('-widemux' only)
        pmux2shiftx    (skip if '-nosrl' and '-widemux=0')
        clean          (skip if '-nosrl' and '-widemux=0')

    map_dsp:    (skip if '-nodsp')
        memory_dff
        techmap -map +/mul2dsp.v -map +/xilinx/{family}_dsp_map.v {options}
        select a:mul2dsp
        setattr -unset mul2dsp
        opt_expr -fine
        wreduce
        select -clear
        xilinx_dsp -family <family>
        chtype -set $mul t:$__soft_mul

    coarse:
        techmap -map +/cmp2lut.v -map +/cmp2lcu.v -D LUT_WIDTH=[46]
        alumacc
        share
        opt
        fsm
        opt -fast
        memory -nomap
        opt_clean

    map_uram:    (only if '-uram')
        memory_bram -rules +/xilinx/{family}_urams.txt
        techmap -map +/xilinx/{family}_urams_map.v

    map_bram:    (skip if '-nobram')
        memory_bram -rules +/xilinx/{family}_brams.txt
        techmap -map +/xilinx/{family}_brams_map.v

    map_lutram:    (skip if '-nolutram')
        memory_bram -rules +/xilinx/lut[46]_lutrams.txt
        techmap -map +/xilinx/lutrams_map.v

    map_ffram:
        simplemap t:$dff t:$adff t:$mux
        dff2dffs [-match-init]    (-match-init for xc6s only)
        opt -fast -full
        memory_map

    fine:
        dff2dffe -direct-match $_DFF_* -direct-match $__DFFS_*
        muxcover <internal options> ('-widemux' only)
        opt -full
        xilinx_srl -variable -minlen 3    (skip if '-nosrl')
        techmap  -map +/techmap.v -D LUT_SIZE=[46] [-map +/xilinx/mux_map.v] -map +/xilinx/arith_map.v
        opt -fast

    map_cells:
        iopadmap -bits -outpad OBUF I:O -inpad IBUF O:I -toutpad $__XILINX_TOUTPAD OE:I:O -tinoutpad $__XILINX_TINOUTPAD OE:O:I:IO A:top    (skip if '-noiopad')
        techmap -map +/techmap.v -map +/xilinx/cells_map.v
        clean

    map_ffs:
        techmap -map +/xilinx/{family}_ff_map.v    ('-abc9' only)

    map_luts:
        opt_expr -mux_undef
        abc -luts 2:2,3,6:5[,10,20] [-dff] [-D 1]    (option for 'nowidelut', '-dff', '-retime')
        clean
        xilinx_srl -fixed -minlen 3    (skip if '-nosrl')
        techmap -map +/xilinx/lut_map.v -map +/xilinx/cells_map.v -map +/xilinx/{family}_ff_map.v -D LUT_WIDTH=[46]
        xilinx_dffopt [-lut4]
        opt_lut_ins -tech xilinx

    finalize:
        clkbufmap -buf BUFG O:I    (skip if '-noclkbuf')
        extractinv -inv INV O:I    (only if '-ise')
        clean

    check:
        hierarchy -check
        stat -tech xilinx
        check -noinit

    edif:
        write_edif -pvector bra 

    blif:
        write_blif 
\end{lstlisting}

\section{tcl -- execute a TCL script file}
\label{cmd:tcl}
\begin{lstlisting}[numbers=left,frame=single]
    tcl <filename> [args]

This command executes the tcl commands in the specified file.
Use 'yosys cmd' to run the yosys command 'cmd' from tcl.

The tcl command 'yosys -import' can be used to import all yosys
commands directly as tcl commands to the tcl shell. Yosys commands
'proc' and 'rename' are wrapped to tcl commands 'procs' and 'renames'
in order to avoid a name collision with the built in commands.

If any arguments are specified, these arguments are provided to the script via
the standard $argc and $argv variables.
\end{lstlisting}

\section{techmap -- generic technology mapper}
\label{cmd:techmap}
\begin{lstlisting}[numbers=left,frame=single]
    techmap [-map filename] [selection]

This pass implements a very simple technology mapper that replaces cells in
the design with implementations given in form of a Verilog or ilang source
file.

    -map filename
        the library of cell implementations to be used.
        without this parameter a builtin library is used that
        transforms the internal RTL cells to the internal gate
        library.

    -map %<design-name>
        like -map above, but with an in-memory design instead of a file.

    -extern
        load the cell implementations as separate modules into the design
        instead of inlining them.

    -max_iter <number>
        only run the specified number of iterations on each module.
        default: unlimited

    -recursive
        instead of the iterative breadth-first algorithm use a recursive
        depth-first algorithm. both methods should yield equivalent results,
        but may differ in performance.

    -autoproc
        Automatically call "proc" on implementations that contain processes.

    -wb
        Ignore the 'whitebox' attribute on cell implementations.

    -assert
        this option will cause techmap to exit with an error if it can't map
        a selected cell. only cell types that end on an underscore are accepted
        as final cell types by this mode.

    -D <define>, -I <incdir>
        this options are passed as-is to the Verilog frontend for loading the
        map file. Note that the Verilog frontend is also called with the
        '-nooverwrite' option set.

When a module in the map file has the 'techmap_celltype' attribute set, it will
match cells with a type that match the text value of this attribute. Otherwise
the module name will be used to match the cell.

When a module in the map file has the 'techmap_simplemap' attribute set, techmap
will use 'simplemap' (see 'help simplemap') to map cells matching the module.

When a module in the map file has the 'techmap_maccmap' attribute set, techmap
will use 'maccmap' (see 'help maccmap') to map cells matching the module.

When a module in the map file has the 'techmap_wrap' attribute set, techmap
will create a wrapper for the cell and then run the command string that the
attribute is set to on the wrapper module.

When a port on a module in the map file has the 'techmap_autopurge' attribute
set, and that port is not connected in the instantiation that is mapped, then
then a cell port connected only to such wires will be omitted in the mapped
version of the circuit.

All wires in the modules from the map file matching the pattern _TECHMAP_*
or *._TECHMAP_* are special wires that are used to pass instructions from
the mapping module to the techmap command. At the moment the following special
wires are supported:

    _TECHMAP_FAIL_
        When this wire is set to a non-zero constant value, techmap will not
        use this module and instead try the next module with a matching
        'techmap_celltype' attribute.

        When such a wire exists but does not have a constant value after all
        _TECHMAP_DO_* commands have been executed, an error is generated.

    _TECHMAP_DO_*
        This wires are evaluated in alphabetical order. The constant text value
        of this wire is a yosys command (or sequence of commands) that is run
        by techmap on the module. A common use case is to run 'proc' on modules
        that are written using always-statements.

        When such a wire has a non-constant value at the time it is to be
        evaluated, an error is produced. That means it is possible for such a
        wire to start out as non-constant and evaluate to a constant value
        during processing of other _TECHMAP_DO_* commands.

        A _TECHMAP_DO_* command may start with the special token 'CONSTMAP; '.
        in this case techmap will create a copy for each distinct configuration
        of constant inputs and shorted inputs at this point and import the
        constant and connected bits into the map module. All further commands
        are executed in this copy. This is a very convenient way of creating
        optimized specializations of techmap modules without using the special
        parameters described below.

        A _TECHMAP_DO_* command may start with the special token 'RECURSION; '.
        then techmap will recursively replace the cells in the module with their
        implementation. This is not affected by the -max_iter option.

        It is possible to combine both prefixes to 'RECURSION; CONSTMAP; '.

    _TECHMAP_REMOVEINIT_<port-name>_
        When this wire is set to a constant value, the init attribute of the wire(s)
        connected to this port will be consumed.  This wire must have the same
        width as the given port, and for every bit that is set to 1 in the value,
        the corresponding init attribute bit will be changed to 1'bx.  If all
        bits of an init attribute are left as x, it will be removed.

In addition to this special wires, techmap also supports special parameters in
modules in the map file:

    _TECHMAP_CELLTYPE_
        When a parameter with this name exists, it will be set to the type name
        of the cell that matches the module.

    _TECHMAP_CONSTMSK_<port-name>_
    _TECHMAP_CONSTVAL_<port-name>_
        When this pair of parameters is available in a module for a port, then
        former has a 1-bit for each constant input bit and the latter has the
        value for this bit. The unused bits of the latter are set to undef (x).

    _TECHMAP_WIREINIT_<port-name>_
        When a parameter with this name exists, it will be set to the initial
        value of the wire(s) connected to the given port, as specified by the init
        attribute. If the attribute doesn't exist, x will be filled for the
        missing bits.  To remove the init attribute bits used, use the
        _TECHMAP_REMOVEINIT_*_ wires.

    _TECHMAP_BITS_CONNMAP_
    _TECHMAP_CONNMAP_<port-name>_
        For an N-bit port, the _TECHMAP_CONNMAP_<port-name>_ parameter, if it
        exists, will be set to an N*_TECHMAP_BITS_CONNMAP_ bit vector containing
        N words (of _TECHMAP_BITS_CONNMAP_ bits each) that assign each single
        bit driver a unique id. The values 0-3 are reserved for 0, 1, x, and z.
        This can be used to detect shorted inputs.

When a module in the map file has a parameter where the according cell in the
design has a port, the module from the map file is only used if the port in
the design is connected to a constant value. The parameter is then set to the
constant value.

A cell with the name _TECHMAP_REPLACE_ in the map file will inherit the name
and attributes of the cell that is being replaced.
A cell with a name of the form `_TECHMAP_REPLACE_.<suffix>` in the map file will
be named thus but with the `_TECHMAP_REPLACE_' prefix substituted with the name
of the cell being replaced.
Similarly, a wire named in the form `_TECHMAP_REPLACE_.<suffix>` will cause a
new wire alias to be created and named as above but with the `_TECHMAP_REPLACE_'
prefix also substituted.

See 'help extract' for a pass that does the opposite thing.

See 'help flatten' for a pass that does flatten the design (which is
essentially techmap but using the design itself as map library).
\end{lstlisting}

\section{tee -- redirect command output to file}
\label{cmd:tee}
\begin{lstlisting}[numbers=left,frame=single]
    tee [-q] [-o logfile|-a logfile] cmd

Execute the specified command, optionally writing the commands output to the
specified logfile(s).

    -q
        Do not print output to the normal destination (console and/or log file).

    -o logfile
        Write output to this file, truncate if exists.

    -a logfile
        Write output to this file, append if exists.

    +INT, -INT
        Add/subtract INT from the -v setting for this command.
\end{lstlisting}

\section{test\_abcloop -- automatically test handling of loops in abc command}
\label{cmd:test_abcloop}
\begin{lstlisting}[numbers=left,frame=single]
    test_abcloop [options]

Test handling of logic loops in ABC.

    -n {integer}
        create this number of circuits and test them (default = 100).

    -s {positive_integer}
        use this value as rng seed value (default = unix time).
\end{lstlisting}

\section{test\_autotb -- generate simple test benches}
\label{cmd:test_autotb}
\begin{lstlisting}[numbers=left,frame=single]
    test_autotb [options] [filename]

Automatically create primitive Verilog test benches for all modules in the
design. The generated testbenches toggle the input pins of the module in
a semi-random manner and dumps the resulting output signals.

This can be used to check the synthesis results for simple circuits by
comparing the testbench output for the input files and the synthesis results.

The backend automatically detects clock signals. Additionally a signal can
be forced to be interpreted as clock signal by setting the attribute
'gentb_clock' on the signal.

The attribute 'gentb_constant' can be used to force a signal to a constant
value after initialization. This can e.g. be used to force a reset signal
low in order to explore more inner states in a state machine.

The attribute 'gentb_skip' can be attached to modules to suppress testbench
generation.

    -n <int>
        number of iterations the test bench should run (default = 1000)

    -seed <int>
        seed used for pseudo-random number generation (default = 0).
        a value of 0 will cause an arbitrary seed to be chosen, based on
        the current system time.
\end{lstlisting}

\section{test\_cell -- automatically test the implementation of a cell type}
\label{cmd:test_cell}
\begin{lstlisting}[numbers=left,frame=single]
    test_cell [options] {cell-types}

Tests the internal implementation of the given cell type (for example '$add')
by comparing SAT solver, EVAL and TECHMAP implementations of the cell types..

Run with 'all' instead of a cell type to run the test on all supported
cell types. Use for example 'all /$add' for all cell types except $add.

    -n {integer}
        create this number of cell instances and test them (default = 100).

    -s {positive_integer}
        use this value as rng seed value (default = unix time).

    -f {ilang_file}
        don't generate circuits. instead load the specified ilang file.

    -w {filename_prefix}
        don't test anything. just generate the circuits and write them
        to ilang files with the specified prefix

    -map {filename}
        pass this option to techmap.

    -simlib
        use "techmap -D SIMLIB_NOCHECKS -map +/simlib.v -max_iter 2 -autoproc"

    -aigmap
        instead of calling "techmap", call "aigmap"

    -muxdiv
        when creating test benches with dividers, create an additional mux
        to mask out the division-by-zero case

    -script {script_file}
        instead of calling "techmap", call "script {script_file}".

    -const
        set some input bits to random constant values

    -nosat
        do not check SAT model or run SAT equivalence checking

    -noeval
        do not check const-eval models

    -edges
        test cell edges db creator against sat-based implementation

    -v
        print additional debug information to the console

    -vlog {filename}
        create a Verilog test bench to test simlib and write_verilog
\end{lstlisting}

\section{test\_pmgen -- test pass for pmgen}
\label{cmd:test_pmgen}
\begin{lstlisting}[numbers=left,frame=single]
    test_pmgen -reduce_chain [options] [selection]

Demo for recursive pmgen patterns. Map chains of AND/OR/XOR to $reduce_*.


    test_pmgen -reduce_tree [options] [selection]

Demo for recursive pmgen patterns. Map trees of AND/OR/XOR to $reduce_*.


    test_pmgen -eqpmux [options] [selection]

Demo for recursive pmgen patterns. Optimize EQ/NE/PMUX circuits.


    test_pmgen -generate [options] <pattern_name>

Create modules that match the specified pattern.
\end{lstlisting}

\section{torder -- print cells in topological order}
\label{cmd:torder}
\begin{lstlisting}[numbers=left,frame=single]
    torder [options] [selection]

This command prints the selected cells in topological order.

    -stop <cell_type> <cell_port>
        do not use the specified cell port in topological sorting

    -noautostop
        by default Q outputs of internal FF cells and memory read port outputs
        are not used in topological sorting. this option deactivates that.
\end{lstlisting}

\section{trace -- redirect command output to file}
\label{cmd:trace}
\begin{lstlisting}[numbers=left,frame=single]
    trace cmd

Execute the specified command, logging all changes the command performs on
the design in real time.
\end{lstlisting}

\section{tribuf -- infer tri-state buffers}
\label{cmd:tribuf}
\begin{lstlisting}[numbers=left,frame=single]
    tribuf [options] [selection]

This pass transforms $mux cells with 'z' inputs to tristate buffers.

    -merge
        merge multiple tri-state buffers driving the same net
        into a single buffer.

    -logic
        convert tri-state buffers that do not drive output ports
        to non-tristate logic. this option implies -merge.
\end{lstlisting}

\section{uniquify -- create unique copies of modules}
\label{cmd:uniquify}
\begin{lstlisting}[numbers=left,frame=single]
    uniquify [selection]

By default, a module that is instantiated by several other modules is only
kept once in the design. This preserves the original modularity of the design
and reduces the overall size of the design in memory. But it prevents certain
optimizations and other operations on the design. This pass creates unique
modules for all selected cells. The created modules are marked with the
'unique' attribute.

This commands only operates on modules that by themself have the 'unique'
attribute set (the 'top' module is unique implicitly).
\end{lstlisting}

\section{verific -- load Verilog and VHDL designs using Verific}
\label{cmd:verific}
\begin{lstlisting}[numbers=left,frame=single]
    verific {-vlog95|-vlog2k|-sv2005|-sv2009|-sv2012|-sv} <verilog-file>..

Load the specified Verilog/SystemVerilog files into Verific.

All files specified in one call to this command are one compilation unit.
Files passed to different calls to this command are treated as belonging to
different compilation units.

Additional -D<macro>[=<value>] options may be added after the option indicating
the language version (and before file names) to set additional verilog defines.
The macros SYNTHESIS and VERIFIC are defined implicitly.


    verific -formal <verilog-file>..

Like -sv, but define FORMAL instead of SYNTHESIS.


    verific {-vhdl87|-vhdl93|-vhdl2k|-vhdl2008|-vhdl} <vhdl-file>..

Load the specified VHDL files into Verific.


    verific [-work <libname>] {-sv|-vhdl|...} <hdl-file>

Load the specified Verilog/SystemVerilog/VHDL file into the specified library.
(default library when -work is not present: "work")


    verific [-L <libname>] {-sv|-vhdl|...} <hdl-file>

Look up external definitions in the specified library.
(-L may be used more than once)


    verific -vlog-incdir <directory>..

Add Verilog include directories.


    verific -vlog-libdir <directory>..

Add Verilog library directories. Verific will search in this directories to
find undefined modules.


    verific -vlog-define <macro>[=<value>]..

Add Verilog defines.


    verific -vlog-undef <macro>..

Remove Verilog defines previously set with -vlog-define.


    verific -set-error <msg_id>..
    verific -set-warning <msg_id>..
    verific -set-info <msg_id>..
    verific -set-ignore <msg_id>..

Set message severity. <msg_id> is the string in square brackets when a message
is printed, such as VERI-1209.


    verific -import [options] <top-module>..

Elaborate the design for the specified top modules, import to Yosys and
reset the internal state of Verific.

Import options:

  -all
    Elaborate all modules, not just the hierarchy below the given top
    modules. With this option the list of modules to import is optional.

  -gates
    Create a gate-level netlist.

  -flatten
    Flatten the design in Verific before importing.

  -extnets
    Resolve references to external nets by adding module ports as needed.

  -autocover
    Generate automatic cover statements for all asserts

  -fullinit
    Keep all register initializations, even those for non-FF registers.

  -chparam name value 
    Elaborate the specified top modules (all modules when -all given) using
    this parameter value. Modules on which this parameter does not exist will
    cause Verific to produce a VERI-1928 or VHDL-1676 message. This option
    can be specified multiple times to override multiple parameters.
    String values must be passed in double quotes (").

  -v, -vv
    Verbose log messages. (-vv is even more verbose than -v.)

The following additional import options are useful for debugging the Verific
bindings (for Yosys and/or Verific developers):

  -k
    Keep going after an unsupported verific primitive is found. The
    unsupported primitive is added as blockbox module to the design.
    This will also add all SVA related cells to the design parallel to
    the checker logic inferred by it.

  -V
    Import Verific netlist as-is without translating to Yosys cell types. 

  -nosva
    Ignore SVA properties, do not infer checker logic.

  -L <int>
    Maximum number of ctrl bits for SVA checker FSMs (default=16).

  -n
    Keep all Verific names on instances and nets. By default only
    user-declared names are preserved.

  -d <dump_file>
    Dump the Verific netlist as a verilog file.


Use Symbiotic EDA Suite if you need Yosys+Verifc.
https://www.symbioticeda.com/seda-suite

Contact office@symbioticeda.com for free evaluation
binaries of Symbiotic EDA Suite.
\end{lstlisting}

\section{verilog\_defaults -- set default options for read\_verilog}
\label{cmd:verilog_defaults}
\begin{lstlisting}[numbers=left,frame=single]
    verilog_defaults -add [options]

Add the specified options to the list of default options to read_verilog.


    verilog_defaults -clear

Clear the list of Verilog default options.


    verilog_defaults -push
    verilog_defaults -pop

Push or pop the list of default options to a stack. Note that -push does
not imply -clear.
\end{lstlisting}

\section{verilog\_defines -- define and undefine verilog defines}
\label{cmd:verilog_defines}
\begin{lstlisting}[numbers=left,frame=single]
    verilog_defines [options]

Define and undefine verilog preprocessor macros.

    -Dname[=definition]
        define the preprocessor symbol 'name' and set its optional value
        'definition'

    -Uname[=definition]
        undefine the preprocessor symbol 'name'

    -reset
        clear list of defined preprocessor symbols

    -list
        list currently defined preprocessor symbols
\end{lstlisting}

\section{wbflip -- flip the whitebox attribute}
\label{cmd:wbflip}
\begin{lstlisting}[numbers=left,frame=single]
    wbflip [selection]

Flip the whitebox attribute on selected cells. I.e. if it's set, unset it, and
vice-versa. Blackbox cells are not effected by this command.
\end{lstlisting}

\section{wreduce -- reduce the word size of operations if possible}
\label{cmd:wreduce}
\begin{lstlisting}[numbers=left,frame=single]
    wreduce [options] [selection]

This command reduces the word size of operations. For example it will replace
the 32 bit adders in the following code with adders of more appropriate widths:

    module test(input [3:0] a, b, c, output [7:0] y);
        assign y = a + b + c + 1;
    endmodule

Options:

    -memx
        Do not change the width of memory address ports. Use this options in
        flows that use the 'memory_memx' pass.

    -keepdc
        Do not optimize explicit don't-care values.
\end{lstlisting}

\section{write\_aiger -- write design to AIGER file}
\label{cmd:write_aiger}
\begin{lstlisting}[numbers=left,frame=single]
    write_aiger [options] [filename]

Write the current design to an AIGER file. The design must be flattened and
must not contain any cell types except $_AND_, $_NOT_, simple FF types,
$assert and $assume cells, and $initstate cells.

$assert and $assume cells are converted to AIGER bad state properties and
invariant constraints.

    -ascii
        write ASCII version of AIGER format

    -zinit
        convert FFs to zero-initialized FFs, adding additional inputs for
        uninitialized FFs.

    -miter
        design outputs are AIGER bad state properties

    -symbols
        include a symbol table in the generated AIGER file

    -map <filename>
        write an extra file with port and latch symbols

    -vmap <filename>
        like -map, but more verbose

    -I, -O, -B, -L
        If the design contains no input/output/assert/flip-flop then create one
        dummy input/output/bad_state-pin or latch to make the tools reading the
        AIGER file happy.
\end{lstlisting}

\section{write\_blif -- write design to BLIF file}
\label{cmd:write_blif}
\begin{lstlisting}[numbers=left,frame=single]
    write_blif [options] [filename]

Write the current design to an BLIF file.

    -top top_module
        set the specified module as design top module

    -buf <cell-type> <in-port> <out-port>
        use cells of type <cell-type> with the specified port names for buffers

    -unbuf <cell-type> <in-port> <out-port>
        replace buffer cells with the specified name and port names with
        a .names statement that models a buffer

    -true <cell-type> <out-port>
    -false <cell-type> <out-port>
    -undef <cell-type> <out-port>
        use the specified cell types to drive nets that are constant 1, 0, or
        undefined. when '-' is used as <cell-type>, then <out-port> specifies
        the wire name to be used for the constant signal and no cell driving
        that wire is generated. when '+' is used as <cell-type>, then <out-port>
        specifies the wire name to be used for the constant signal and a .names
        statement is generated to drive the wire.

    -noalias
        if a net name is aliasing another net name, then by default a net
        without fanout is created that is driven by the other net. This option
        suppresses the generation of this nets without fanout.

The following options can be useful when the generated file is not going to be
read by a BLIF parser but a custom tool. It is recommended to not name the output
file *.blif when any of this options is used.

    -icells
        do not translate Yosys's internal gates to generic BLIF logic
        functions. Instead create .subckt or .gate lines for all cells.

    -gates
        print .gate instead of .subckt lines for all cells that are not
        instantiations of other modules from this design.

    -conn
        do not generate buffers for connected wires. instead use the
        non-standard .conn statement.

    -attr
        use the non-standard .attr statement to write cell attributes

    -param
        use the non-standard .param statement to write cell parameters

    -cname
        use the non-standard .cname statement to write cell names

    -iname, -iattr
        enable -cname and -attr functionality for .names statements
        (the .cname and .attr statements will be included in the BLIF
        output after the truth table for the .names statement)

    -blackbox
        write blackbox cells with .blackbox statement.

    -impltf
        do not write definitions for the $true, $false and $undef wires.
\end{lstlisting}

\section{write\_btor -- write design to BTOR file}
\label{cmd:write_btor}
\begin{lstlisting}[numbers=left,frame=single]
    write_btor [options] [filename]

Write a BTOR description of the current design.

  -v
    Add comments and indentation to BTOR output file

  -s
    Output only a single bad property for all asserts

  -c
    Output cover properties using 'bad' statements instead of asserts

  -i <filename>
    Create additional info file with auxiliary information
\end{lstlisting}

\section{write\_cxxrtl -- convert design to C++ RTL simulation}
\label{cmd:write_cxxrtl}
\begin{lstlisting}[numbers=left,frame=single]
    write_cxxrtl [options] [filename]

Write C++ code for simulating the design. The generated code requires a driver;
the following simple driver is provided as an example:

    #include "top.cc"

    int main() {
      cxxrtl_design::p_top top;
      while (1) {
        top.p_clk.next = value<1> {1u};
        top.step();
        top.p_clk.next = value<1> {0u};
        top.step();
      }
    }

The following options are supported by this backend:

    -O <level>
        set the optimization level. the default is -O5. higher optimization
        levels dramatically decrease compile and run time, and highest level
        possible for a design should be used.

    -O0
        no optimization.

    -O1
        elide internal wires if possible.

    -O2
        like -O1, and localize internal wires if possible.

    -O3
        like -O2, and elide public wires not marked (*keep*) if possible.

    -O4
        like -O3, and localize public wires not marked (*keep*) if possible.

    -O5
        like -O4, and run `splitnets -driver; opt_clean -purge` first.
\end{lstlisting}

\section{write\_edif -- write design to EDIF netlist file}
\label{cmd:write_edif}
\begin{lstlisting}[numbers=left,frame=single]
    write_edif [options] [filename]

Write the current design to an EDIF netlist file.

    -top top_module
        set the specified module as design top module

    -nogndvcc
        do not create "GND" and "VCC" cells. (this will produce an error
        if the design contains constant nets. use "hilomap" to map to custom
        constant drivers first)

    -gndvccy
        create "GND" and "VCC" cells with "Y" outputs. (the default is "G"
        for "GND" and "P" for "VCC".)

    -attrprop
        create EDIF properties for cell attributes

    -pvector {par|bra|ang}
        sets the delimiting character for module port rename clauses to
        parentheses, square brackets, or angle brackets.

Unfortunately there are different "flavors" of the EDIF file format. This
command generates EDIF files for the Xilinx place&route tools. It might be
necessary to make small modifications to this command when a different tool
is targeted.
\end{lstlisting}

\section{write\_file -- write a text to a file}
\label{cmd:write_file}
\begin{lstlisting}[numbers=left,frame=single]
    write_file [options] output_file [input_file]

Write the text from the input file to the output file.

    -a
        Append to output file (instead of overwriting)


Inside a script the input file can also can a here-document:

    write_file hello.txt <<EOT
    Hello World!
    EOT
\end{lstlisting}

\section{write\_firrtl -- write design to a FIRRTL file}
\label{cmd:write_firrtl}
\begin{lstlisting}[numbers=left,frame=single]
    write_firrtl [options] [filename]

Write a FIRRTL netlist of the current design.
The following commands are executed by this command:
        pmuxtree
\end{lstlisting}

\section{write\_ilang -- write design to ilang file}
\label{cmd:write_ilang}
\begin{lstlisting}[numbers=left,frame=single]
    write_ilang [filename]

Write the current design to an 'ilang' file. (ilang is a text representation
of a design in yosys's internal format.)

    -selected
        only write selected parts of the design.
\end{lstlisting}

\section{write\_intersynth -- write design to InterSynth netlist file}
\label{cmd:write_intersynth}
\begin{lstlisting}[numbers=left,frame=single]
    write_intersynth [options] [filename]

Write the current design to an 'intersynth' netlist file. InterSynth is
a tool for Coarse-Grain Example-Driven Interconnect Synthesis.

    -notypes
        do not generate celltypes and conntypes commands. i.e. just output
        the netlists. this is used for postsilicon synthesis.

    -lib <verilog_or_ilang_file>
        Use the specified library file for determining whether cell ports are
        inputs or outputs. This option can be used multiple times to specify
        more than one library.

    -selected
        only write selected modules. modules must be selected entirely or
        not at all.

http://www.clifford.at/intersynth/
\end{lstlisting}

\section{write\_json -- write design to a JSON file}
\label{cmd:write_json}
\begin{lstlisting}[numbers=left,frame=single]
    write_json [options] [filename]

Write a JSON netlist of the current design.

    -aig
        include AIG models for the different gate types

    -compat-int
        emit 32-bit or smaller fully-defined parameter values directly
        as JSON numbers (for compatibility with old parsers)


The general syntax of the JSON output created by this command is as follows:

    {
      "modules": {
        <module_name>: {
          "ports": {
            <port_name>: <port_details>,
            ...
          },
          "cells": {
            <cell_name>: <cell_details>,
            ...
          },
          "netnames": {
            <net_name>: <net_details>,
            ...
          }
        }
      },
      "models": {
        ...
      },
    }

Where <port_details> is:

    {
      "direction": <"input" | "output" | "inout">,
      "bits": <bit_vector>
    }

And <cell_details> is:

    {
      "hide_name": <1 | 0>,
      "type": <cell_type>,
      "parameters": {
        <parameter_name>: <parameter_value>,
        ...
      },
      "attributes": {
        <attribute_name>: <attribute_value>,
        ...
      },
      "port_directions": {
        <port_name>: <"input" | "output" | "inout">,
        ...
      },
      "connections": {
        <port_name>: <bit_vector>,
        ...
      },
    }

And <net_details> is:

    {
      "hide_name": <1 | 0>,
      "bits": <bit_vector>
    }

The "hide_name" fields are set to 1 when the name of this cell or net is
automatically created and is likely not of interest for a regular user.

The "port_directions" section is only included for cells for which the
interface is known.

Module and cell ports and nets can be single bit wide or vectors of multiple
bits. Each individual signal bit is assigned a unique integer. The <bit_vector>
values referenced above are vectors of this integers. Signal bits that are
connected to a constant driver are denoted as string "0", "1", "x", or
"z" instead of a number.

Bit vectors (including integers) are written as string holding the binaryrepresentation of the value. Strings are written as strings, with an appendedblank in cases of strings of the form /[01xz]* */.

For example the following Verilog code:

    module test(input x, y);
      (* keep *) foo #(.P(42), .Q(1337))
          foo_inst (.A({x, y}), .B({y, x}), .C({4'd10, {4{x}}}));
    endmodule

Translates to the following JSON output:

    {
      "modules": {
        "test": {
          "ports": {
            "x": {
              "direction": "input",
              "bits": [ 2 ]
            },
            "y": {
              "direction": "input",
              "bits": [ 3 ]
            }
          },
          "cells": {
            "foo_inst": {
              "hide_name": 0,
              "type": "foo",
              "parameters": {
                "Q": 1337,
                "P": 42
              },
              "attributes": {
                "keep": 1,
                "src": "test.v:2"
              },
              "connections": {
                "C": [ 2, 2, 2, 2, "0", "1", "0", "1" ],
                "B": [ 2, 3 ],
                "A": [ 3, 2 ]
              }
            }
          },
          "netnames": {
            "y": {
              "hide_name": 0,
              "bits": [ 3 ],
              "attributes": {
                "src": "test.v:1"
              }
            },
            "x": {
              "hide_name": 0,
              "bits": [ 2 ],
              "attributes": {
                "src": "test.v:1"
              }
            }
          }
        }
      }
    }

The models are given as And-Inverter-Graphs (AIGs) in the following form:

    "models": {
      <model_name>: [
        /*   0 */ [ <node-spec> ],
        /*   1 */ [ <node-spec> ],
        /*   2 */ [ <node-spec> ],
        ...
      ],
      ...
    },

The following node-types may be used:

    [ "port", <portname>, <bitindex>, <out-list> ]
      - the value of the specified input port bit

    [ "nport", <portname>, <bitindex>, <out-list> ]
      - the inverted value of the specified input port bit

    [ "and", <node-index>, <node-index>, <out-list> ]
      - the ANDed value of the specified nodes

    [ "nand", <node-index>, <node-index>, <out-list> ]
      - the inverted ANDed value of the specified nodes

    [ "true", <out-list> ]
      - the constant value 1

    [ "false", <out-list> ]
      - the constant value 0

All nodes appear in topological order. I.e. only nodes with smaller indices
are referenced by "and" and "nand" nodes.

The optional <out-list> at the end of a node specification is a list of
output portname and bitindex pairs, specifying the outputs driven by this node.

For example, the following is the model for a 3-input 3-output $reduce_and cell
inferred by the following code:

    module test(input [2:0] in, output [2:0] out);
      assign in = &out;
    endmodule

    "$reduce_and:3U:3": [
      /*   0 */ [ "port", "A", 0 ],
      /*   1 */ [ "port", "A", 1 ],
      /*   2 */ [ "and", 0, 1 ],
      /*   3 */ [ "port", "A", 2 ],
      /*   4 */ [ "and", 2, 3, "Y", 0 ],
      /*   5 */ [ "false", "Y", 1, "Y", 2 ]
    ]

Future version of Yosys might add support for additional fields in the JSON
format. A program processing this format must ignore all unknown fields.
\end{lstlisting}

\section{write\_simplec -- convert design to simple C code}
\label{cmd:write_simplec}
\begin{lstlisting}[numbers=left,frame=single]
    write_simplec [options] [filename]

Write simple C code for simulating the design. The C code written can be used to
simulate the design in a C environment, but the purpose of this command is to
generate code that works well with C-based formal verification.

    -verbose
        this will print the recursive walk used to export the modules.

    -i8, -i16, -i32, -i64
        set the maximum integer bit width to use in the generated code.

THIS COMMAND IS UNDER CONSTRUCTION
\end{lstlisting}

\section{write\_smt2 -- write design to SMT-LIBv2 file}
\label{cmd:write_smt2}
\begin{lstlisting}[numbers=left,frame=single]
    write_smt2 [options] [filename]

Write a SMT-LIBv2 [1] description of the current design. For a module with name
'<mod>' this will declare the sort '<mod>_s' (state of the module) and will
define and declare functions operating on that state.

The following SMT2 functions are generated for a module with name '<mod>'.
Some declarations/definitions are printed with a special comment. A prover
using the SMT2 files can use those comments to collect all relevant metadata
about the design.

    ; yosys-smt2-module <mod>
    (declare-sort |<mod>_s| 0)
        The sort representing a state of module <mod>.

    (define-fun |<mod>_h| ((state |<mod>_s|)) Bool (...))
        This function must be asserted for each state to establish the
        design hierarchy.

    ; yosys-smt2-input <wirename> <width>
    ; yosys-smt2-output <wirename> <width>
    ; yosys-smt2-register <wirename> <width>
    ; yosys-smt2-wire <wirename> <width>
    (define-fun |<mod>_n <wirename>| (|<mod>_s|) (_ BitVec <width>))
    (define-fun |<mod>_n <wirename>| (|<mod>_s|) Bool)
        For each port, register, and wire with the 'keep' attribute set an
        accessor function is generated. Single-bit wires are returned as Bool,
        multi-bit wires as BitVec.

    ; yosys-smt2-cell <submod> <instancename>
    (declare-fun |<mod>_h <instancename>| (|<mod>_s|) |<submod>_s|)
        There is a function like that for each hierarchical instance. It
        returns the sort that represents the state of the sub-module that
        implements the instance.

    (declare-fun |<mod>_is| (|<mod>_s|) Bool)
        This function must be asserted 'true' for initial states, and 'false'
        otherwise.

    (define-fun |<mod>_i| ((state |<mod>_s|)) Bool (...))
        This function must be asserted 'true' for initial states. For
        non-initial states it must be left unconstrained.

    (define-fun |<mod>_t| ((state |<mod>_s|) (next_state |<mod>_s|)) Bool (...))
        This function evaluates to 'true' if the states 'state' and
        'next_state' form a valid state transition.

    (define-fun |<mod>_a| ((state |<mod>_s|)) Bool (...))
        This function evaluates to 'true' if all assertions hold in the state.

    (define-fun |<mod>_u| ((state |<mod>_s|)) Bool (...))
        This function evaluates to 'true' if all assumptions hold in the state.

    ; yosys-smt2-assert <id> <filename:linenum>
    (define-fun |<mod>_a <id>| ((state |<mod>_s|)) Bool (...))
        Each $assert cell is converted into one of this functions. The function
        evaluates to 'true' if the assert statement holds in the state.

    ; yosys-smt2-assume <id> <filename:linenum>
    (define-fun |<mod>_u <id>| ((state |<mod>_s|)) Bool (...))
        Each $assume cell is converted into one of this functions. The function
        evaluates to 'true' if the assume statement holds in the state.

    ; yosys-smt2-cover <id> <filename:linenum>
    (define-fun |<mod>_c <id>| ((state |<mod>_s|)) Bool (...))
        Each $cover cell is converted into one of this functions. The function
        evaluates to 'true' if the cover statement is activated in the state.

Options:

    -verbose
        this will print the recursive walk used to export the modules.

    -stbv
        Use a BitVec sort to represent a state instead of an uninterpreted
        sort. As a side-effect this will prevent use of arrays to model
        memories.

    -stdt
        Use SMT-LIB 2.6 style datatypes to represent a state instead of an
        uninterpreted sort.

    -nobv
        disable support for BitVec (FixedSizeBitVectors theory). without this
        option multi-bit wires are represented using the BitVec sort and
        support for coarse grain cells (incl. arithmetic) is enabled.

    -nomem
        disable support for memories (via ArraysEx theory). this option is
        implied by -nobv. only $mem cells without merged registers in
        read ports are supported. call "memory" with -nordff to make sure
        that no registers are merged into $mem read ports. '<mod>_m' functions
        will be generated for accessing the arrays that are used to represent
        memories.

    -wires
        create '<mod>_n' functions for all public wires. by default only ports,
        registers, and wires with the 'keep' attribute are exported.

    -tpl <template_file>
        use the given template file. the line containing only the token '%%'
        is replaced with the regular output of this command.

[1] For more information on SMT-LIBv2 visit http://smt-lib.org/ or read David
R. Cok's tutorial: http://www.grammatech.com/resources/smt/SMTLIBTutorial.pdf

---------------------------------------------------------------------------

Example:

Consider the following module (test.v). We want to prove that the output can
never transition from a non-zero value to a zero value.

        module test(input clk, output reg [3:0] y);
          always @(posedge clk)
            y <= (y << 1) | ^y;
        endmodule

For this proof we create the following template (test.tpl).

        ; we need QF_UFBV for this proof
        (set-logic QF_UFBV)

        ; insert the auto-generated code here
        %%

        ; declare two state variables s1 and s2
        (declare-fun s1 () test_s)
        (declare-fun s2 () test_s)

        ; state s2 is the successor of state s1
        (assert (test_t s1 s2))

        ; we are looking for a model with y non-zero in s1
        (assert (distinct (|test_n y| s1) #b0000))

        ; we are looking for a model with y zero in s2
        (assert (= (|test_n y| s2) #b0000))

        ; is there such a model?
        (check-sat)

The following yosys script will create a 'test.smt2' file for our proof:

        read_verilog test.v
        hierarchy -check; proc; opt; check -assert
        write_smt2 -bv -tpl test.tpl test.smt2

Running 'cvc4 test.smt2' will print 'unsat' because y can never transition
from non-zero to zero in the test design.
\end{lstlisting}

\section{write\_smv -- write design to SMV file}
\label{cmd:write_smv}
\begin{lstlisting}[numbers=left,frame=single]
    write_smv [options] [filename]

Write an SMV description of the current design.

    -verbose
        this will print the recursive walk used to export the modules.

    -tpl <template_file>
        use the given template file. the line containing only the token '%%'
        is replaced with the regular output of this command.

THIS COMMAND IS UNDER CONSTRUCTION
\end{lstlisting}

\section{write\_spice -- write design to SPICE netlist file}
\label{cmd:write_spice}
\begin{lstlisting}[numbers=left,frame=single]
    write_spice [options] [filename]

Write the current design to an SPICE netlist file.

    -big_endian
        generate multi-bit ports in MSB first order
        (default is LSB first)

    -neg net_name
        set the net name for constant 0 (default: Vss)

    -pos net_name
        set the net name for constant 1 (default: Vdd)

    -nc_prefix
        prefix for not-connected nets (default: _NC)

    -inames
        include names of internal ($-prefixed) nets in outputs
        (default is to use net numbers instead)

    -top top_module
        set the specified module as design top module
\end{lstlisting}

\section{write\_table -- write design as connectivity table}
\label{cmd:write_table}
\begin{lstlisting}[numbers=left,frame=single]
    write_table [options] [filename]

Write the current design as connectivity table. The output is a tab-separated
ASCII table with the following columns:

  module name
  cell name
  cell type
  cell port
  direction
  signal

module inputs and outputs are output using cell type and port '-' and with
'pi' (primary input) or 'po' (primary output) or 'pio' as direction.
\end{lstlisting}

\section{write\_verilog -- write design to Verilog file}
\label{cmd:write_verilog}
\begin{lstlisting}[numbers=left,frame=single]
    write_verilog [options] [filename]

Write the current design to a Verilog file.

    -norename
        without this option all internal object names (the ones with a dollar
        instead of a backslash prefix) are changed to short names in the
        format '_<number>_'.

    -renameprefix <prefix>
        insert this prefix in front of auto-generated instance names

    -noattr
        with this option no attributes are included in the output

    -attr2comment
        with this option attributes are included as comments in the output

    -noexpr
        without this option all internal cells are converted to Verilog
        expressions.

    -siminit
        add initial statements with hierarchical refs to initialize FFs when
        in -noexpr mode.

    -nodec
        32-bit constant values are by default dumped as decimal numbers,
        not bit pattern. This option deactivates this feature and instead
        will write out all constants in binary.

    -decimal
        dump 32-bit constants in decimal and without size and radix

    -nohex
        constant values that are compatible with hex output are usually
        dumped as hex values. This option deactivates this feature and
        instead will write out all constants in binary.

    -nostr
        Parameters and attributes that are specified as strings in the
        original input will be output as strings by this back-end. This
        deactivates this feature and instead will write string constants
        as binary numbers.

    -extmem
        instead of initializing memories using assignments to individual
        elements, use the '$readmemh' function to read initialization data
        from a file. This data is written to a file named by appending
        a sequential index to the Verilog filename and replacing the extension
        with '.mem', e.g. 'write_verilog -extmem foo.v' writes 'foo-1.mem',
        'foo-2.mem' and so on.

    -defparam
        use 'defparam' statements instead of the Verilog-2001 syntax for
        cell parameters.

    -blackboxes
        usually modules with the 'blackbox' attribute are ignored. with
        this option set only the modules with the 'blackbox' attribute
        are written to the output file.

    -selected
        only write selected modules. modules must be selected entirely or
        not at all.

    -v
        verbose output (print new names of all renamed wires and cells)

Note that RTLIL processes can't always be mapped directly to Verilog
always blocks. This frontend should only be used to export an RTLIL
netlist, i.e. after the "proc" pass has been used to convert all
processes to logic networks and registers. A warning is generated when
this command is called on a design with RTLIL processes.
\end{lstlisting}

\section{write\_xaiger -- write design to XAIGER file}
\label{cmd:write_xaiger}
\begin{lstlisting}[numbers=left,frame=single]
    write_xaiger [options] [filename]

Write the top module (according to the (* top *) attribute or if only one module
is currently selected) to an XAIGER file. Any non $_NOT_, $_AND_, $_ABC9_FF_, ornon (* abc9_box_id *) cells will be converted into psuedo-inputs and
pseudo-outputs. Whitebox contents will be taken from the '<module-name>$holes'
module, if it exists.

    -ascii
        write ASCII version of AIGER format

    -map <filename>
        write an extra file with port and box symbols
\end{lstlisting}

\section{xilinx\_dffopt -- Xilinx: optimize FF control signal usage}
\label{cmd:xilinx_dffopt}
\begin{lstlisting}[numbers=left,frame=single]
    xilinx_dffopt [options] [selection]

Converts hardware clock enable and set/reset signals on FFs to emulation
using LUTs, if doing so would improve area.  Operates on post-techmap Xilinx
cells (LUT*, FD*).

    -lut4
        Assume a LUT4-based device (instead of a LUT6-based device).
\end{lstlisting}

\section{xilinx\_dsp -- Xilinx: pack resources into DSPs}
\label{cmd:xilinx_dsp}
\begin{lstlisting}[numbers=left,frame=single]
    xilinx_dsp [options] [selection]

Pack input registers (A2, A1, B2, B1, C, D, AD; with optional enable/reset),
pipeline registers (M; with optional enable/reset), output registers (P; with
optional enable/reset), pre-adder and/or post-adder into Xilinx DSP resources.

Multiply-accumulate operations using the post-adder with feedback on the 'C'
input will be folded into the DSP. In this scenario only, the 'C' input can be
used to override the current accumulation result with a new value, which will
be added to the multiplier result to form the next accumulation result.

Use of the dedicated 'PCOUT' -> 'PCIN' cascade path is detected for 'P' -> 'C'
connections (optionally, where 'P' is right-shifted by 17-bits and used as an
input to the post-adder -- a pattern common for summing partial products to
implement wide multipliers). Limited support also exists for similar cascading
for A and B using '[AB]COUT' -> '[AB]CIN'. Currently, cascade chains are limited
to a maximum length of 20 cells, corresponding to the smallest Xilinx 7 Series
device.

This pass is a no-op if the scratchpad variable 'xilinx_dsp.multonly' is set
to 1.


Experimental feature: addition/subtractions less than 12 or 24 bits with the
'(* use_dsp="simd" *)' attribute attached to the output wire or attached to
the add/subtract operator will cause those operations to be implemented using
the 'SIMD' feature of DSPs.

Experimental feature: the presence of a `$ge' cell attached to the registered
P output implementing the operation "(P >= <power-of-2>)" will be transformed
into using the DSP48E1's pattern detector feature for overflow detection.

    -family {xcup|xcu|xc7|xc6v|xc5v|xc4v|xc6s|xc3sda}
        select the family to target
        default: xc7
\end{lstlisting}

\section{xilinx\_srl -- Xilinx shift register extraction}
\label{cmd:xilinx_srl}
\begin{lstlisting}[numbers=left,frame=single]
    xilinx_srl [options] [selection]

This pass converts chains of built-in flops (bit-level: $_DFF_[NP]_, $_DFFE_*
and word-level: $dff, $dffe) as well as Xilinx flops (FDRE, FDRE_1) into a
$__XILINX_SHREG cell. Chains must be of the same cell type, clock, clock polarity,
enable, and enable polarity (where relevant).
Flops with resets cannot be mapped to Xilinx devices and will not be inferred.
    -minlen N
        min length of shift register (default = 3)

    -fixed
        infer fixed-length shift registers.

    -variable
        infer variable-length shift registers (i.e. fixed-length shifts where
        each element also fans-out to a $shiftx cell).
\end{lstlisting}

\section{zinit -- add inverters so all FF are zero-initialized}
\label{cmd:zinit}
\begin{lstlisting}[numbers=left,frame=single]
    zinit [options] [selection]

Add inverters as needed to make all FFs zero-initialized.

    -all
        also add zero initialization to uninitialized FFs
\end{lstlisting}



\include{CHAPTER_Appnotes}
% \include{CHAPTER_StateOfTheArt}

\bibliography{literature}
\bibliographystyle{alphadin}

\bibliographyweblink{weblinks}
\bibliographystyleweblink{abbrv}

\end{document}
